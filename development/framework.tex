\section{Software framework}\label{sec:framework}
In order to answer the questions proposed in the thesis while adhering to the design principles outlined we decided to utilize ideas from previous researchers within the field of educational robotics. 
One of the eariest researchers looking into how robotics could be used in schools and one of the creators of the Logo programming language were Seymour Papert (\cite{papert1980mindstorms}). Who has several decades of experience with using robots as a tool in education.
We therefore felt it natural to lean on this preexisting knowledge, and utilize the basis of Logo and its successors for our thesis. We felt this would give us a solid pedagogical fundation to start working with the application, however the complexity of the Logo language was overkill for this project. In order to mitigate this we propose a system where parts of the programming language can be disabled to make it easier for the participants to understand and grasp. In addition to basing much of the language of Logo we drew inspiration from MIT's scratch program\footnote{\url{http://scratch.mit.edu/about/}}, which is a programming tool based on predefined codeblocks in order to remove the need for writing the correct syntax to make your program compile. By utilizing a similar strategy to MIT would could allow the participants to create their program through a fully interactive graphics interface by dragging and dropping codeblocks into position. 

\bigskip\noindent
We proposed to create the system as a two component system, the \chirp robot and a programming device. 
From the get-go we experience trouble in using the arduino integrated developer environment(AIDE), and decided that using it as the programming device would not work.
Previous work with the robot had however shown that bluetooth was a viable option for communication between a mobile device and the robot(\cite{chrip2013ResearcherNight}).
A solution based on bluetooth communication would allow the system to be vastly more usable for non-technical people and children.
It was therefore decided that the system would use tablets(android, iPad, ect) as its programming device.

\bigskip\noindent
As the system was created to be used in an educational setting all the way down to lower parts of our educational system, 
we needed to make it easy to use for everyone, while still maintaining the complexity needed for children of an higher age and/or skillset. 
We wanted to create a system that could be used to teach fourth graders maths, and still be used to teach tenth grader at the same time. 
This reinforced the need for the system to be extendable, but also have an easy to use user interface. 
	
\bigskip\noindent
	The first decision to be made were regarding the platform and how to achieve a crossplatform solution. 
	A native approach was quickly discarded as it would require multiple implementations in multiple languages (E.g. Objective-C for iOS devices, and Java for android devices, etc.). 
	Removing the native approach from our solution pool still left us with several options, including game frameworks and standardized cross platform tools.
	In a report by vision mobile in January 2013 they present a study involving six hundred developers and their preferences (\cite{developerCPT}). 
	In addition to the top five most used frameworks from vision mobile's report and some newer frameworks, we initially included game frameworks into the pool of frameworks. 
	The game frameworks were however discarded relativly fast as they rely heavily on custom built user interfaces and low level programming(openGL/webGL) for performance reasons.
	
	\bigskip\noindent
	After removing the game frameworks from our solution pool we looked throught the remaining options, PhoneGap(also known as Apache Cordova), Appcelerator, Adobe AIR, Sencha, Qt, DevExpress and Xamarin.
	
	\smalltable{Overview over potensial cross platform solutions}{table:crossplatform}{
		\begin{tabular}{lll}
			Framework & Prog. lang. & Notes\\\hline
			Apache Cordova & Javascript and HTML & \wrap{Free of charge under the apache 2 license}{0.45}\\
			Appcelerator &  Javascript & \wrap{High abstraction level. Not free to use in a commercial setting}{0.45}\\
			Adobe AIR & Flash and ActionScript & \wrap{}{0.45}\\
			Sencha & Javascript & \wrap{Commercial licensing}{0.45}\\
			Qt & C++ & \wrap{Pay to use}{0.45}\\
			DevExpress & Javascript and HTML & \wrap{Pay to use. Based on Apache Cordova technology.}{0.45}\\
			Xamarin & C\# & \wrap{Free starter pack, but have usage restrictions}{0.45}\\\hline
		\end{tabular}
	}
	
	\bigskip\noindent
	The framwork that had commercial licenses or a "`pay to use"' business model were removed from the list of potensial frameworks, which left us with the choice between Apache Cordova, Adobe AIR and Xamarin.
	We decided to move forward with PhoneGap due to language preferences in the team. Adobe AIR, while stating that one can use javascript, does in fact require the use of Flash and ActionScript for its mobile applications, and the we had very little experience with C\#.
	Apache Cordova on the other hand is build around standard web technologies as javascript, HTML and CSS, which both members of the project have been working with before.
	
	