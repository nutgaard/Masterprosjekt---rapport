\section{Tools and application}
	\subsection{Requirements}
In order to answer the questions proposed in the thesis we decided to utilize ideas from previous researchers within the field of educational robotics. 
One of the eariest researchers looking into how robotics could be used in schools and one of the creators of the Logo programming language were Seymour Papert~\cite{papert1980mindstorms}. Who has several decades of experience with using robots as a tool in education.
We therefore felt it natural to lean on this preexisting knowledge, and utilize the basis of Logo and its successor for our thesis.

\bigskip\noindent
In order to create the system we needed a two component system, the \chirp robot and a programming device. 
From the get go we experience the troubles of using the arduino integrated developer environment(AIDE), and decided that using it as the programming device would not work.
Previous work with the robot had however shown that bluetooth was a viable options~\cite{chrip2013ResearcherNight}.
A solution which would allow the system to be vastly more usable for non-technical people and children. 
It was therefore decided that the system would use tablets(android, iPad, ect) as its programming device.
As a result we decided to make the system platform independent in order to make it more accessible for everyone. 

\bigskip\noindent
This project and thesis is pilot project for the use of the \chirp robot in schools, as such,
one of the goals for this project was to clear the way for future research using the \chirp robot.
It was therefore important that the system we created could serve such a purpose.
Extendability and modifiability were therefore key concepts that were needed when designing the system.

\bigskip\noindent
As the system was created to be used in an educational setting all the way down to lower parts of our educational system, 
we needed to make it easy to use for everyone, while still maintaining the complexity needed for children of an higher age and/or skillset. 
We wanted to create a system that could be used to teach fourth graders maths, and still be used to teach tenth grader at the same time. 
This reinforced the need for the system to be extendable, but also have an easy to use user interface. 


\bigskip\noindent
Based on this we created a set of requirements that our system had to adhere to:


\bigskip\noindent
\begin{tabular}{l|l|l}
	\textbf{\#} & \textbf{Name} & \textbf{Description}\\
	\hline\hline
	R1 & Extendability & \wrap{The system should be easy to extend with new functionality.}{}\\\hline
	R2 & Modifiability & \wrap{The system should be easy to modify. Settings/options should be included to allow for changes at runtime.}{}\\\hline
	R3 & Usability & \wrap{The system should provide an easy to use user interface.}{}\\\hline
	R4 & Platform & \wrap{The system should work on as many as possible platform.}{}\\
\end{tabular}

\subsection{Design decision}
	