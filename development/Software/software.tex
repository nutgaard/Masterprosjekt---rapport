\section{Software requirements}
One of the many goals of this project was to clear the way for future work with the \chirp robot in an educational setting. In order for us to achieve this goal we were determined to create a system which would not only work for our current setting, but that could be extended and/or modified to fulfill other purposes as well.

\bigskip\noindent
In addition to make the system easy to extend and modify we had to enforce a high degree of usability throughout the application, this was mainly due to the fact that were going to use the application alongside children. This got us into thinking about utilizing bluetooth as a means to program the robot as using the arduino development environment was tedious and errorprone in various ways. 


\bigskip\noindent
As utilizing bluetooth became a viable options for programming the robot we determined that a mobile solution using already existing smartphones and/or tablets would probably the easiest solution to use for all parties involved. This did however pose another requirement on the system as we did not want to create a vendor specific solution, e.g. forcing the use of android, iphone or windows phone on the users. 
This can also be seen as a continuation of one of the design goals when the \chirp robot was created, as it was meants as a cheaper alternative to other existing robots. 


In order to answer the questions proposed in the thesis we decided to utilize ideas from previous researchers within the field of educational robotics. 
One of the eariest researchers looking into how robotics could be used in schools and one of the creators of the Logo programming language were Seymour Papert~\cite{papert1980mindstorms}. Who has several decades of experience with using robots as a tool in education.
We therefore felt it natural to lean on this preexisting knowledge, and utilize the basis of Logo and its successors for our thesis. 

\bigskip\noindent


\bigskip\noindent
In order to create the system we needed a two component system, the \chirp robot and a programming device. 
From the get go we experience the troubles of using the arduino integrated developer environment(AIDE), and decided that using it as the programming device would not work.
Previous work with the robot had however shown that bluetooth was a viable options~\cite{chrip2013ResearcherNight}.
A solution which would allow the system to be vastly more usable for non-technical people and children. 
It was therefore decided that the system would use tablets(android, iPad, ect) as its programming device.
As a result we decided to make the system platform independent in order to make it more accessible for everyone. 

\bigskip\noindent
This project and thesis is pilot project for the use of the \chirp robot in schools, as such,
one of the goals for this project was to clear the way for future research using the \chirp robot.
It was therefore important that the system we created could serve such a purpose.
Extendability and modifiability were therefore key concepts that were needed when designing the system.

\bigskip\noindent
As the system was created to be used in an educational setting all the way down to lower parts of our educational system, 
we needed to make it easy to use for everyone, while still maintaining the complexity needed for children of an higher age and/or skillset. 
We wanted to create a system that could be used to teach fourth graders maths, and still be used to teach tenth grader at the same time. 
This reinforced the need for the system to be extendable, but also have an easy to use user interface. 


\bigskip\noindent
Based on this we created a set of requirements that our system had to adhere to:


\bigskip\noindent
\begin{tabular}{l|l|l}
	\textbf{\#} & \textbf{Name} & \textbf{Description}\\
	\hline\hline
	R1 & Extendability & \wrap{The system should be easy to extend with new functionality.}{}\\\hline
	R2 & Modifiability & \wrap{The system should be easy to modify. Settings/options should be included to allow for changes at runtime.}{}\\\hline
	R3 & Usability & \wrap{The system should provide an easy to use user interface.}{}\\\hline
	R4 & Platform & \wrap{The system should work on as many as possible platform.}{}\\
\end{tabular}

\subsection{Design decision}
	In order to accomplish the requirements outlined we went through each one of them in turn, and tried to come up with a strategy that would 
	to create the best possible solution.
	
	\bigskip\noindent
	The first decision to be made were regarding the platform and how to achieve a crossplatform solution. 
	A native approach was quickly discarded as it would require multiple implementations in multiple languages (E.g. Objective-C for iOS devices, and Java for android devices, etc.). 
	Removing the native approach from our solution pool still left us with several options, including game frameworks and standardized cross platform tools.
	In a report by vision mobile in January 2013 they present a study involving six hundred developers and their preferences~\cite{developerCPT}. 
	In addition to the top five most used frameworks from vision mobile's report and some newer frameworks, we initially included game frameworks into the pool of frameworks. 
	The game frameworks were however discarded relativly fast as they rely heavily on custom built user interfaces and low level programming(openGL/webGL) for performance reasons.
	
	\bigskip\noindent
	After pruning the game frameworks from our solution pool we looked throught the remaining options, PhoneGap(Cordova), Appcelerator, Adobe AIR, Sencha, Qt, DevExpress and Xamarin.
	Qt, Sencha, DevExpress, Xamarin and Appcelerator was removed from the pool as they all costs money in order to use the full frameworks commercially and to a large extend focuses on more enterprise applications.
	This left us with Adobe AIR and PhoneGap as possible frameworks. 
	We decided to move forward with PhoneGap\footnote{Actually Apache Cordova, which is practicly the same but provided with an Apache licence.}
	due to language preferences. Adobe AIR, while stating that one can use javascript, does in fact require the use of Flash and ActionScript for its mobile applications.
	Apache Cordova on the other hand is build around standard web technologies as javascript, HTML and CSS.
	
	
	
	
	
	
	
	