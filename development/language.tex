\chapter{Choice of robot programming language}
When choosing what programming language we should control the robot with we had 4 criteria. The language must utilize click and drag, to be easy to use on a tablet. It must provide an intuitive way to control the robots and reduce the need for programming understanding, as this would take time to explain and are not necessary to teach mathematics. It must be able to function cross platform. The 4th criteria was that the program must compile to Arduino Micro, which is the hardware chip that controls the robot we are using. We experienced a problem with this last criteria. In order to function cross platform, compiling the program on the ChIRP is not an option. It takes a tremendous effort to connect an Arduino to a tablet, and even if this is managed, the solution is different for every tablet. The only solution we found was to interpret commands on the ChIRP and send these commands through a bluetooth connection. The 4th criteria therefore states the programming language must be able to send commands, and not compile the program, as was the original idea.

\bigskip\noindent
There exists several programming languages that are created for educational purposes. ROBOLAB is one alternative and is used in several studies. ROBOLAB is specifically created to control Lego Mindstorms, an educational robot created by Lego. There also exist alternatives to ROBOLAB which can program other robots than Lego Mindstorms. However, almost none of these programs are compatible with Arduino micro. The ones that are violates the 4th criteria, as they only compile their programs. This is a problem with all other programming languages as well and it became clear that we must create our own programming interface. We can still use the syntax from other programming languages, and there was no reason to create our own programming language, especially since there has been done much research with on languages in a mathematics context.

\bigskip\noindent
The choice came down to Scratch and LOGO. Scratch is a multimedia authoring tool that is created for educational purposes. It utilizes click and drag , it abstracts away hard programming concepts such as syntax and we could have created a similar version to interpret on the Arduino. However it has to many functions and contains many traditional programming functionalities that are hard to avoid. It is not particularly well suited for robots either. There has not been done much research on scratch as it first appeared in 2006.

\bigskip\noindent
LOGO is an educational programming language created for robotics. It was the first programming language created for such purposes by \citeauthor{papert1980mindstorms} and his team at MIT. It is discussed in in detail in \cite{papert1980mindstorms}. LOGO is made for learning and it is very easy to learn. It was made to control a ``screen turtle'' and a turtle robot, which draws lines where it moves. It is therefore provides an intuitive way to control a robot's movements. The language contains commands such as fwd and back to make the robot move forward and backwards respectively. To make the turtle turn, there exist 2 commands, turn left and turn right. Both need an amount of degrees as a parameter, thus learning about degrees and angles is one of the most basic concepts to teach through this language. Programming in this way abstracts away the programming part. In LOGO we use a block called fwd for forward, where in a usual programming language you would have to write forward(); for example. This can create confusion and you would have to explain what the parenthesis and semicolon are for. It is also easy to make just a subset of elements available, without limiting what the program can do. This is done to avoid confusion and keep the programs simple, and easy to reason about. In this way we can put the emphasis on mathematics and not use too much time explaining the programming part of it. Originally it was implemented as a writing programming language that compiles. However as we are implementing it ourselves we can fill in the missing criteria.

\bigskip\noindent
There has been done a lot of research on learning mathematics with LOGO. The first was as we mentioned earlier \cite{papert1980mindstorms}. \citeauthor{clements2001logo} is also a central researcher when it comes to mathematics and LOGO. He has published several papers explaining the pedagogical aspects of LOGO \cite{clements2001logo, clements1990effects, clements1996development, clements1993research}. We contacted him and asked what he thought about using LOGO to test the difference between a robot and a simulator. He answered ``An excellent idea! I'm so glad you're doing this.''. In addition there are several books published discussing existing research on LOGO in mathematics concluding that LOGO is an effective tool for teaching mathematics. There also exists many books on example curriculum to teach different mathematics concepts through LOGO \cite{hoyles1992learning}. Overall research strongly suggest using LOGO in mathematics education \cite{jones2005using}. For us, LOGO is therefore also an excellent choice because using this programming language provides a platform which can easily be expanded to teach other mathematics concepts and do different experiments as well. 

\bigskip\noindent
We chose to use the LOGO syntax and semantics. The missing criteria in this language are: cross platform, click and drag and command sending instead of compilation. To satisfy our criteria we had to expand LOGO with some functionality. In this process we got inspiration from the Scratch programming language. We use programming blocks instead of writing the commands on the keyboard. This also enable us to easily satisfy the click and drag constraint. We chose to send the commands one by one to a pre-compiled interpreter program on the Arduino, thus satisfying all the criteria we had put on our language.
