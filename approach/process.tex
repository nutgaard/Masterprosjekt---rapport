\section{Execution of project}
This entire project was conducted from February to July 2014. During this time period we've created a pilot-project looking into the use of robotics in an educational setting with a focus on angles and how these can be taught.
All we have done leads up to a pilot test at the Trondheim International School (THIS), where we tested our hypothesis and how well suited the \chirp robot was in an educational setting. 
The contact established between the \chirp team and the school will hopefully help foster further research within educational robotics from the \chirp team.
%This chapter will provide a walkthrough of the different phases of this project. 
%
%This chapter will highlight some of the tasks needed to conduct this project. 
%Everything we have done has been done since february. This made our process long and intense. We have conducted a pre-project to find a relevant angle on robotics in education and then we have focused further into the specific field of robotics in maths education.

\paragraph{Literature Reviews}~\\	
In order to understand educational robotics state of the art we performed a literature study into educational robotics with a focus on mathematics. The setup for this review can be read in section~\ref{ch:literatureProcess}.
The applicability of robotics in a mathematics setting has proven itself over and over again. The results from the literature study lead us to focus more specifically on angles instead of mathematics in general.
We have chosen to focus on angles as it is one of the core concepts within mathematics, students tend to struggle with it, and visualization of angles can be achieved with the help of a robot.
After the state of the art literature review we identified the need for an additional literature study. This study was conducted in order to learn more about the current pedagogical theories and test setups, which would help us find a suitable approach to create a solid platform for this project and potential successors. 
%How we approached is described below. 
%We also want to get an idea of what way we can introduce robotics to a school.
%As a part of defined the goals and research questions for this project we conducted a literature review. 
In this chapter we will first present the process, planning and results of the review. Then to aid the reader we've included a short introduction to the theory of constructionism, as this was used in the majority of research within education robotics. 
Finally, in section~\ref{sec:methodsss}, we'll provide an introduction to different quantitative and qualitative methods availble to use in an experimental setting. 
%The process and planning of the review is documented in this chapter, whereas the concluding thoughs and results of the review is found in chapter~\ref{ch:literatureConclusion}. 
%In order to get proper research questions and state of the art of educational robotics we performed a literature study. 
%Here is how we conducted this. 

%After finding a lot of papers relevant to these questions, we analyzed the content and purpose of each study.
%The reviewed articles suggest that educational robotics usually results in an increase of content knowledge and a more positive attitude towards STEM. 
%Though there were also studies that reported that educational robotics did not yield any tangiable results. 
%The studies that reported negative (no increase in content knowledge) do however form the minorty out of all the studies reviewed. 
	
%\paragraph{Method}~\\
\section{Review setup}\label{ch:literatureProcess}
The review was conducted in line with the guidelines drafted by Kitchenham's and Khan's guide for writing a systematic review~\cite{kitchenham2007guidelines,khan2001undertaking}. 
%The review was completed as several phases, based upon the Kitchenham's \cite{kitchenham2007guidelines} and Khan's~\cite{khan2001undertaking} guide for writing systematic reviews. 
The initial steps of identifying a need for the review and commissioning a review was however omitted. 
The phases used during the review was; 
%The steps followed is written down below:
\begin{description}
	\item[Phase 1: ] Planning
		\begin{enumerate}
			\item Specifying the research question(s)
			\item Developing a review protocol
			\item Evaluating the review protocol
		\end{enumerate}
	\item[Phase 2: ] Conducting the review
		\begin{enumerate}
			\item Identification of research
			\item Selection of primary studies
			\item Study quality assessment
			\item Data extaction and monitoring
			\item Data synthesis
		\end{enumerate}
	\item[Phase 3: ] Reporting the review
		\begin{enumerate}
			\item Communicating the results through a report.
		\end{enumerate}
\end{description}

%\paragraph{Planning the review}\label{sec:questions}~\\
\bigskip\noindent
Initially, we preformed an general search into educational robotics, in order to obtain some fundemental knowledge regarding the subject. In this search we stumbled upon another review written in 2011 by Fabiane Barreto Vavassori Benitti\cite{Benitti2012978}, which to a very large extend covers the same topic as initially planned by this review. There was however some minor differences between our research question, but the papers relevant to Benitti proved to a large extend to be relevant for our review as well. 

\bigskip\noindent
Where Benitti asked general questions like "`\textit{What topics are taught through robotics in school}s?"', "`\textit{is robotics an effective tool for teaching? What do the studies show?}"', and "`\textit{How is student learning evaluated?}"'. We focused on narrowing down the scope of the review, limiting the review to research related to math and how math can be taught in schools using robotics.
%While this review is going to use more narrow question, limiting the research to math and how math can be taught in schools using robotics.
The research questions used for this review is stated below; 
\begin{description}
	\item[Question 1: ] How well did the different mathematics concepts get taught with robotics?
	\item[Question 2: ] What advantages or disadvantages except learning gains are there?
	\item[Question 3: ] Were any suggested improvements to the experiments identified?
	\item[Question 4: ] What lacks of research are there in the literature?
\end{description}

\bigskip\noindent
The review was conducted in February and March of 2014, with papers retrieved from all the major bibligraphic databases, including, but not limited to, CiteSeer, ACM Digital Library, SpringerLink, ERIC, IEEE XPLORE, Wiley Inter Science, and ScienceDirect. In addition to these databases the search was applied to the google scholars search engine to ensure that we got most of the relevant papers.

\bigskip\noindent
The general search query was created using groups of synonyms, concatenated by the \texttt{and}/\texttt{or} operators before adjustmentint the query to each unique database (e.g making sure the search query was compatible with the search engine at any given site). The search query used in this review was: \texttt{(math or stem or mathematics) and (education or learn or learning or educational or teach or teaching) and (robot or robotics or robots) and (school or k-12)}. 

\bigskip\noindent
To remove papers that not were of any interest we first applied a set of inclusion criteria, before applying a set of exclusion criteria. In order for a paper to be included in the review it had to pass all inclusion criteria, and not violate any of the exclusion criteria. 

\begin{description}
	\item[IC1] The purpose of the paper is to investigate the usage of robotics in school, where the goal is not to teach about robotics itself.
	\item[IC2] The paper should contain some sort of assessment, quantitative or qualitative, of the learning outcome and/or experiences from the study. 
	\item[IC3] The assessment must address the development of math skills. 
	\item[IC4] The study should be done in an elementary, middle or highschool context.
	\item[IC5] The study should involve the use of physical robots.
	\item[EC1] Article does not address the subject of using robotics to teach maths. 
	\item[EC2] Article does not address, or refer to, any pedagogical foundation. 
\end{description}
These criteria diverge from Benitti's review in that qualitative assessments also are included. 
We justify this by acknowledging the fact that non-immediate returns of educational robotics may be equally important to immediate curricular related returns,
and to reflect and investigate this we allow qualitative research to take part of this review. 	

%\bigskip\noindent
%By negating the inclusion criteria above we get a hold of the exclusion criteria used for this review. 
%The only criteria which does not have any clear negated form is IC2, we therefore define EC2 to be "`does not include any form of assessment in the form of a study"'. 




%Another lack in research is an explanation of how they taught the students. They may mention that they implemented a 1 year curriculum but don't provide any examples of tasks they created. It is therefore very hard to build upon and do further work on it. 


%\paragraph{Deciding on a specific topic}\label{sec:decideTopic}~\\

%A lot of topics can be learned using robotics. We have focused on mathematics. It is said that all math can be taught in some way with robotics. We present examples for some of them.

%\bigskip\noindent
 %with a robot can be achieved without to much trouble. 
%We have chosen to focus on angles as it is very basic, people struggle with it and using logo uses angles to decide how much to turn. It is therefore a natural starting point for our investigation in mathematics with robotics. 
%
%\bigskip\noindent
%As a pedagogical foundation we looked to research conducted by Seymour Papert, who is known for his early contributions to the field of educational robotics~\cite{papert1980mindstorms,papertGrant}.
%We also draw pedagogical reasoning behind angles and geometry from papert. The reason angles and geometry in general is easier is that a point rather than being defined as a place in space with no heading the robot that serves as the point here has a heading just like us humans. It can therefore be related to. 

%\section{Literature Study on available programming languages and pedagogy}
%\paragraph{Secondary Literature Study}~\\
%After deciding to investige the concept of angles we identified the need for a secondary literature study. 
%This study was conducted in order to learn more about the current pedagogical theories and test setups, which would help us find a suitable approach to create a solid platform for this project and potensial successors.

%Avsnittet virker som om det handler mer om det første literature studiet.
%In order to focus on the right points, learn about the pedagogic theory and testing setup, use a suitable approach and create a solid platform to work on we needed to conduct a literature review to find state of the art on educational robotics as well as pedagogic theory. 

\paragraph{Design and Development}~\\
In order to conduct this project we needed to develop several accessories for the \chirp and a tablet application for testing in an elementary school. 
Since this was considered a pilot-project for the \chirp robot we aimed at designing and developing a prototype with emphasis on expandability, modifiability and portability. 
%Vet ikke helt hvordan denne skal uttrykkes, eller den i det hele tatt bør bli med?
%We will find workarounds for the chirp that does not require major modifications to the chassis or default circuit board with sensors. 
We also needed to develop a small circuit to enable bluetooth communication between the robot and tablet application. 
All in all there was a lot of work to be done before a prototype would be ready for the experiment. 
%Tror ikke vi trenger å skrive så mye begrunnelser her, siden det meste kommer til å bli nevnt i design and development kapitlet. 
%as we found that compiling and uploading code to the ChIRP is not suitable for every school grade and we don't know which grade we will be working with, another point is that this application again should be easily expandable and implementable for different grade students.  

%Igjen, tror ikke vi trenger begrunnelse her. 
%\bigskip\noindent
%The design and development of this software and hardware will be a very demanding phase. Designing a new application from scratch can be very time-consuming([39] - Another master %.Mention hardware too REFERANSE). However we build upon LOGO so we have some guidance to what the functionality should entail. 

%Again we will not have time to create an optimal solution as this would require testing and then redesign etc. We aim to create an easily manageable and changeable platform to build upon in future research. 

\paragraph{Experiment}~\\
To answer our research questions we had to test the robot in a school context. We contacted several schools, had meetings with Ungt Entreprenørskap and other researchers at IDI responsible for attracting students to study at IDI, all of whom were interested in the project. The time frame of the project did however not fit within their existing schedules. Our supervisor (Pauline) had previously been in contact with Trondheim International School (THIS), and they were very enthusiastic about the project, very helpful throughout the whole project, and we were lucky to be able to conduct our experiment with their students. 
We had 2 meetings with the teacher of the class we were assigned before the experiment itself. In these meetings we discussed our tests and curriculum and got feedback which we used to improve our experiment. It was important for us to include teachers as the end goal is to integrate the robotic activities to normal schools.
The experiments were run at the end of the school year because of the long process of finding a school and developing the system. This resulted in a smaller experimental time span than initially intended.

