\chapter{Process}
This entire project was conducted from February to July 2014. During this time period we've created a pilot-project looking into the use of robotics in an educational setting with a focus on angles and how these can be taught.
This chapter will highlight some of the tasks needed to conduct this project. 
%Everything we have done has been done since february. This made our process long and intense. We have conducted a pre-project to find a relevant angle on robotics in education and then we have focused further into the specific field of robotics in maths education.

\section{Literature Study on robotics in education}
In order to get an idea of the state of the art of robotics in education we initially performed a literature study. The results from this study can be read in chapter~\ref{ch:literatureProcess} through \ref{ch:literatureConclusion}.
%We also want to get an idea of what way we can introduce robotics to a school.

\bigskip\noindent
The literature study showed great promise for robotics in education. However few papers have cooperated with schools and pedagogs in an attempt to integrate or check how their ideas will integrate into the school curriculum. Many mention that educational robotics requires a change in the traditional curriculum in order to be more consistent with the constructionism theories. This does, in our opinion, pose one of the greatest barriers for educational robotics.
%However calling for a huge change in schools is probably not the easiest way to introduce robotics to schools if this require lots of alterations. 

\bigskip\noindent
One major denominator in the current research is the lack of details in publications. While general trends and statistics are included in the majority of publications, we still found that the discussions related to validity and reliability often were excluded. This made our process of determining the current state of educational robotics extremely difficult. 
%Another lack in research is an explanation of how they taught the students. They may mention that they implemented a 1 year curriculum but don't provide any examples of tasks they created. It is therefore very hard to build upon and do further work on it. 


\section{Deciding on a specific topic}\label{sec:decideTopic}
The applicability of robotics in an educational setting has proven itself to be extremely big. Thus, in order to narrow down the project, we chose to focus on mathematics. 
%A lot of topics can be learned using robotics. We have focused on mathematics. It is said that all math can be taught in some way with robotics. We present examples for some of them.

\bigskip\noindent
We have chosen to focus on angles as it is one of the core concepts within mathematics, people tend to struggle with it, and visualization of angles with a robot can be achieved without to much trouble. 
%We have chosen to focus on angles as it is very basic, people struggle with it and using logo uses angles to decide how much to turn. It is therefore a natural starting point for our investigation in mathematics with robotics. 

\bigskip\noindent
As a pedagogical foundation we looked to research conducted by Seymour Papert, who is known for his early contributions to the field of educational robotics.
%We also draw pedagogical reasoning behind angles and geometry from papert. The reason angles and geometry in general is easier is that a point rather than being defined as a place in space with no heading the robot that serves as the point here has a heading just like us humans. It can therefore be related to. 

%\section{Literature Study on available programming languages and pedagogy}
\section{Secondary Literature Study}
After deciding to investige the concept of angles we identified the need for a secondary literature study. 
This study was conducted in order to learn more about the current pedagogical theories and test setups, which would help us find a suitable approach to create a solid platform for this project and potensial successors.

%Avsnittet virker som om det handler mer om det første literature studiet.
%In order to focus on the right points, learn about the pedagogic theory and testing setup, use a suitable approach and create a solid platform to work on we needed to conduct a literature review to find state of the art on educational robotics as well as pedagogic theory. 

\section{Design and Development}
In order to conduct this project we needed to develop several accessories for the \chirp and a tablet application for testing in an elementary school. 
Since this was considered a pilot-project for the \chirp robot we aimed at designing and developing a prototype with emphasis on expandability. 
%Vet ikke helt hvordan denne skal uttrykkes, eller den i det hele tatt bør bli med?
%We will find workarounds for the chirp that does not require major modifications to the chassis or default circuit board with sensors. 
We also needed to develop a small circuit to enable bluetooth communication between the robot and tablet application. 
All in all there was a lot of work to be done before a prototype would be ready for the experiment. 
%Tror ikke vi trenger å skrive så mye begrunnelser her, siden det meste kommer til å bli nevnt i design and development kapitlet. 
%as we found that compiling and uploading code to the ChIRP is not suitable for every school grade and we don't know which grade we will be working with, another point is that this application again should be easily expandable and implementable for different grade students.  

%Igjen, tror ikke vi trenger begrunnelse her. 
%\bigskip\noindent
%The design and development of this software and hardware will be a very demanding phase. Designing a new application from scratch can be very time-consuming([39] - Another master %.Mention hardware too REFERANSE). However we build upon LOGO so we have some guidance to what the functionality should entail. 

%Again we will not have time to create an optimal solution as this would require testing and then redesign etc. We aim to create an easily manageable and changeable platform to build upon in future research. 

\section{Experiment}
To answer our research questions we had to test the robot in a school context. We contacted several schools and had meetings with Ungt Entreprenørskap and other researchers at IDI responsible for attracting students to study at IDI. We did not get much response, but eventually we got in contact with the international school in Trondheim. This is where we conducted our experiments. We had 2 meetings with the teacher of the class we were assigned. In these meetings we discussed our tests and curriculum and got feedback which we used to improve our experiment. It was important for us to include teachers as we wanted to integrate the robotic activities to normal schools.

\bigskip\noindent
The experiments were run at the end of the school year because of the long process of finding a school we could work with. This resulted in a smaller experiment time span than we would have liked. 


