\chapter{Research questions}
After the initial literature review (detail in part~\ref{part:literature}) we found several areas within the research which could benefit of additional research. 
%We have found a couple of research questions through an initial literature study. 
In an attempt to limit the scope of this project we selected mathematics as our main subject of interest. And in collaboration with people at NTNU we decided to take on one of the research areas which would benefit from more research, physical versus virtual robot. 
%The field of educational robotics is very broad and we needed to limit our focus. 
We set out to create a platform which could be used in an educational setting with the \chirp robot as its core component, with the end goal of providing empirical evidence that could help us answer our research questions. 
%We propose to create an intuitive platform for learning math with robotics that take the emphasis of programming.

\bigskip\noindent
The formal articulation of the research questions ended up as:
\begin{description}
	\item[RQ1: ] Can using robotics in school help students understand what an angle is and aid in their angle estimation skills?
	\item[RQ2: ] What are the strength and weaknesses of using a robot compared to a simulated environment?
\end{description}
