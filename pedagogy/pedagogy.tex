\chapter{Pedagogy}
We base this discussion on the book Mindstorm Children, computers and powerful ideas. The book is really about how a culture, a way of thinking, an idea comes to inhabit a young mind. The book introduces the pedagogy creationism. Creationism is the main pedagogic ideas behind all educational robotics. Creationism builds on constructivism.

Constructivism is not a pedagogy but a theory describing how learning happens. The theory suggests that we construct knowledge out of experiences. Therefore it implies investigation in education, experimental classroom design. a model of children as builders of their own intellectual structures

Constructionism however is a pedagogy describing how to use objects to create mental models. The main difference is that where piaget would explain the slower development of a particular concept by its greater complexity or formality, papert see the critical factor as the relative poverty of the culture in those materials that would make the concept concrete and simple. Aka a huge focus on objects as catalysts for knowledge.

It basically says that humans generate knowledge and meaning from an interaction between their experiences and their ideas in a process called assimilation. When we assimilate we incorporate the new experience into an already existing framework without changing that framework. 

Experimental design

Normally when using computers or robots in education we make the computer program the child by asking questions etc. In constructionism, teaching at students is replaced by assisting them to understand while they program the computer or robot instead. This to explore how themselves think. They have to think about what instructions to give the robot to for example run in a square. Then they have to think about how they themselves think first. In addition to this thinking process, when programming the computer the child acquires a sense of mastery over a modern powerful technology. When programming the children learn mathematics as a living language by learning to talk to a computer. Talking to the computer or robot and telling it to move etc. Papert compares this to learning french while living in france rather than in french class. We also learn to learn and love math here. 

Intellectual model building

Teaching without curriculum means supporting children as they build their own intellectual structures with materials drawn from the surrounding culture. Through creating works of art with the robot they need to think about how to make the robot move in a certain pattern and therefore reflect upon how they themselves think. object to think with - turtle. Of course the turtle can help in the teaching of traditional curriculum, but I have thought of it as a vehicle for piagetian learning, which to me is learning without curriculum.


A turtle is only one example. We can use anything and we will use the ChIRP.


Mircoworlds

We do not concern ourselves with school curriculum but we want to create a platform for teaching with a robot and as such might introduce or propose small alterations to the classroom dynamic. As papert mentions it is important for students to reach a goal by themselves, and not be programmed by the curriculum.

We will make students walk and turn according to what they think the program will do. Maybe with one arm held out in the direction they started turning and then one arm in front of them as they turn.

From constructivist theories of psychology we take a view of learning as a reconstruction rather than as a transmission of knowledge. Then we extend the idea of manipulative materials to the idea that learning is most effective when part of an activity the learner experiences as constructing is a meaningful product.  - papert

Chapter 1 paragrapf 2:
This powerful image of child as epistemologist caught my imagination while I was working with Piaget. In 1964, after five years at Piaget's Center for genetic epistemology in geneva, I came away impressed by his way of looking at children as the active builders of their own intellectual structures. But to say that intellectual structures are built by the learner rather than taught by a teacher does not mean that they are built from nothing. On the contrary: Like other builders, children appropriate to their own use materials they find about them, most saliently the models and methaphores suggested by the surrounding culture. 