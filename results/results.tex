\chapter{Results}
In order to determine each students level of understanding and knowledge prior to the experiment, seven days before the first stession, a pretest was preformed using the test created in collaboration with the class' teacher~(Appendix~\ref{appendix:pretest}). 
The same test, augmented by a short questionaire was used for the posttest~(Appendix~\ref{appendix:posttest}). The results (means) from these two tests can be seen in table~\ref{table:means} alongside with the standard deviation and p-values from running a paired-samples t-test between the pretest and posttest. For a complete writeup of this process we refer you to chapter~\ref{ch:pairedttest}.
For the comparison between groups we used a independent-samples t-test in order to determine if there was any
significant difference in the gains between the results of the robotics group and the simulator group (chapter~\ref{ch:independentttest}).
In chapter~\ref{ch:additionalTests}, "`\nameref{ch:additionalTests}"', we have included a set of additonal test,
where we further analyse the dataset to see if any significant connections can be found. 

\bigskip\noindent
All the following chapters(chapter~\ref{ch:independentttest}, \ref{ch:pairedttest} and \ref{ch:additionalTests}) have utilized different analyses, but share a common structure in how they are presented.
They start with the assumptions related to each analysis, if assumptions are violated, appropriate meaures are used to show that the violation itself does not affect the resulting outcome.
At the very end they all end with a small summary of the analysis. 
The very last chapter (chapter~\ref{ch:cronbach}) in this part of the report shows the Cronbach's alpha analysis for measuring the internal consistency reliability and test-retest analysis using Pearson's product-moment correlations between pretest and posttest scores to measure the stability reliability.

\bigskip\noindent
For simplicity and readability we've included a small table with some of the statistics from the different analyses in table~\ref{table:general}. Note some of the datapoint calculation is not explicitly shown in their respective chapter, but all assumptions were checked before including the data.

\smalltable{Statistics from the study}{table:statistics}{
	\subfloat[General statistics\label{table:general}]{
		\begin{tabular}{lll}
			& Simulator & Robotics\\\hline
			Participants & 6 & 5\\
			Male & 1 & 2\\
			Female & 5 & 3\\
			\hline
		\end{tabular}
	}
	\subfloat[Comparisons between the groups calculated by independent t-test. * outlier removed\label{table:genComp}]{
		\begin{tabular}{llll}
			& t & df & Sig\\\hline
			Pretest & -2.274 & 9 & 0.049\\
			Pretest* & -2.166 & 8 & 0.062\\
			Gains & 0.967 & 9 & 0.359\\
			Gains* & 0.232 & 7.05 & 0.823\\
			Normalized Gains & -0.117 & 9 & 0.910 \\
			Normalized Gains* & -1.903 & 8 & 0.094\\
			\hline
		\end{tabular}
	}\\
	\subfloat[Means. SD in paranthesis. * outlier removed. Sig.Gain calculated by paired t-test\label{table:means}]{
		\begin{tabular}{lllll}
			& Simulator & Robotics & Robotics* & Total\\\hline
			Pretest mean & 13.17(4.36) & 18.40(2.97) & 18.75(3.30) & 15.55(4.52)\\
			Posttest mean & 15.17(3.43) & 18.80(4.97) & 20.50(3.69)& 18.82(4.40)\\
			Gain & 2.00(2.37) & 0.40(3.13) & 1.75(0.96) & 1.27(2.72)\\
			Normalized gain & 0.17(0.17) & 0.19(0.56) & 0.42(0.26) & 0.18(0.37)\\
			Sig. Gain (p) & 0.09 & 0.79 & 0.03 & 0.152\\
			\hline
		\end{tabular}
	}\\
	
	\subfloat[Means of question categories\label{table:genQuestionMeans}]{
		\begin{tabular}{lllllll}
		\textbf{Group} & \textbf{Cat. 1}(2) & \textbf{Cat. 2}(7) & \textbf{Cat. 3}(6) & \textbf{Cat. 4}(3) & \textbf{Cat. 5}(4) & \textbf{Cat. 6}(2)\\\hline
		Robot-pre & 2.00(0.00) & 6.60(0.55) & 2.40(2.30) & 2.60(0.89) & 3.20(0.45) & 1.60(0.89) \\
		Robot-post & 2.00(0.00) & 7.00(0.00) & 3.40(3.13) & 2.40(1.34) & 3.00(0.00) & 1.00(1.00) \\
		Robot-pre* & 2.00(0.00) & 6.50(0.58) & 3.00(2.16) & 2.50(1.00) & 3.25(0.50) & 1.50(1.00)\\
		Robot-post* & 2.00(0.00) & 7.00(0.00) & 4.25(2.87) & 3.00(0.00) & 3.00(0.00) & 1.25(0.96)\\
		Sim.-pre & 1.83(0.41) & 4.67(2.07) & 1.67(2.42) & 2.33(0.82) & 1.50(1.38) & 1.17(0.75) \\
		Sim.-post & 2.00(0.00) & 6.67(0.52) & 2.33(2.73) & 1.50(1.38) & 2.00(1.26) & 0.67(0.82) \\\hline
		\multicolumn{7}{c}{In parenthesis; header: max score, body: standard deviation. * outlier removed}
		\end{tabular}
	}
}
\section{Independent-samples t-test}\label{ch:independentttest}
This test is used to determine if a difference between the means in the two groups exists, and if this difference is statistically significant.
An alternative analysis that could be used here is the ANOVA analysis, as this would similar results when the independent variable is dichotomous.
In order to use this analysis, six assumption regarding study design and the nature of the data must first be confirmed to hold.

\subsection*{Assumptions regarding study design}\label{sec:assumptionStudy}
The first three assumptions are directed at how the study was designed.
The assumptions are; 
1) must have one dependent variable measured at the continous level, 
2) must have one independent dichotomous variable, and 
3) should have independence of observations. 
In our case all of these three assumptions hold. 
The dependent variable is \texttt{diff}, representing the diffrence from pretest to posttest is a continous variable. 
The independent variable in this case is wether or not a participant were exposed to the robot or just the simulator, 
which we have encoded as \texttt{robot}. Since \texttt{robot} have just two possible values it is dichotomous.
The last assumption states that should be no overlap between the different group, which also holds true for our study. 
It should however bit noted that participants from the different group could communicate with eachother when they were not actively participating the study(e.g. during the weekend, break, etc) and that this could introduce some errors, but we do not see this as a major infraction of the assumption.

\subsection*{Assumptions regarding the data}
The last three assumption are directed at the nature of the data, and will require some extra analysis to verify. 
The assumptions are;
4) there should be no significant outliers in the two groups,
5) the dependent variable should be approximately normally distributed for each group, and
6) there is homogeneity of variances between the two groups.

\subsubsection*{Testing for outliers}\label{sec:outliers}
The forth assumption assumes that there should be no significant outliers in the different group in terms of the dependent variable. 

\image{images/results/boxplot.PNG}{\linewidth}{Boxplot of gains grouped by treatment}{fig:boxplot}

\bigskip\noindent
As seen in figure~\ref{fig:boxplot} an extreme outlier was found in our dataset within the robotics group(The asterix represents an extreme outlier in SPSS). 
By reassessing the tests, and investigating the circumstances during the pretest and posttest it was determined that the outlier were most likely not caused by errors during assessment, data entry or the tests. 
As a result of this, a comparison of the results with and without the outlier was conducted (see table~\ref{table:outliers}) to see how the outlier effected the analysis. 
It should be noted that the results without the outliers failed Levene's test for equality of variances ($p = 0.033$) and results in the table are therefore produced by using the Welch t-test. 
Both do however show that there was no statistically significant difference in the gains between the robotics and simulator group($p > 0.05$ for both results).
For good measure we have included two entries in the following table for the robotics group, one where the outlier was included and one where the outlier was not included.


\bigskip\noindent
\smalltable{Comparasion of results with and without outliers, using independent samples t-test. 95\% CI}{table:outliers}{
	\begin{tabular}{llllll}\hline
		Include outliers & t & df & Sig. & Mean difference & Std. Error difference\\\hline
		No & 0.232 & 7.05 & 0.823 & 0.25 ($-2.296..2.796$) & 1.078\\
		Yes & 0.967 & 9 & 0.359 & 1.6 ($-2.143..5.343$) & 1.655\\\hline
	\end{tabular}
}

\subsubsection*{Testing the normality}\label{sec:normality}
The fifth assumption assumes that the dependent variable should be approximately normally distributed for each group. This assumption was tested before making the comparisons seen in table~\ref{table:outliers} as this assumption must hold in order to make the comparison valid. 
In order to check the this assumption hold we calculated the skewness and kurtosis values and their respective z-values, looked at the values produces by using Shapiro-Wilks test for normality and looked at the normal Q-Q plots for each group(see tables~\ref{table:skew} and ~\ref{table:shapiro}, and figures~\ref{fig:QQRobot},\ref{fig:QQRobotPruned} and ~\ref{fig:QQSimulator}).

\bigskip\noindent
\smalltable{Z-values for skewness and kurtosis}{table:skew}{
	\begin{tabular}{lllll}
		Group & Skewness & Kurtosis & Z-skewness & Z-kurtosis\\\hline
		Robot with outlier & -1.838 & 3.751 & -2.013 & 1.876 \\
		Robot without outlier & 0.855 & -1.289 & 0.843 & -0.492\\
		Simulator & 0.000 & -1.875 & 0.000 & -1.077 \\\hline
	\end{tabular}
}
\bigskip\noindent
\smalltable{Test for normality}{table:shapiro}{
	\begin{tabular}{l|lll|lll}\hline
		Group & \multicolumn{3}{l}{Kolmogorov-Smirnov} & \multicolumn{3}{l}{Shapiro-Wilk}\\\cline{2-4}\cline{5-7}
		& Statistic & df & Sig. & Statistic & df & Sig.\\\hline
		Robot with outlier & 0.376 & 5 & 0.020 & 0.788 & 5 & 0.065\\
		Robot without outlier & 0.283 & 4 & - & 0.863 & 4 & 0.272\\
		Simulator & 0.164 & 6 & 0.200 & 0.950 & 6 & 0.739\\\hline
	\end{tabular}
}
\image{images/results/QQPlotRobot.PNG}{\linewidth}{Normal Q-Q plot of gains for the robotics group with outliers}{fig:QQRobot}
\image{images/results/QQPlotRobotWithoutOutlier.PNG}{\linewidth}{Normal Q-Q plot of gains for the robotics group without outliers}{fig:QQRobotPruned}
\image{images/results/QQPlotSimulator.PNG}{\linewidth}{Normal Q-Q plot of gains for the simulator group}{fig:QQSimulator}

\bigskip\noindent
As seen by this(tables~\ref{table:skew} and ~\ref{table:shapiro}, and figures~\ref{fig:QQRobot},~\ref{fig:QQRobotPruned}, and ~\ref{fig:QQSimulator}) we can be fairly certain that the assumption regarding normality is not violated as all z-values are within the $\pm 2.58$ range (representing a statistical significance level of $0.01$) and the Shapiro-Wilk test shows $p > 0.05$ for both groups. 

\subsubsection*{Testing for homogeneity between variances}
The last assumption relates to homogeneity of variances within each group.
As mentioned when comparing the results with and without outliers, this assumption was violated when the outlier was excluded($p = 0.033$), but held when it was included($p = 0.863$). None of the results do however show a statistically significant difference, and it therefore fair to assume that there is no statistically difference between the two groups.

\subsection*{Analysis summary}
The independent-samples t-test was used to determine if there were differences in gains from pretest to posttest between the robotics group and simulator group. One outlier was identified within the dataset (figure~\ref{fig:boxplot}), but did not effect the decision of keeping or rejection the null hypothesis and was thus kept in the dataset during the rest of the analysis. 
The gains for each group were normally distributed as shown by z-values and Shapiro-Wilk's test ($p > 0.05$), and there was homogeneity of variances when the outlier was included($p = 0.863$). The increase from pretest to posttest was greater for the simulator group ($M = 2.00, SD = 3.13$) than for the robot group ($M = 0.40, SD = 2.37$ with the outlier included, and $M=1.75, SD = 0.96$ when it was excluded), but not a statistically significant difference, $M = 1.60$, $95\% CI [-2.14, 5.34]$, $t(9) = 0.967, p = 0.359$ with the outlier included, and $M = 0.25$, $95\% CI [-2.30, 2.80]$, $t(7.05) = 0.232, p = 0.823$ when it was excluded.

\bigskip\noindent
The analysis presented here have been using the gain from pretest to posttest directly as the dependent variable. A similar analysis was conducted using the normalized gain in order to explain the gain in relation to how much each participant could potensially improve to the posttest. After checking that no assumptions were violated, with the exception of assumption four (no outliers within the groups), we were able to gather the results. 

\smalltable{Results of independent t-test. * outlier removed}{table:fullIndependeny}{
	\begin{tabular}{llllll}
		Variable & t & df & Sig. & Mean diff & Std. Error Diff\\
		Pretest & -2.274 & 9 & 0.049 & -5.23 ($-10.440..-0.026$) & 2.301\\
		Pretest* & -2.166 & 8 & 0.062 & -5.58 ($-11.528..0.361$) & 2.578\\
		Gain & 0.967 & 9 & 0.359 & 1.6 ($-2.143..5.343$) & 1.655\\
		Gain* & 0.232 & 7.05 &  0.823 & 0.25 ($-2.296..2.796$) & 1.078\\
		NormalizedGain & -0.117 & 9 & 0.910 & -0.03 ($-0.565..0.509$) & 0.238\\
		NormalizedGain* & -1.903 & 8 & 0.094 & -0.26 ($-0.565..0.054$) & 0.134\\
		\hline
	\end{tabular}
}
\chapter{Paired T-test}
\smalltable{Paired samples T-test of differences}{table:paired}{
	\begin{tabular}{lllllll}\hline	
		Group & \multicolumn{3}{l}{Paired differences} & t & df & Sig.(2-tailed)\\
		\cline{2-4}
		& Mean & Std deviation & Std. error mean & & &\\\hline
		Robot & 0.40 & 3.13 & 1.40 & 0.29 & 4 & 0.79\\
		Simulator & 2.00 & 2.37 & 0.97 & 2.07 & 5 & 0.09\\\hline
	\end{tabular}
}
\chapter{Additional analyses}\label{ch:additionalTest}
\smalltable{Indenpendent Samples T-test of differences, group by pretest-score}{table:score}{
	\begin{tabular}{lllll}\hline
		Group & N & Difference mean & Std. Deviation & Std. Error Mean\\\hline
		$Score >= 15$ & 6 & 0.33 & 2.80 & 1.15\\
		$Score < 15$ & 5 & 2.44 & 2.41 & 1.08\\\hline
	\end{tabular}
}
\chapter{Cronbach's Alpha}
In order to measure the internal consistency (reliability) of the pretest and posttest we used Cronbach's Alpha as a measurement. 
The pretest and posttest was divided into six different scales, measuring
basic understanding of angles, sizes of inner angles, sizes of reflex angles, sizes of complementary angles and angles in shapes, understanding of the robot turning, and maths involving angles. A list of tasks and the categoriy they affect is listed in table~\ref{table:testCategories}. 

\bigskip\noindent
The task indicators, \texttt{q1..q24}, refer to the tasks in the pretest and posttest where the participants could score point. The first task, \texttt{q1}, would be the second task on the pretest; \textit{draw an angle}. The fourth task on the pretest include five subtasks, these are represented as the task indicators \texttt{q3} to \texttt{q7}, and so on.
The same categories were used for the posttest as well with task indicators from \texttt{qq1} to \texttt{qq24} to avoid any confusion.

\smalltable{Tasks and categories}{table:testCategories}{
	\begin{tabular}{ll}
		Category & Tasks\\\hline
		Category 1 & q1 and q2\\
		Category 2 & q3, q4, q5, q11, q12, q13 and q14\\
		Category 3 & q6, q7, q15, q16, q17 and q18\\
		Category 4 & q8 and q9\\
		Category 5 & q19, q20, q21 and q22\\
		Category 6 & q23 and q24\\
	\end{tabular}
}

\bigskip\noindent
Table~\ref{table:cronbachPretest} and \ref{table:cronbachPosttest} shows the Cronbach's alpha for eachs of these categories for the pretest and posttest respectively. We've added a column for describing other noteworthy observations during this analysis as well. 
The cronbach's alpha for the whole pretest was $\alpha = 0.812$, for the posttest $\alpha = 0.860$, and for all tests results $\alpha = 0.908$. All of which were above the $0.7$ threshold as recommended by \cite{devellis2003scale} and \cite{kline2005principles}.


\smalltable{Cronbach's alpha for pretest}{table:cronbachPretest}{
	\begin{tabular}{lll}
		Category & Alpha & Notes\\\hline
		Category 1 & * & \wrap{\texttt{q1} was excluded because it had zero variance. Could not calculate with just one variable.}{}\\
		Category 2 & 0.739 & \wrap{\texttt{q12} showed negative correlation with the rest of the variables. Deleting it would give an alpha value of $0.815$ }{} \\
		Category 3 & 0.873 & \wrap{deleting \texttt{q6} would give an alpha value of $0.877$}{} \\
		Category 4 & 0.703 & \wrap{\texttt{q9} was excluded because had zero variance}{}\\
		Category 5 & 0.613 & \wrap{\texttt{q20} had low correlations with the other variables. Deleting it would give an alpha value of $0.750$}{}\\
		Category 6 & 0.556 & \wrap{}{}\\
		Whole test & 0.812 & \wrap{Low \textit{Corrected Item-Total Correlations} for the following variables: \texttt{q2}, \texttt{q6}, \texttt{q7}, \texttt{q10}, \texttt{q12}, \texttt{q15}, \texttt{q19}, \texttt{q20} and \texttt{q23}}{}\\\hline
	\end{tabular}
}
\smalltable{Cronbach's alpha for posttest}{table:cronbachPosttest}{
	\begin{tabular}{lll}
		Category & Alpha & Notes\\\hline
		Category 1 & * & \wrap{\texttt{q1} and \texttt{q2} was excluded because they had zero variance. Could not calculate with zero variables.}{}\\
		Category 2 & * & \wrap{\texttt{q3}, \texttt{q4}, \texttt{q5}, \texttt{q11}, \texttt{q12} and \texttt{q14} was excluded becayse they had zero variance. Could not calulate with just one variable.}{}\\
		Category 3 & 0.962 & \wrap{Low correlation for \texttt{q17}. Deleting it would give an alpha value of $0.978$}{}\\
		Category 4 & 0.909 & \wrap{}{}\\
		Category 5 & 0.384 & \wrap{Low \textit{Corrected Item-Total Correlation} for every variable except \texttt{q21}}{}\\
		Category 6 & 0.714 & \wrap{}{}\\
		Whole test & 0.860 & \wrap{Low \textit{Corrected Item-Total Correlation} for \texttt{q19}, \texttt{q20} and \texttt{22}}{}\\
	\end{tabular}
}


\chapter{Observation}
The observation during the experiment was performed in order to understand how the students though, worked and cooperated during the experiment. Here we report interesting findings.

\bigskip\noindent
The students appeared to have more fun with the robot activity than the simulator activity. The teacher mentioned this as well. She said that when they were done they were all excited and told her how fun it had been. Although it is nice that the students are having fun this might pose a problem if it shifts their focus away from the actual learning. One of the groups even started playing around and doing other things than the exercises. The same group started filming the robots in the second lecture. But after a while the teacher intervened and told them to put the phone away and focus on the tasks. This problem was not observed in the simulator group. In lecture when the teacher told them that the simulator group goes first, several students shouted YES! Some students also asked if they could stay and work with the robots during the break after the first robot session. 

\bigskip\noindent
In the robot group on the first day everyone on all teams discussed with each other. Nobody seemed uninterested. In one group there was a particularly well taught student. This team did not cooperate very well. The good student tried to include the others but he mainly told them what to do even though they controlled the tablet. In the second lecture one of the group members seemed quite uninterested. Most of the students asked for help among their fellow students. Some even asked questions to other groups. They did not specifically ask how they solved the task, but they asked if they had gotten similar errors to what they were facing at the moment.

\bigskip\noindent
The different groups applied different approaches. The choice of approach did not differ much or seem do be dependent on whether they were in the experimental group or the control group. A trial and error approach was present to some degree in most groups. Some groups would create a plan at first by measuring all the angles and programming everything step by step, but when the robot did not create the path they wanted, they tried again or changed the program slightly and tried again, without reflecting on why it did not work. Many groups created the shape in exercise 3 when attempting to create the shape in exercise 2. We hoped this would cause some reflection, but many of the students simply checked exercise 3 as done, and then could not figure out how to create the shape from exercise 2. The students who talked it over and measured the angles before programming managed to do more tasks than the teams that did not. After a while students started to used their hands to keep track of which direction the robot would face at a certain point in the program. This is also reflected in the depth questions from the test, where several students wrote that they used their hands to aid estimating the angles.

\bigskip\noindent
The students were very engaged and worked the whole duration of the activity. Sometimes the students would get frustrated when they could not find the solution, but then they would return to the task when they had thought a little bit about it, or let someone else on the group use the tablet for a while. During the experiment we overheard many students come to realizations about which way and how much to turn. They would say ``Oh, now I see'' or ``It has to turn 135! not 45!''. 

\bigskip\noindent
Even though some students tried to explain how they got the answer they were lacking the necessary vocabulary to express themselves. They did not describe the robot's movements or their actions with correct mathematical expressions. This is illustrated by quotes like: ``We forgot to take it up!'', overheard when a student didn't move forward after a turn. 

\bigskip\noindent
The simulator group's alternative solutions to angle creation in exercise 4 from lecture 1 was interesting. Where the robot group could not come up with a new solution or eventually understood that they needed to use the outer angle and turn right instead of left or left instead of right. The simulator students were happy with the program TR FWD TL FWD as an alternative to FWD TL FWD, because the simulated robot moved in a different direction. When using a physical robot you can place the robot where you want and therefore they did not think of this as an alternative solution.

\bigskip\noindent
The students were allowed to use protractors, but they did not use it the way we intended. The correct way to use the protractor would be to measure the angle between the robot's movement path and the next line in the path. Instead they measured the internal angle of the shape and then had to do calculations to figure out how far to turn the robot.

\bigskip\noindent
Regarding the difficulty of the exercises, the closed shape exercises in lecture 2 turned out to be a lot harder for students to manage compared to the open polygons or paths as we call them from lecture one. We thought the first 3 shapes would be relatively easy since they were regular polygons in which all sides and angles are equal. The program you have to make is merely the same step 3,4 or 5 times. Even though it was hard, the students seemed to grasp the ideas pretty fast. We could have introduced the idea of a repeat block, but that would have taken away from the main activity, and again we did not have that much time.
\section{Depth questions}
In addition to the scored items on the test there were a few questions asking how the students thought or how they found the answer. The findings from these questions are presented here. The answers did not vary between experimental and control group students, so all results are presented at the same time here.

\bigskip\noindent
In question 1 nobody managed to accurately describe what an angle is on the pretest. The definition of an angle is ``A shape, formed by two lines or rays diverging from a common point''. Most students answered this question by drawing an arrow pointing to a drawing in question 2 and wrote ``this is an example of an angle''. However one student was close to the definition. This student wrote: ``an angle is two or more lines that meet into a point''. The error being two or more lines when it should be just two lines. In the posttest there was a major improvement in all almost all the replies. 4 students got the question classified as wrong. Everyone else wrote that it is 2 lines that meet or comes together. 3 of the students also mentioned that the two lines must share a common point.

\bigskip\noindent
In question 2 everyone drew an angle that were classified as correct in the pretest. In question 3 everyone except one student drew a larger angle in the pretest. However their explanations of why it was bigger were not accurate. Only 3 of the students were classified as correct. They wrote: ``this is bigger because its more open'', ``it is bigger because it's more open'' and ``it has a bigger opening''. In the posttest everyone got 2 points for drawing an angle and drawing a bigger angle. The explanations had improved significantly. 4 students did not provide accurate descriptions of why it was bigger, but the remaining students were classified as correct. Their explanations are grouped by the mentioned properties. 1 student said the lines are longer away, 1 said longer out, 2 said that it had a bigger opening, 1 student said that the angle was wider and the last 2 students explained it by saying the angle in task 2 was $\frac{1}{4}$ of a circle while the angle in task 3 were $\frac{1}{2}$ of a circle.

\bigskip\noindent
In tasks 5-10 there were a lot of different answers. We primarily wanted to change the way they think, but most of the explanations remain the same from pretest to posttest. We will focus on the explanations that changed from the pretest to posttest. The biggest changes were found in question 5 and 6. The findings are presented below by quoting what they answered in the pretest and posttest and discussing these results.

\bigskip\noindent
In question 5 most students used the same approach to the problem in both the pretest and the posttest. From the students that did change their approach we can see that 2 students started using their hands to aid in the estimation. One student went from a calculation approach to an estimation approach and one student went from an estimation approach to a calculation approach. It is hard to conclude with a positive or negative overall effect when there is no trend across students to what effect the experiment had on their approach. The changes are discussed in detail below.

\bigskip\noindent
\textbf{Pretest:} ``by estimating and looking at the other angles for example''
\par\noindent \textbf{Posttest:} ``45+45 = 90  +90 = 180''
\par\noindent \textbf{Discussion:} This student has gone from an estimation approach to a calculation approach to the problem. The student has realized that there is 180 degrees in a triangle even though we never explicitly mentioned this.

\bigskip\noindent
\textbf{Pretest:} ``example in the question number 2. I just knew that if I plussed 60+60 degrees it would be 120''
\par\noindent \textbf{Posttest:} ``I used my fingers to measure the angles''
\par\noindent \textbf{Discussion:} This student had correct answers to questions 5.2 and 5.3 in the pretest, but they were wrong in the posttest. We can see that the though process was wrong in the pretest. The student got 120 degrees on question 5.2 by adding 60 to 60. Even though the student now got the tasks wrong, the student now has a more correct approach, even though the approach we wanted to teach them was to subtract 60 from 180.

\bigskip\noindent
\textbf{Pretest:} ``45+45=90''
\par\noindent \textbf{Posttest:} ``The two other angles are smaller so I just guessed''
\par\noindent \textbf{Discussion:} This student has gone from a calculation approach to an estimation approach. This resulted in a wrong answer on task 5.1 in the posttest. The student had a correct answer to 5.1 in the pretest.

\bigskip\noindent
\textbf{Pretest:} ``I looked at the small circles in the corners to see which one is biggest''
\par\noindent \textbf{Posttest:} ``I measured with my fingers to see how big''
\par\noindent \textbf{Discussion:} This student used estimation in both pre and post test; however in the posttest the student started using his/her hands. 

\bigskip\noindent
In question 6 most students, like in question 5, used the same approach. From the students that did change their approach we can see that they predominantly changed into a calculation approach. One student improved his/her estimation by using his/her hands. Overall the experiment have improved the students approach to estimating the angles in question 6. The changes are discussed in detail below. 

\bigskip\noindent
\textbf{Pretest:} ``I looked at which angle fits which number''
\par\noindent \textbf{Posttest:} ``I measured with my fingers''
\par\noindent \textbf{Discussion:} This student answered correctly on question 6.4 in the post test. The student answered wrong on this question in the pretest. The rest of the results are the same. Using his/her fingers have improved his/her ability to estimate angles. 

\bigskip\noindent
\textbf{Pretest:} ``I plussed 180 and 180 and got the answer''
\par\noindent \textbf{Posttest:} ``I knew that a straight line is 180 so I took about 35 away and found the answer''
\par\noindent \textbf{Discussion:} This student went from a wrong answer on 5.4 in the pretest to full score in the posttest. The approach taken in the posttest is correct, although he/she should subtract 45 degrees, not 35, but it is an estimation task after all. The approach taken in the pretest is wrong and his/her approach has improved during the experiment.

\bigskip\noindent
\textbf{Pretest:} ``With my eyes''
\par\noindent \textbf{Posttest:} ``I plussed 45+45+45 = 135''
\par\noindent \textbf{Discussion:} This student got a full score on both the pretest and the posttest; however the approach taken in the posttest is a calculation approach and considered better than the estimation approach used in the pretest.

\bigskip\noindent
\textbf{Pretest:} ``I used the other numbers on number 6 to help me. And measuring with stripes with my fingers''
\par\noindent \textbf{Posttest:} ``I took and divided it into 90 degrees and saw it was 30 degrees + it together''
\par\noindent \textbf{Discussion:} This student had marked a 90 degree angle on the 135 degree angle and estimated the degree of the remaining angle to be 30 degree, although it was 45. However he/she used a calculation approach on the posttest instead of an estimation approach used in the pretest.

\bigskip\noindent
In questions 7-10 there was no change in the students approach. Students still struggle with these tasks, as they did in the pretest. They measure the inner angle and not the outer angle which is marked. We covered what the outside angle is and how it is marked in our introduction, but it is a technical issue that is hard to learn by experimentation. If they did not understand it when we described it, it would be hard to discover this on their own. The only noteworthy change was in question 7, where a student explained his approach with ``360-90''. In the pretest this student simply pointed an arrow to question 7.2 and wrote that it these angles are the same. In question 8 there was general many wrong statements and explanations. The students did not manage to accuratly describe how they thought. We would suggest having conversations or interviews with the students to understand how they think. Only one student had a full score on the post test, but this student had a full score on the pretest as well. The other students struggled with both the amount of rotation and the direction they should rotate. In question 9 most students used a mathematical approach, either writing that 12X30 = 360 or 360/30 = 12. 

\bigskip\noindent
Question 11 was not scored but included to assess the students' attitude towards educational robotics. We saw an increase in the student attitude on this question. Two students answered blank on both the pretest and the posttest. 5 students answered ``yes'' on both pretest and posttest. One student answered ``sometimes'' on both pretest and posttest. The last three students changed their attitude positively. They went from ``no'', ``maybe'' and ``I don't know'' to ``yes'', ``yes, we have'' and ``I think so''.

\chapter{Questionaire}
Overall the questionaire did not yield very significant results. There was about the same spread in both robot and simulation group. Surprisingly there was a difference in question 17: Do you think imagining a robot turning can help you when working with angles? For the robot group 2 answered 0 and 2 answered + giving a total score of 2. In the simulator group 3 answered ++ and 1 answered + giving a total score of 7. This is the main factor we thought would be prevalent in the robot group. As seen in figure~\ref{fig:questionare} there was a slightly higher score for the robot group in the questions regarding interest, motivation and entertainment value when summed together. Question 15 and 16 scored the same for both groups.

\image{images/results/questionaire.PNG}{\linewidth}{Histogram of answers in questions 12,13 and 14}{fig:questionare}


