\section{Depth questions}
In addition to the scored items on the test there were a few questions asking how the students thought or how they found the answer. The findings from these questions are presented here. The answers did not vary between experimental and control group students, so all results are presented at the same time here.

\bigskip\noindent
In question 1 nobody managed to accurately describe what an angle is on the pretest. The definition of an angle is ``A shape, formed by two lines or rays diverging from a common point''. Most students answered this question by drawing an arrow pointing to a drawing in question 2 and wrote ``this is an example of an angle''. However one student was close to the definition. This student wrote: ``an angle is two or more lines that meet into a point''. The error being two or more lines when it should be just two lines. In the posttest there was a major improvement in all almost all the replies. 4 students got the question classified as wrong. Everyone else wrote that it is 2 lines that meet or comes together. 3 of the students also mentioned that the two lines must share a common point.

\bigskip\noindent
In question 2 everyone drew an angle that were classified as correct in the pretest. In question 3 everyone except one student drew a larger angle in the pretest. However their explanations of why it was bigger were not accurate. Only 3 of the students were classified as correct. They wrote: ``this is bigger because its more open'', ``it is bigger because it's more open'' and ``it has a bigger opening''. In the posttest everyone got 2 points for drawing an angle and drawing a bigger angle. The explanations had improved significantly. 4 students did not provide accurate descriptions of why it was bigger, but the remaining students were classified as correct. Their explanations are grouped by the mentioned properties. 1 student said the lines are longer away, 1 said longer out, 2 said that it had a bigger opening, 1 student said that the angle was wider and the last 2 students explained it by saying the angle in task 2 was $\frac{1}{4}$ of a circle while the angle in task 3 were $\frac{1}{2}$ of a circle.

\bigskip\noindent
In tasks 5-10 there were a lot of different answers. We primarily wanted to change the way they think, but most of the explanations remain the same from pretest to posttest. We will focus on the explanations that changed from the pretest to posttest. The biggest changes were found in question 5 and 6. The findings are presented below by quoting what they answered in the pretest and posttest and discussing these results.

\bigskip\noindent
In question 5 most students used the same approach to the problem in both the pretest and the posttest. From the students that did change their approach we can see that 2 students started using their hands to aid in the estimation. One student went from a calculation approach to an estimation approach and one student went from an estimation approach to a calculation approach. It is hard to conclude with a positive or negative overall effect when there is no trend across students to what effect the experiment had on their approach. The changes are discussed in detail below.

\bigskip\noindent
\textbf{Pretest:} ``by estimating and looking at the other angles for example''
\par\noindent \textbf{Posttest:} ``45+45 = 90  +90 = 180''
\par\noindent \textbf{Discussion:} This student has gone from an estimation approach to a calculation approach to the problem. The student has realized that there is 180 degrees in a triangle even though we never explicitly mentioned this.

\bigskip\noindent
\textbf{Pretest:} ``example in the question number 2. I just knew that if I plussed 60+60 degrees it would be 120''
\par\noindent \textbf{Posttest:} ``I used my fingers to measure the angles''
\par\noindent \textbf{Discussion:} This student had correct answers to questions 5.2 and 5.3 in the pretest, but they were wrong in the posttest. We can see that the though process was wrong in the pretest. The student got 120 degrees on question 5.2 by adding 60 to 60. Even though the student now got the tasks wrong, the student now has a more correct approach, even though the approach we wanted to teach them was to subtract 60 from 180.

\bigskip\noindent
\textbf{Pretest:} ``45+45=90''
\par\noindent \textbf{Posttest:} ``The two other angles are smaller so I just guessed''
\par\noindent \textbf{Discussion:} This student has gone from a calculation approach to an estimation approach. This resulted in a wrong answer on task 5.1 in the posttest. The student had a correct answer to 5.1 in the pretest.

\bigskip\noindent
\textbf{Pretest:} ``I looked at the small circles in the corners to see which one is biggest''
\par\noindent \textbf{Posttest:} ``I measured with my fingers to see how big''
\par\noindent \textbf{Discussion:} This student used estimation in both pre and post test; however in the posttest the student started using his/her hands. 

\bigskip\noindent
In question 6 most students, like in question 5, used the same approach. From the students that did change their approach we can see that they predominantly changed into a calculation approach. One student improved his/her estimation by using his/her hands. Overall the experiment have improved the students approach to estimating the angles in question 6. The changes are discussed in detail below. 

\bigskip\noindent
\textbf{Pretest:} ``I looked at which angle fits which number''
\par\noindent \textbf{Posttest:} ``I measured with my fingers''
\par\noindent \textbf{Discussion:} This student answered correctly on question 6.4 in the post test. The student answered wrong on this question in the pretest. The rest of the results are the same. Using his/her fingers have improved his/her ability to estimate angles. 

\bigskip\noindent
\textbf{Pretest:} ``I plussed 180 and 180 and got the answer''
\par\noindent \textbf{Posttest:} ``I knew that a straight line is 180 so I took about 35 away and found the answer''
\par\noindent \textbf{Discussion:} This student went from a wrong answer on 5.4 in the pretest to full score in the posttest. The approach taken in the posttest is correct, although he/she should subtract 45 degrees, not 35, but it is an estimation task after all. The approach taken in the pretest is wrong and his/her approach has improved during the experiment.

\bigskip\noindent
\textbf{Pretest:} ``With my eyes''
\par\noindent \textbf{Posttest:} ``I plussed 45+45+45 = 135''
\par\noindent \textbf{Discussion:} This student got a full score on both the pretest and the posttest; however the approach taken in the posttest is a calculation approach and considered better than the estimation approach used in the pretest.

\bigskip\noindent
\textbf{Pretest:} ``I used the other numbers on number 6 to help me. And measuring with stripes with my fingers''
\par\noindent \textbf{Posttest:} ``I took and divided it into 90 degrees and saw it was 30 degrees + it together''
\par\noindent \textbf{Discussion:} This student had marked a 90 degree angle on the 135 degree angle and estimated the degree of the remaining angle to be 30 degree, although it was 45. However he/she used a calculation approach on the posttest instead of an estimation approach used in the pretest.

\bigskip\noindent
In questions 7-10 there was no change in the students approach. Students still struggle with these tasks, as they did in the pretest. They measure the inner angle and not the outer angle which is marked. We covered what the outside angle is and how it is marked in our introduction, but it is a technical issue that is hard to learn by experimentation. If they did not understand it when we described it, it would be hard to discover this on their own. The only noteworthy change was in question 7, where a student explained his approach with ``360-90''. In the pretest this student simply pointed an arrow to question 7.2 and wrote that it these angles are the same. In question 8 there was general many wrong statements and explanations. The students did not manage to accuratly describe how they thought. We would suggest having conversations or interviews with the students to understand how they think. Only one student had a full score on the post test, but this student had a full score on the pretest as well. The other students struggled with both the amount of rotation and the direction they should rotate. In question 9 most students used a mathematical approach, either writing that 12X30 = 360 or 360/30 = 12. 

\bigskip\noindent
Question 11 was not scored but included to assess the students' attitude towards educational robotics. We saw an increase in the student attitude on this question. Two students answered blank on both the pretest and the posttest. 5 students answered ``yes'' on both pretest and posttest. One student answered ``sometimes'' on both pretest and posttest. The last three students changed their attitude positively. They went from ``no'', ``maybe'' and ``I don't know'' to ``yes'', ``yes, we have'' and ``I think so''.
