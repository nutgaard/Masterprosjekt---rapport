\section{Reliability estimates}\label{ch:cronbach}
\subsection*{Cronbach's alpha}
In order to measure the internal consistency reliability of the pretest and posttest we used Cronbach's Alpha as a measurement. 
The pretest and posttest was divided into six different scales, measuring
basic understanding of angles, sizes of inner angles, sizes of reflex angles, sizes of complementary angles and angles in shapes, understanding of the robot turning, and maths involving angles. A list of tasks and the categoriy they affect is listed in table~\ref{table:testCategories}. 

\bigskip\noindent
The task indicators, \texttt{q1..q24}, refer to the tasks in the pretest and posttest where the participants could score point. The first task, \texttt{q1}, would be the second task on the pretest; \textit{draw an angle}. The fourth task on the pretest include five subtasks, these are represented as the task indicators \texttt{q3} to \texttt{q7}, and so on.
The same categories were used for the posttest as well with task indicators from \texttt{qq1} to \texttt{qq24} to avoid any confusion.

\smalltable{Tasks and categories}{table:testCategories}{
	\begin{tabular}{lll}
		Category & Description & Tasks\\\hline
		Category 1 & General angle understanding & \texttt{q1, q2}\\
		Category 2 & Normal (inner) angles & \texttt{q3, q4, q5, q11, q12, q13, q14}\\
		Category 3 & Reflex (outer) angles & \texttt{q6, q7, q15, q16, q17, q18}\\
		Category 4 & Complementary angles & \texttt{q8, q9, q10}\\
		Category 5 & Orientation and estimation & \texttt{q19, q20, q21, q22}\\
		Category 6 & Calculations with angles & \texttt{q23, q24}\\
	\end{tabular}
}

\bigskip\noindent
Table~\ref{table:cronbachPretest} and \ref{table:cronbachPosttest} shows the Cronbach's alpha for eachs of these categories for the pretest and posttest respectively. We've added a column for describing other noteworthy observations during this analysis as well. 
The cronbach's alpha for the whole pretest was $\alpha = 0.812$, for the posttest $\alpha = 0.860$, and for all tests results $\alpha = 0.908$.
All of which were above the $0.7$ threshold as recommended by \cite{devellis2003scale} and \cite{kline2005principles}, indicating a high degree of internal consistency. The alpha value when comparing the pretest scores to the posttest scores was $a = 0.897$, indicating a high degree of stability.	The comparison of pretest scores to posttest scores are further analysied in section~\ref{sec:testretest} where Pearson's product-moment correlations are calculated.


\smalltable{Cronbach's alpha for pretest}{table:cronbachPretest}{
	\begin{tabular}{lll}
		Category & Alpha & Notes\\\hline
		Category 1 & * & \wrap{\texttt{q1} was excluded because it had zero variance. Could not calculate with just one variable.}{}\\
		Category 2 & 0.739 & \wrap{\texttt{q12} showed negative correlation with the rest of the variables. Deleting it would give an alpha value of $0.815$ }{} \\
		Category 3 & 0.873 & \wrap{deleting \texttt{q6} would give an alpha value of $0.877$}{} \\
		Category 4 & 0.703 & \wrap{\texttt{q9} was excluded because had zero variance}{}\\
		Category 5 & 0.613 & \wrap{\texttt{q20} had low correlations with the other variables. Deleting it would give an alpha value of $0.750$}{}\\
		Category 6 & 0.556 & \wrap{}{}\\
		Whole test & 0.812 & \wrap{Low \textit{Corrected Item-Total Correlations} for the following variables: \texttt{q2}, \texttt{q6}, \texttt{q7}, \texttt{q10}, \texttt{q12}, \texttt{q15}, \texttt{q19}, \texttt{q20} and \texttt{q23}}{}\\\hline
	\end{tabular}
}
\smalltable{Cronbach's alpha for posttest}{table:cronbachPosttest}{
	\begin{tabular}{lll}
		Category & Alpha & Notes\\\hline
		Category 1 & * & \wrap{\texttt{q1} and \texttt{q2} was excluded because they had zero variance. Could not calculate with zero variables.}{}\\
		Category 2 & * & \wrap{\texttt{q3}, \texttt{q4}, \texttt{q5}, \texttt{q11}, \texttt{q12} and \texttt{q14} was excluded becayse they had zero variance. Could not calulate with just one variable.}{}\\
		Category 3 & 0.962 & \wrap{Low correlation for \texttt{q17}. Deleting it would give an alpha value of $0.978$}{}\\
		Category 4 & 0.909 & \wrap{}{}\\
		Category 5 & 0.384 & \wrap{Low \textit{Corrected Item-Total Correlation} for every variable except \texttt{q21}}{}\\
		Category 6 & 0.714 & \wrap{}{}\\
		Whole test & 0.860 & \wrap{Low \textit{Corrected Item-Total Correlation} for \texttt{q19}, \texttt{q20} and \texttt{22}}{}\\
	\end{tabular}
}

\subsection{Test-retest correlations}\label{sec:testretest}
In order to measure the stability reliability we used the Pearson's product-moment correlations between the pretest scores and posttest scores. The initial analysis showed indicated that the relationship to be linear with both variables normally distributed(Shapiro-Wilk, $p > 0.05$). 
There was a strong positive correlation between the pretest scores and posttest scores ($r(9) = 0.814, p = 0.002$).
