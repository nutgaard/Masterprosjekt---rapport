\chapter{Observation}
The observation during the experiment was performed in order to understand how the students though, worked and cooperated during the experiment. Here we report interesting findings.

\bigskip\noindent
The students appeared to have more fun with the robot activity than the simulator activity. The teacher mentioned this as well. She said that when they were done they were all excited and told her how fun it had been. Although it is nice that the students are having fun this might pose a problem if it shifts their focus away from the actual learning. One of the groups even started playing around and doing other things than the exercises. The same group started filming the robots in the second lecture. But after a while the teacher intervened and told them to put the phone away and focus on the tasks. This problem was not observed in the simulator group. In lecture when the teacher told them that the simulator group goes first, several students shouted YES! Some students also asked if they could stay and work with the robots during the break after the first robot session. 

\bigskip\noindent
In the robot group on the first day everyone on all teams discussed with each other. Nobody seemed uninterested. In one group there was a particularly well taught student. This team did not cooperate very well. The good student tried to include the others but he mainly told them what to do even though they controlled the tablet. In the second lecture one of the group members seemed quite uninterested. Most of the students asked for help among their fellow students. Some even asked questions to other groups. They did not specifically ask how they solved the task, but they asked if they had gotten similar errors to what they were facing at the moment.

\bigskip\noindent
The different groups applied different approaches. The choice of approach did not differ much or seem do be dependent on whether they were in the experimental group or the control group. A trial and error approach was present to some degree in most groups. Some groups would create a plan at first by measuring all the angles and programming everything step by step, but when the robot did not create the path they wanted, they tried again or changed the program slightly and tried again, without reflecting on why it did not work. Many groups created the shape in exercise 3 when attempting to create the shape in exercise 2. We hoped this would cause some reflection, but many of the students simply checked exercise 3 as done, and then could not figure out how to create the shape from exercise 2. The students who talked it over and measured the angles before programming managed to do more tasks than the teams that did not. After a while students started to used their hands to keep track of which direction the robot would face at a certain point in the program. This is also reflected in the depth questions from the test, where several students wrote that they used their hands to aid estimating the angles.

\bigskip\noindent
The students were very engaged and worked the whole duration of the activity. Sometimes the students would get frustrated when they could not find the solution, but then they would return to the task when they had thought a little bit about it, or let someone else on the group use the tablet for a while. During the experiment we overheard many students come to realizations about which way and how much to turn. They would say ``Oh, now I see'' or ``It has to turn 135! not 45!''. 

\bigskip\noindent
Even though some students tried to explain how they got the answer they were lacking the necessary vocabulary to express themselves. They did not describe the robot's movements or their actions with correct mathematical expressions. This is illustrated by quotes like: ``We forgot to take it up!'', overheard when a student didn't move forward after a turn. 

\bigskip\noindent
The simulator group's alternative solutions to angle creation in exercise 4 from lecture 1 was interesting. Where the robot group could not come up with a new solution or eventually understood that they needed to use the outer angle and turn right instead of left or left instead of right. The simulator students were happy with the program TR FWD TL FWD as an alternative to FWD TL FWD, because the simulated robot moved in a different direction. When using a physical robot you can place the robot where you want and therefore they did not think of this as an alternative solution.

\bigskip\noindent
The students were allowed to use protractors, but they did not use it the way we intended. The correct way to use the protractor would be to measure the angle between the robot's movement path and the next line in the path. Instead they measured the internal angle of the shape and then had to do calculations to figure out how far to turn the robot.

\bigskip\noindent
Regarding the difficulty of the exercises, the closed shape exercises in lecture 2 turned out to be a lot harder for students to manage compared to the open polygons or paths as we call them from lecture one. We thought the first 3 shapes would be relatively easy since they were regular polygons in which all sides and angles are equal. The program you have to make is merely the same step 3,4 or 5 times. Even though it was hard, the students seemed to grasp the ideas pretty fast. We could have introduced the idea of a repeat block, but that would have taken away from the main activity, and again we did not have that much time.