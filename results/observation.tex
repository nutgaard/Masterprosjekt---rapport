\section{Observation}
The observation during the experiment was performed in order to understand the students' thought process, and how they worked and cooperated during the experiment. Here we report interesting findings.

\bigskip\noindent
The students appeared to have more fun with the robot activity than the simulator activity. 
Some students even asked if they could stay and work with the robots during the break after the first robot session. 
The teacher mentioned this as well. She said that when they were done they were all excited and told her how fun it had been. Although it is nice that the students are having fun this might pose a problem if it shifts their focus away from the actual learning. One of the groups started playing and doing other activities than the exercises. The same group started filming the robots in the second lecture. After a while the teacher intervened and told them to put the phone away and focus on the tasks. This problem was not observed in the simulator group. 
The teacher had told the students about the experiment and that it would involve robots before we did the experiment. 
Some students in the control group could also see the other activities in the first lecture, since we were sharing the classroom with them, separated only by a whiteboard.
This could have caused a negative effect in the control group students, because they were expecting robots, but got to work only with the simulator. 
However in the second lecture when the teacher told them that the simulator group goes first, several students shouted YES! 

\bigskip\noindent
In the robot group on the first day everyone discussed the problems with each other. No students seemed uninterested or sat by idle. In one group there was one student who understood the concepts very well. This team did not cooperate very well. The good student tried to include the others but even though they controlled the tablet, they did not get to experiment much as he knew the answer to most of the problems. In the second lecture one of the group members seemed quite uninterested. This student was on the group with the student who knew most of the concepts. Most of the students asked for help among their fellow students. Some even asked questions to other groups. They did not specifically ask how they solved the task, but they asked if they had gotten similar errors to what they were facing at that time.

\bigskip\noindent
The different groups used different approaches. The choice of approach did not differ much or seem do be dependent on whether they were in the experimental group or the control group. A trial and error approach was present to some degree in most groups. Some groups would create a plan at first by measuring all the angles and programming everything step by step, but when the robot did not create the path they wanted, they tried again or changed the program slightly and tried again, without reflecting on why it did not work. Many groups created the shape in exercise 3 when attempting to create the shape in exercise 2 in the first lecture. We hoped this would cause some reflection, but many of the students simply checked exercise 3 as done, and then could not figure out how to create the shape from exercise 2. The students who talked it over and measured the angles before programming managed to do more tasks than the teams that did not. After a while students started to used their hands to keep track of which direction the robot would face at a certain point in the program. This is also reflected in the depth questions from the test, where several students wrote that they used their hands to aid estimating the angles.

\bigskip\noindent
The students were very engaged and worked the whole duration of the activity. Sometimes the students would get frustrated when they could not find the solution, but then they would return to the task when they had thought a little bit about it, or let someone else on the group use the tablet for a while. During the experiment we overheard many students come to realizations about which way and how much to turn. They would say ``Oh, now I see'' or ``It has to turn 135! not 45!''. 

\bigskip\noindent
Even though some students tried to explain how they got the answer they were lacking the necessary vocabulary to express themselves. They did not describe the robot's movements or their actions with correct mathematical expressions. This is illustrated by quotes like: ``We forgot to take it up!'', overheard when a student didn't move forward after a turn. 

\bigskip\noindent
One benefit of the robots physical element became clear when the students had to create alternative programs for the robot, to create the same shape. A physical robot can be placed at an initial starting position and with an initial starting direction at the users choice. With the simulator this was not an option and students in the simulator group therefore thought of 2 identical programs, with the exception of turning right 90 degrees at the start, as different solutions. For example they created \texttt{TR FWD TL FWD} as an alternative to \texttt{FWD TL FWD}, because the simulated robot moved in a different direction.

\bigskip\noindent
The students were allowed to use protractors, but they did not use it the way we intended. The correct way to use the protractor would be to measure the angle between the robot's movement path and the next line in the path. Instead they measured the internal angle of the shape and then had to do calculations to figure out how far to turn the robot.

\bigskip\noindent
Regarding the difficulty of the exercises, the closed shape exercises in lecture 2 turned out to be a lot harder for students to manage compared to the open polygons or paths as we call them from lecture one. We thought the first 3 shapes would be relatively easy since they were regular polygons in which all sides and angles are equal. The program you have to make is merely the same step 3,4 or 5 times. Even though it was hard, the students seemed to grasp the ideas pretty fast. We could have introduced the idea of a repeat block, but that would have taken away from the main activity, and again we did not have that much time.

\paragraph{conclusion}
The robots were more fun for the students than the simulator.
This might have positive and negative effects, as we observed students working with the robot loosing focus on the tasks.  
Both the robot and simulator foster collaboration, but if there is a significant skill difference between the students it might be a problem. This student will know the answers before the others, and they wont be able to experiment, fail and reason about these failures. 
One solution to this problem is tasks requiring multiple group roles. In this way everyone has to do one job and the task can't be done by just one student.  
A trial and error approach was present in most groups. We wanted the students to reason about their failures and why the robot behaved the way it did. One solution to this problem can be to make them explain their solution as a part of the task. This would however require many teachers per classroom, which is often not realistic. One alternative solution is having students prepare a presentation of the solution and have one random group present their solution and explain it at the end of the session. We unfortunately did not have time to do this.  
The robot's physical element was shown to be more intuitive in understanding what makes programs different from each other.
The students did not use their protractors correctly. They measured the inner angle instead of the supplementary angle which exist between the robots direction vector and the next path in the program. A protractor could have been used in the introduction, but we wanted them to figure it out by themselves. However this appeared to be harder for them than anticipated.