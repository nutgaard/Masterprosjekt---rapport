\section{Independent-samples t-test}\label{ch:independentttest}
This test is used to determine if a difference between the means in the two groups exists, and if this difference is statistically significant.
An alternative analysis that could be used here is the ANOVA analysis, as this would similar results when the independent variable is dichotomous.
In order to use this analysis, six assumption regarding study design and the nature of the data must first be confirmed to hold.

\subsection*{Assumptions regarding study design}\label{sec:assumptionStudy}
The first three assumptions are directed at how the study was designed.
The assumptions are; 
1) must have one dependent variable measured at the continous level, 
2) must have one independent dichotomous variable, and 
3) should have independence of observations. 
In our case all of these three assumptions hold. 
The dependent variable is \texttt{diff}, representing the diffrence from pretest to posttest is a continous variable. 
The independent variable in this case is wether or not a participant were exposed to the robot or just the simulator, 
which we have encoded as \texttt{robot}. Since \texttt{robot} have just two possible values it is dichotomous.
The last assumption states that should be no overlap between the different group, which also holds true for our study. 
It should however bit noted that participants from the different group could communicate with eachother when they were not actively participating the study(e.g. during the weekend, break, etc) and that this could introduce some errors, but we do not see this as a major infraction of the assumption.

\subsection*{Assumptions regarding the data}
The last three assumption are directed at the nature of the data, and will require some extra analysis to verify. 
The assumptions are;
4) there should be no significant outliers in the two groups,
5) the dependent variable should be approximately normally distributed for each group, and
6) there is homogeneity of variances between the two groups.

\subsubsection*{Testing for outliers}\label{sec:outliers}
The forth assumption assumes that there should be no significant outliers in the different group in terms of the dependent variable. 

\image{images/results/boxplot.PNG}{\linewidth}{Boxplot of gains grouped by treatment}{fig:boxplot}

\bigskip\noindent
As seen in figure~\ref{fig:boxplot} an extreme outlier was found in our dataset within the robotics group(The asterix represents an extreme outlier in SPSS). 
By reassessing the tests, and investigating the circumstances during the pretest and posttest it was determined that the outlier were must likely not caused by errors during assessment, data entry or the tests. 
As a result of this, a comparison of the results with and without the outlier was conducted (see table~\ref{table:outliers}) to see how the outlier effected the analysis. 
It should be noted that the results without the outliers failed Levene's test for equality of variances ($p = 0.033$) and results in the table are therefore produced by using the Welch t-test. 
Both do however show that there was no statistically significant difference in the gains between the robotics and simulator group($p > 0.05$ for both results).
For good measure we have included two entries in the following table for the robotics group, one where the outlier was included and one where the outlier was not included.


\bigskip\noindent
\smalltable{Comparasion of results with and without outliers, using independent samples t-test. 95\% CI}{table:outliers}{
	\begin{tabular}{llllll}\hline
		Include outliers & t & df & Sig. & Mean difference & Std. Error difference\\\hline
		No & 0.232 & 7.05 & 0.823 & 0.25 ($-2.296..2.796$) & 1.078\\
		Yes & 0.967 & 9 & 0.359 & 1.6 ($-2.143..5.343$) & 1.655\\\hline
	\end{tabular}
}

\subsubsection*{Testing the normality}\label{sec:normality}
The fifth assumption assumes that the dependent variable should be approximately normally distributed for each group. This assumption was tested before making the comparisons seen in table~\ref{table:outliers} as this assumption must hold in order to make the comparison valid. 
In order to check the this assumption hold we calculated the skewness and kurtosis values and their respective z-values, looked at the values produces by using Shapiro-Wilks test for normality and looked at the normal Q-Q plots for each group(see tables~\ref{table:skew} and ~\ref{table:shapiro}, and figures~\ref{fig:QQRobot},\ref{fig:QQRobotPruned} and ~\ref{fig:QQSimulator}).

\bigskip\noindent
\smalltable{Z-values for skewness and kurtosis}{table:skew}{
	\begin{tabular}{lllll}
		Group & Skewness & Kurtosis & Z-skewness & Z-kurtosis\\\hline
		Robot with outlier & -1.838 & 3.751 & -2.013 & 1.876 \\
		Robot without outlier & 0.855 & -1.289 & 0.843 & -0.492\\
		Simulator & 0.000 & -1.875 & 0.000 & -1.077 \\\hline
	\end{tabular}
}
\bigskip\noindent
\smalltable{Test for normality}{table:shapiro}{
	\begin{tabular}{l|lll|lll}\hline
		Group & \multicolumn{3}{l}{Kolmogorov-Smirnov} & \multicolumn{3}{l}{Shapiro-Wilk}\\\cline{2-4}\cline{5-7}
		& Statistic & df & Sig. & Statistic & df & Sig.\\\hline
		Robot with outlier & 0.376 & 5 & 0.020 & 0.788 & 5 & 0.065\\
		Robot without outlier & 0.283 & 4 & - & 0.863 & 4 & 0.272\\
		Simulator & 0.164 & 6 & 0.200 & 0.950 & 6 & 0.739\\\hline
	\end{tabular}
}
\image{images/results/QQPlotRobot.PNG}{\linewidth}{Normal Q-Q plot of gains for the robotics group with outliers}{fig:QQRobot}
\image{images/results/QQPlotRobotWithoutOutlier.PNG}{\linewidth}{Normal Q-Q plot of gains for the robotics group without outliers}{fig:QQRobotPruned}
\image{images/results/QQPlotSimulator.PNG}{\linewidth}{Normal Q-Q plot of gains for the simulator group}{fig:QQSimulator}

\bigskip\noindent
As seen by this(tables~\ref{table:skew} and ~\ref{table:shapiro}, and figures~\ref{fig:QQRobot},~\ref{fig:QQRobotPruned}, and ~\ref{fig:QQSimulator}) we can be fairly certain that the assumption regarding normality is not violated as all z-values are within the $\pm 2.58$ range (representing a statistical significance level of $0.01$) and the Shapiro-Wilk test shows $p > 0.05$ for both groups. 

\subsubsection*{Testing for homogeneity between variances}
The last assumption relates to homogeneity of variances within each group.
As mentioned when comparing the results with and without outliers, this assumption was violated when the outlier was excluded($p = 0.033$), but held when it was included($p = 0.863$). None of the results do however show a statistically significant difference, and it therefore fair to assume that there is no statistically difference between the two groups.

\subsection*{Analysis summary}
The independent-samples t-test was used to determine if there were differences in gains from pretest to posttest between the robotics group and simulator group. One outlier was identified within the dataset (figure~\ref{fig:boxplot}), but did not effect the decision of keeping or rejection the null hypothesis and was thus kept in the dataset during the rest of the analysis. 
The gains for each group were normally distributed as shown by z-values and Shapiro-Wilk's test ($p > 0.05$), and there was homogeneity of variances when the outlier was included($p = 0.863$). The increase from pretest to posttest was greater for the simulator group ($M = 2.00, SD = 3.13$) than for the robot group ($M = 0.40, SD = 2.37$ with the outlier included, and $M=1.75, SD = 0.96$ when it was excluded), but not a statistically significant difference, $M = 1.60$, $95\% CI [-2.14, 5.34]$, $t(9) = 0.967, p = 0.359$ with the outlier included, and $M = 0.25$, $95\% CI [-2.30, 2.80]$, $t(7.05) = 0.232, p = 0.823$ when it was excluded.

\bigskip\noindent
The analysis presented here have been using the gain from pretest to posttest directly as the dependent variable. A similar analysis was conducted using the normalized gain in order to explain the gain in relation to how much each participant could potensially improve to the posttest. After checking that no assumptions were violated, with the exception of assumption four (no outliers within the groups), we were able to gather the results. 

\smalltable{Results of independent t-test. * outlier removed}{table:fullIndependeny}{
	\begin{tabular}{llllll}
		Varable & t & df & Sig. & Mean diff & Std. Error Diff\\
		Gain & 0.967 & 9 & 0.359 & 1.6 ($-2.143..5.343$) & 1.655\\
		Gain* & 0.232 & 7.05 &  0.823 & 0.25 ($-2.296..2.796$) & 1.078\\
		NormalizedGain & -0.117 & 9 & 0.910 & -0.03 ($-0.565..0.509$) & 0.238\\
		NormalizedGain* & -1.903 & 8 & 0.094 & -0.26 ($-0.565..0.054$) & 0.134\\
		\hline
	\end{tabular}
}

%\bigskip\noindent
%\smalltable{Tests means for different groups}{table:means}{
%	\begin{tabular}{lllll}
%		Group & Pretest mean & Posttest Mean & Diff. Mean & Diff. SD\\\hline
%		Robot & 18.40 & 18.80 & 0.40 & 2.37\\
%		Simulator & 13.17 & 15.17 & 2.00 & 3.13\\
%		All & 15.54 & 16.82 & 1.27 & 2.72\\\hline
%		Robot, no outlier & 18.75 & 20.50 & 1.75 & 0.96\\\hline
%	\end{tabular}
%}
