\chapter{Process}
Everything we have done has been done since february. This made our process long and intense. We have conducted a pre-project to find a relevant angle on robotics in education and then we have focused further into the specific field of robotics in maths education.

\section{Literature Study on robotics in education}
In order to get an idea of the state of the art in robotics in education we performed a literature study on this topic. We also want to get an idea of what way we can introduce robotics to a school.

The literature study shows great promise for robotics in education. However few papers have cooperated with schools and pedagogs in an attempt to integrate or check how their ideas will integrate into the school curriculum. Many mention that robotics call for a change in the traditional curriculum according to the constructionism pedagogy. However calling for a huge change in schools is probably not the easiest way to introduce robotics to schools if this require lots of alterations. 

Another lack in research is an explanation of how they taught the students. They may mention that they implemented a 1 year curriculum but don't provide any examples of tasks they created. It is therefore very hard to build upon and do further work on it. 


\section{Deciding on a specific topic}
A lot of topics can be learned using robotics. We have focused on mathematics. It is said that all math can be taught in some way with robotics. We present examples for some of them.

We have chosen to focus on angles as it is very basic, people struggle with it and using logo uses angles to decide how much to turn. It is therefore a natural starting point for our investigation in mathematics with robotics. 

We also draw pedagogical reasoning behind angles and geometry from papert. The reason angles and geometry in general is easier is that a point rather than being defined as a place in space with no heading the robot that serves as the point here has a heading just like us humans. It can therefore be related to. 

\section{Literature Study on available programming languages and pedagogy}
In order to focus on the right points, learn about the pedagogic theory and testing setup, use a suitable approach and create a solid platform to work on we needed to conduct a literature review to find state of the art on educational robotics as well as pedagogic theory. 


\section{Design and Development}
In order to answer some of our research questions we need to develop accessories for the ChIRP as well as a tablet application for testing in an elementary school. Therefore we will design and develop a prototype with emphasis on expandability. We will find workarounds for the chirp that does not require major modifications to the chassis or default circuit board with sensors. We will also need to develop a module for bluetooth communication as we found that compiling and uploading code to the ChIRP is not suitable for every school grade and we don't know which grade we will be working with, another point is that this application again should be easily expandable and implementable for different grade students.  

The design and development of this software and hardware will be a very demanding phase. Designing a new application from scratch can be very time-consuming[39] - Another master (Mention hardware too). However we build upon LOGO so we have some guidance to what the functionality should entail. 

Again we will not have time to create an optimal solution as this would require testing and then redesign etc. We aim to create an easily manageable and changeable platform to build upon in future research. 
