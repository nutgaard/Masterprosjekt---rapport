\section{Approaches / How should we go about using robotics in schools?}
activities which are carefully designed to favor strategies that include math as a central rather than supplemental part of the activity have the best chance for achieving learning gains while sustaining engagement[Silk preface]. 

[does lego training stimulate] several researchers including Becker[1987] have stressed that the main evidence showing that LOGO can procude measurable learning when used in “discovery” classes has been obtained in situations close to individual tutoring. In normal-sized classes, the evidence clearly shows the need for direct instruction in the concepts and skills to be learned from LOGo, as well as further direct instruction to enable students to generalize what they have learned for transfer to other situations. This is in complete opposition to papert’s conception of the discovery approach to LOGO.

[does lego training stimulate] There needs to be a large space for the pupils to work, they must be able to spread the LEGO material on the ground, play around, and test different kind of solutions for each kind of project they face. The working groups should not be too big (max 2-3 per group). the task given to the pupils must be concrete, relevant and realistic to solve. it is very important that the pupils can relate the material to their ordinary school work and their different subjects. This test was done in a regular school and therefore addressed these problems. Many other studies does not attempt to do this and therefore is not that relevant in the question: how can we integrate this technology into schools. 

[does lego training stimulate] the advantage of two teachers working parallel in the classroom. This made it possible for one teacher to aid a specific group while the other maintained control over the classroom. 

[acquisition] Provides 4 principles for design and implementation of robotics programs. 

1: Just-in-time resources such as lessons, tutorials and examples should be embedded to support scientific inquiry and acquisition of content knowledge. The finding of the study is consistent with the literature. many researchers emphasize the importance of providing resources such as case examples, informative resources and various scaffolding tools to support student-centered learning. Future studies are needed to develop and research various types of resources that are important in robotics programs. 

2: Students should be encouraged to explain their design by citing related scientific concepts and principles during debriefings. Must ask students to use ecientific concepts to explain the design of their robots and the strategies that worked or did not work for them. 

3: A robotics program should provide opportunities for students to explore the learning environment but at the sam etime encourage them to follow the process of scientific inquiry to complete design challenges. Future research may be needed to examine how to maintain the balance between free play and problem solving structured by the scientific inquiry process. 

It is a challenge to designers to maintain the interest while also bridgning that interest into a depper, more conceptually productive form of engagement. 

Challenges

Focused content, how do they learn math not just play

Motivated activity, how do you get students to care

Accessible problems, how can students understand what they are supposed to do?

Useful resources, how to you provide the resources they need to solve it?
