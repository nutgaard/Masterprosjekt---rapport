\chapter{Experimental design}
The selected students were divided into two groups by the teacher in advance. 
The teacher created the groups with two factors in mind, equality in terms of content knowledge level and by which participants she knew could work well together.
%The teacher had divided the group by trying to divide the skill level equally between groups, she also wanted to avoid any conflict that might arise between students. 
We randomly assigned one group to the simulator activity and one to the robot activity. The robotics or experimental group (seven students: three boys and four girls, where one boy and one girl could not attend both sessions) was appointed to work with the physical robot activity. Considering that simulators have shown to be an efficient learning tool \cite{mitnik2009collaborative}, and \tcite{papert1980mindstorms} mention that there is not an obvious difference between mental model building in robots and simulators, the control group (eight students: one boy and seven girls, where two girl could not attend both sessions) worked with an on-screen simulation of the robot, seen from a top down perspective on the tablet(see figure~\ref{fig:appSimulator}). Other than this, the activity remained the same. 
The choice of control group activity was also influenced by our findings in the prestudy, stating that most studies use a control group that has not done any alternative activities, which we feel is a problem. 

\bigskip\noindent
%Each group either had a tablet or android phone. 
We altered which group started first because, as the teacher pointed out, when teaching, you always do best the second time, so we wanted to minimize this effect between the groups. The materials required for this experiment were the robots (one robot per group) for the robot group, the tablet or android phone (one tablet/phone per group) and protractors. Experimental and control groups had no interaction during the activity sessions, however some of the other simulator group members came over and watched the robotic activity sometimes. 
The experiment was conducted in the classroom with a teacher and the other group present in the room due to legal issues. 
%We were sharing the classroom with the teacher and the other group, because of legal issues. 
The students were free to choose their team mates once the activity started. They then stayed in these groups throughout both sessions. The tasks were the same in both groups.

\bigskip\noindent
Groups of two members were preferred. This decision was based on our findings in section \ref{sec:whatAdvantage}. When we discussed the experimental design with the teachers, they agreed there should be 2 students per group based on their experience. However to avoid the case of 1 student working alone and considering our lack of tablets, some groups had to contain 3 students.
