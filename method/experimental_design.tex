\chapter{Experimental design}
The selected students were divided into two groups. The robot or experimental group (7 students: 3 boys and 4 girls, where one boy could not attend the first session) was appointed to work with the physical robot activity. Considering that simulators have shown to be an efficient learning tool (graph potting activity We should also write much more about the choice of simulator for the control group) the control group (8 students: 1 boy and 7 girls, where one girl could not attend the first session) worked with an on-screen simulation of the robot seen from a top down perspective on the tablet. Other than this, the activity remained the same. Each group either had a tablet or android phone. The tasks were the same in both groups. We altered which group started first because as the teacher pointed out, you always do best the first time, so we wanted to equalize this effect. The materials required for this experiment were the robots (one robot per group) for the robot group, the tablet or android phone (one tablet per group) and protractors. Experimental and control groups had no interaction during the activity sessions, however some of the other groups came over and watched the other group sometimes. We were sharing the classroom with the teacher because of legal issues. The set of daily exercises was common for both groups. The students were free to choose their team mates once the activity started. They then stayed in these groups throughout both sessions. The teacher decided who was gonna be in what group by trying to divide the skill level equally between groups, she also wanted to avoid any conflict that might arise between students. Then we randomly assigned one group to the simulator activity and one to the robot activity. 

Groups of two members were preferred. This is consistent with other research of diminishing cooperation effect of groups more than 3 students but 2 specifically was chosen on the basis of what the teachers thought best through their experience with the students. However to avoid the case of 1 student working alone some groups had to be 3 students. There was also the case of not enough tablets to work with anyways.

We chose to use a quantitative analysis of the angles and turn measurement skills because of several factors. First off we could not do interviews as this would take too much time from the teacher but also, A qualitative data collection method might have been prefered in order to dig deeper into the students brains, but there was not enough time left in the school year for interviewing students. The advantage of quantitative is that we can do statistic analysis on it, even though it is a small sample size. We also included some qualitative questions in the test, and therefore we have a little bit of both. There were barely enough time to do this project. 
