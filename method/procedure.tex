\chapter{Procedure}
The study was divided into 2 sessions held on a friday and the following monday. The first session lasted 40 minutes and the second session lasted 45 minutes. The first session was delayed because we were waiting for an extra tablet which never showed up, forcing us to use a phone for one of the groups instead of a tablet. Each session consisted of a short review of what they are supposed to learn, followed by a set of 10 difficulty increasing robot exercises. In the first session there was also a short introduction of about 4 minutes explaining the software. During the exercise phase the students worked on their own, while we acted as tutors, solving doubts regarding software use and the activity itself. We also acted a bit as 'teachers', yet only when required by the students themselves or if they were stuck on a problem for far too long. We never gave them the answers or any code, we merely said it might help if you walk the route, or which angle are you measuring now? etc. Other than this the students received no external help or instruction. The exercises of the first session were relatively easy, since their main purpose was for the students to familiarize with the technology, and everyone had to be able to do something, but no one should be able to do everything. They were only paths where you don't end up where you started. In the second session every exercise was a closed shape. The students were free to choose their group members. Groups of two were prefered, however some groups had three members to avoid the case of one student working alone. 

\section{First lecture}
We arrived at the school 9:00. Then the students were already in the classroom ready to start, but the tablet we were going to borrow had not been brought to the school so we had to improvise and use one of our phones instead of the tablet. Therefore one group in each session used a phone instead of a tablet. The only difference is a smaller screen and it did not seem to make a difference, at least not on the robot group as they don't use the simulator there anyways. 

We started with the robot group. The first 10 minutes ca was supposed to be exactly the same. First a small introduction of how the application works, then a short introduction to supplementary angles and reflex angles. However everything is performed better the second time around and the introduction went faster and with fewer questions in the second group. We had already agreed to start with the simulation group first the next session because we had anticipated this.

Everything we said about supplementary angles was:

---------- 
Does anyone know how many degrees there are in a straight line?
 - 180

Yes what about if we insert a line here
\
   \
     \
----------
which is 45 degrees. What is the other angle then?
 - 135 
 
Yes. These two angles are called supplementary angles because together they create 180 degrees (a straight line). 

The reason you need to know this is because the robot turns in degrees. But you need to keep supplementary angles in mind if you run into problems. This is the main way we can calculate how the robot should turn. 

Everything we said about reflex angles was:

When two lines comes together at a point (draw 45 deg) there will always be two angles. One here (mark 45 deg) and one here (mark 315). Does anyone know how big this is? (the 315 deg)... Got all kinds of answers but waited till someone said 315. Yes that is correct. Do you know why? etc.

In a circle there are 360 degrees. If we cut out 90 degrees of it anywhere.. illustrate
Then the rest will be 360-90. 

The reason you need to know this is that we have some questions where we ask you to create alternative programs to make the same path. Keep this in mind then! 



We mentioned that it might help if you move your body as the program tells you. And they should cooperate. They should also try to understand why the program did not work before trying a new solution. 



They chose their own groups. The main groups was taken out by the teacher. 


Next followed a 32 minutes session where they experimented with the robot / simulator and tried to solve some sample tasks that we had printed out before the lesson. 

During the experimentation phase they were allowed to use protractors and pens. 

\section{Second lecture}
The second lecture started with a small recap on what we did last time with simple examples like last time and then a short talk introducing regular polygons mentioning that if you are to complete a shape then you need to turn all the way around during the run. If you turn all the way around that is 360 degrees, keep this in mind. The recap focused on supplementary and reflex angles.

Both groups had the small intro then followed 45 minutes of robot / simulator activity. 