\section{Procedure}
The study was divided into two sessions, one held on a Friday and one on the following Monday. The first session lasted forty minutes and the second session lasted fortyfive minutes. 
One of the groups had to use a phone instead of a tablet due to unavailability of one tablet. 
The only difference of our application when used on a phone compared to a tablet is a smaller screen, and it did not seem to make a difference, at least not on the robot group, as they don't use the simulator, which is the part requiring a larger screen. Each session consisted of a short review of what they were supposed to learn, followed by a set of 10 difficulty increasing robot exercises. In the first session there was also a short introduction of about four minutes explaining how the the software worked. During the exercise phase the students worked on their own, while we acted as tutors, solving doubts regarding software and the activity itself. If groups were stuck on a single task without making progress we provided subtle hints to help them along. 
%We also gave hints about the tasks, yet only when required by the students themselves or if they were stuck on a problem for far too long. 
We never gave them the answers or any code, but explained that visualizing the program block by block and think about the effects on the robot. If it was clear that they measured the wrong angle we helped them get on the right track by having them explain how the robot moved. 
%We never gave them the answers or any code, we merely said it might help if they go through the program block by block and think about what will happen to the robot, or asked them to show us which angle they were measuring and why they were measuring that angle. 
Other than this the students received no external help or instruction. The exercises of the first session were relatively easy, since their main purpose was for the students to familiarize with the technology. The exercises was designed so that everyone should be able to do something, but no one should be able to do everything. Only open polygons, or paths, where you don't end up where you started, were used in the first lecture exercises. This was done so that students would not get caught up in the precision of the robot or the length of movement. In the second session every exercise was a closed shape, and the students had be careful of how far the robot moved in forward or backward block.  

\subsection{First lecture}
The robotic activity was the first learning activity that the students in group 1 did on that day. 
%This may have a positive effect on their learning. 
Group 2 did other activities, not related to our work, with the teacher  while they waited for their turn.

\bigskip\noindent
We started with the robot group. The first ten minutes was supposed to be exactly the same. First a small introduction of how the application works, then a short introduction to supplementary angles and reflex angles. However everything is performed better the second time around and the introduction went faster and with fewer questions in the second group. We had already agreed to start with the simulation group first the next session because we had anticipated this.

\bigskip\noindent
During the introduction we explained that they had to cooperate and let everyone try at some point. They should also try to understand why the program did not work before trying a new solution to avoid an trial and error approach.

\bigskip\noindent
Next followed a thirtytwo minutes session where the students experimented with the robot / simulator and tried to solve some sample tasks. The tasks are included as appendix~\ref{appendix:experiment1}.
The tasks met our expectations, as everyone managed at least four tasks and the best group finished the last task right before the session ended. When solving these tasks, the students were allowed to use protractors and pens. 

\subsection{Second lecture}
To keep things consistent with the first lecture we used one phone as substitute for the missing tablet.

\bigskip\noindent
The session was the first learning activity that the students in the simulator group did on that day. 
%This may have a positive effect on their learning. 
The roboitcs group had physical education before we swapped groups.

\bigskip\noindent
The second lecture started with a small recap on what we did last time. Then followed a short talk introducing regular polygons. We mentioned that, in order to complete a shape and make the robot end up where it started, it needs to turn all the way around during one program run. The students understood that this meant turning $360\,^{\circ}$. The recap focused on supplementary and reflex angles. Then followed fortyfive minutes of robot / simulator task solving activity. The tasks are included as appendix~\ref{appendix:experiment2}.
As in the first lecture the tasks we created for the second lecture also met our expectations, as everyone managed at least four tasks and the best group finished the last task right before the session ended. 