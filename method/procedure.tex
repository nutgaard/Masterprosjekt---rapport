\chapter{Procedure}
The study was divided into 2 sessions held on a friday and the following monday. The first session lasted 40 minutes and the second session lasted 45 minutes. The first session was delayed because we were waiting for an extra tablet which never showed up, forcing us to use a phone for one of the groups instead of a tablet. The only difference is a smaller screen and it did not seem to make a difference, at least not on the robot group as they don’t use the simulator there anyways. Each session consisted of a short review of what they were supposed to learn, followed by a set of 10 difficulty increasing robot exercises. In the first session there was also a short introduction of about 4 minutes explaining the software. During the exercise phase the students worked on their own, while we acted as tutors, solving doubts regarding software use and the activity itself. We also acted a bit as “teachers”, yet only when required by the students themselves or if they were stuck on a problem for far too long. We never gave them the answers or any code, we merely said it might help if you walk the route, or asked them to show us which angle they were measuring and why they were measuring that one. Other than this the students received no external help or instruction. The exercises of the first session were relatively easy, since their main purpose was for the students to familiarize with the technology. The exercises was designed so that everyone should be able to do something, but no one should be able to do everything. Only open polygons or as we call them paths, where you don’t end up where you started, were used in the first lecture exercises. In the second session every exercise was a closed shape. 


\section{First lecture}
We arrived at the school 9:00. Then the students were already in the classroom ready to start. 

\bigskip\noindent
We started with the robot group. The first 10 minutes ca was supposed to be exactly the same. First a small introduction of how the application works, then a short introduction to supplementary angles and reflex angles. However everything is performed better the second time around and the introduction went faster and with fewer questions in the second group. We had already agreed to start with the simulation group first the next session because we had anticipated this.

\bigskip\noindent
During the introduction we mentioned that they should cooperate and let everyone try. They should also try to understand why the program did not work before trying a new solution to avoid trial and error. When the students ran into problems, we mentioned that it might help if you move your body step by step according to the program.

\bigskip\noindent
Next followed a 32 minutes session where the students experimented with the robot / simulator and tried to solve some sample tasks that we had printed out before the lesson. The tasks are included as appendix C.

\bigskip\noindent
During the experimentation phase they were allowed to use protractors and pens. 

\bigskip\noindent
The tasks met our hopes and expectations as everyone managed at least 4 tasks and the best group just finished the last one as we ran out of time. 

\section{Second lecture}
To keep things consistent with the first lecture we used our phones for this lecture as well. We did not hear anything new about the tablet anyways. 

\bigskip\noindent
The second lecture started with a small recap on what we did last time with simple examples from last time. Then followed a short talk introducing regular polygons mentioning that if you are to complete a shape then you need to turn all the way around during the run. We also asked how many degrees you have to turn to get all the way around. They answered 360 degrees, which is correct. Then we said that if you are going to end up where you started you have to turn all the way around. The recap focused on supplementary and reflex angles.

\bigskip\noindent
Both groups had the small intro then followed 45 minutes of robot / simulator activity. 

\bigskip\noindent
As in the first lecture the tasks we created for the second lecture also met our hopes and expectations as everyone managed at least 4 tasks and the best group just finished the last one as we ran out of time. 