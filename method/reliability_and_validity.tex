\section{Validity and Reliability in the experiment}
	In the literature section (section~\ref{sec:reliabilityLiterature}) we provided an introduction to some of the common threats to reliability and validity. This section will try to tie these potensial threats to the experiment conducted during this project.
	
	\bigskip\noindent
	In the introduction to reliability we mention four key concepts that often are used to describe how reliable an experiment is. The \textit{interrater reliability} is in our case not relevant as it addresses the differences between two researchers rating the participants using qualitative measures. Since this project relies on quantitive data and the only introduction for subjective opinions to contaminate the results was during the scoring process for the pretest and posttest. In order to mitigate this we conducted the scoring process side by side, setting up strict rules for what we would consider a valid answer (e.g $\pm 15^{\circ}$ for tasks involving estimation).
	
	\bigskip\noindent
	The remaining three concepts, \textit{internal consistency}, \textit{equivalence reliability} and \textit{stability reliability}, are however of interest to this project. 
	For a more detailed description of the estimation process we refer you to chapter~\ref{ch:cronbach}, where the results from the pretest and posttest are analysed using Cronbach's alpha and Pearson's product-moment correlations. The result from these analyses showed a alpha value above $0.8$ for the pretest and posttest, above $0.9$ for all the results combined, and a strong positive correlation between the pretest scores and posttest scores. These analyses indicate a high degree of internal consistency and stability.
	
	\bigskip\noindent
	The discussion related to validity is in many ways more complicated then that of reliability as many of the threats to validity can only be verified through being critical towards yourself, and not a scientific analysis. 
	
	\bigskip\noindent
	One of the benefits to validity during this project is the relatively short timespan between the pretest and posttest, which help mitigate several of the threats to validity. This is however a double-edged sword as the short timespan may make some threats more likely to reduce the validity. In our opinion the threats that were most likely to have been interfering with our study are the \textit{testing effect}, \textit{selection bias} and \textit{diffusion between groups}\footnote{We use this as an umbrella term for when participants interact or are affected by eachother}. 
	Some threats that haven't been mentioned during the introduction as \textit{history effects}, \textit{maturation}, etc. have intentionally been overseen as these effects are more likely to happen during longitudinal studies and thus it is very unlikely that they have played a role during this project. 
	
	\bigskip\noindent
	The testing effect, more specificly the learning effect, may however have played a role during experiment as we used the same test for both the pretest and the posttest. In order to mitigate this effect there was not given any feedback on the pretest results to the participants and the pretest was given a week in advance of the experiment itself. We grew more confident that this effect had not played a role in our experiment when comparing the pretest results with the posttest results, as it was seen that the participants in some cases would get a question wrong on the posttest where they had previously answered correctly on the pretest. 
	
	\bigskip\noindent
	Selection bias is also a potensial threat to this study as the groups were put together by the teacher and then randomly assigned to either the robotics group or the simulator group. The analysis in chapter~\ref{ch:independentttest} showed that the pretest mean score of the robotics group were five points higher on average (table~\ref{table:means}). The groups were created by the teachers as at the time when the groups were created we were left with very few students, and any random assignment could possibly skew the distribution of highly skilled participants into one group. It was therefore suggested that the teacher, which have an in-depth knowledge about each participant and who they may work well with. The analyses in part~\ref{part:results} (\nameref{part:results}) shows that the average gain between the groups were very equal if outliers were pruned from the data set. This may indicate that the gains were equals between the group, but it may also indicate that the selection bias promoted more learning in one group over the other. There is no way of determining the effect of this, we therefore urge the reader to keep this fact in the back of the head. 
	
	\bigskip\noindent
	The last big potensial threat indentified to this experiment was the umbrella term \textit{diffusion between group}. This effect may have manifested itself throughout several different ways, it could be simple interactions between the participants in the groups during the break time or during the weekend between the two sessions. Another possibility is that the participants of one group became more motivated or less excited because of the other group, e.g. the participants of the simulator group may have given the little extra in order to show that they should be allowed to use the robot, or (perhaps more likely) they may become demoralized because they were in the simulator group and felt that the robotics group was the better alternative.
	During the sessions we had some indications that some of the participants in the simulator group felt unjustly placed in the simulator group, and thus may not have utilized their full potensial. 
	This was addressed by the teacher with the children after the first sessions, but we are uncertain about how this may have affected the results. 