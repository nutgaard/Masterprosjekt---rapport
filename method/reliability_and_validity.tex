\chapter{Reliability and validity}
Reliability and Validity are important topics within all types of research. 
These topics provides our discussion for whether we are measuring what we think we are measuring and whether these measures can be viewed as  consistent and valid. 
Reliability and valididy comes in many different shapes and form, all measuring different aspect regarding how the study was conducted.

\section{Reliability}
	When talking about reliability we usually talk about the four key concepts, \textit{equivalence reliability}, \textit{stability reliability}, \textit{internal consistency} and  \textit{interrater reliability}. 
	
	\subsection{Equivalence reliability}
	Equivalence reliability is the extent to which two items measure identical concepts at an identical level of difficulty~\cite{colostateReliability}. This is often measured as the correlation coefficient, measuring the strength of the correlation between the dependent variable and the independent variabel. 
	
	\subsection{Stability reliability}
	Stability reliability is a measure of the instruments stability. To test an answer to this, one would usually repeat a given test to see if it gives the same results.
	
	\subsection{Internal consistency}
	Internal consistency is a measure of how well an instructument measures the same underlying concepts. A common way of measuring this is to use Cronbach's alpha.
	
	\subsection{Interrater reliability}
	Interrater reliability is the extent to which raters agree, and is used as a measure of the rating system. 
	
\section{Validity}
	Validity assesses the degree of which an experiment and design measures the concept that the researchers intended to measure.
	When talking about reliability, we usually talk about \textit{internal validity} and \textit{external validity}, each with its own subcategories like \textit{face validity}, \textit{construct validity} and \textit{content validity}.
	Internal validity refers to how the study was design, organized and conducted. While external validity looks at how a study can be generalized and transferable.
	
	\subsection{Internal validity}
	Common threats to internal validity include \textit{testing effects}, \textit{statistical regression}, \textit{selection bias}, \textit{experimental mortality}, \textit{diffusion between groups} and so on. 
	
	\subsection{External validity}
	External validity suffers from some of the same threats as internal validity (selection bias etc.), but selection bias internally and externally may infact be two very different threats when conducting a study.
	Some of the more unique threats to external validity is the \textit{"`real-world"' versus "`experimental world"'} and \textit{"`faulty construts"'} threats, where the first refers to when participants are aware that they are part of a study and may therefore alter their behaviour because of this. The latter can be a bit more subtle, but refers to how well construct have been narrowed down from concepts, and how these constructs are measured.

Our test might suck! We should address this here

Ideally we might have wanted to split the groups randomly. 
However the teacher said that as she knew the skill level of everyone and to avoid conflicts we should let her assign the groups. 
We saw no major problem with this, and as we have no experience with teaching and are not even familiar with the kids we thought it best to grant her wish.

The tests reliability and internal consistency should be determined by 'KR-20' coefficient maybe. 

Possible problems
The teacher teaches the other group in the same classroom as we are doing the tests because of legal issues. 
Maybe they overhear things etc. 

Maybe we should have picked random student groups, however we saw no reason to do this. 
Particularly because everyone knows each other and everyone matched up good. 

We left out the top scoring students from the study because they have nothing to learn and the teacher was concerned that if they were included that they would help the other students a lot and we want to test the robot mainly, not how they help. 
