\chapter{Reliability and validity}
Reliability and Validity are important topics within research. These topics provides our discussion for whether we are studying what we think we are studying and whether the measures are consistent. 
Reliability and validity are important in the quality assessment of scientific investigations. Reliability means that the data is as reliable as possible. Reliability is high if independent measurements give the same results. It aims to have as few as possible error sources in the test. 

Our test might suck! We should address this here

Ideally we might have wanted to split the groups randomly. However the teacher said that as she knew the skill level of everyone and to avoid conflicts we should let her assign the groups. We saw no major problem with this, and as we have no experience with teaching and are not even familiar with the kids we thought it best to grant her wish.

The tests reliability and internal consistency should be determined by �KR-20� coefficient maybe. 

Possible problems
The teacher teaches the other group in the same classroom as we are doing the tests because of legal issues. Maybe they overhear things etc. 

Maybe we should have picked random student groups, however we saw no reason to do this. Particularly because everyone knows each other and everyone matched up good. 

We left out the top scoring students from the study because they have nothing to learn and the teacher was concerned that if they were included that they would help the other students a lot and we want to test the robot mainly, not how they help. 
