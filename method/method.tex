The inspiration behind the study was a wish to test the robot as an educational tool in an experimental classroom setup. We wished to increase students' understanding of angles and their angle estimation skills. We also wished to increase the students' cooperation and motivation. As we mentioned in chapter~\ref{chap:constructionism}, the pedagogic theory we rely on is constructionism, which is quite different from the blackboard and task driven mathematics education, which according to \tcite{alseth2003endringer} happens in most of the classrooms in Norway, including the classrooms at THIS. %When we talked to the teacher at THIS, it seemed like this is the prevalent method there as well. 
\chapter{Choice of methods}
Most research methods can be placed in one of two categories: quantitative or qualitative methods. Qualitative research is often used to get an insight into how students perceive something, and their thoughts and feelings about the subject. It is also used to understand their behavior. Quantitative methods have a focus on finding the general correlations, and discover tendencies that give interesting insights for future research or can be supported through other methods. This is done by pointing out special factors that are the cause of a given tendency \cite{tjora2012kvalitative}. If you want a broad understanding of a phenomenon you should use quantitative methods. Testing a hypothesis statistically is an example of quantitative research methods. 

\section{Test}
Tests are commonly used  in schools to find out what the students know and what they have learned. All the papers presented in the pre study applied a pretest and a posttest to support their findings. This is due to one of the literature study inclusion criteria. We wanted statistically significant results to support our decisions. 

\bigskip\noindent
One advantage of using a test is that students are used to being tested in this manner. It is relatively easy and efficient to get an overview of what the students know, or don't know and what they have learned.

\bigskip\noindent
Using tests alone has limitations to what can be measured. It is hard to test for deeper kinds of knowledge, as test questions often have only one right answer. It only test whether students know the answer or not, even though they might have managed the task with a little help. For example if a student did not understand the question due to language limitations. With this in mind the question about tests are: does this test really reflect what they understand and what is outside their understanding \cite{andersen2009evaluering}. A way to work around this would be to include several items in the test that measures the same underlying knowledge. Interview or an oral test can also be used in this case to get better knowledge of their understanding.

\bigskip\noindent
A test should be built on a solid knowledge within the given area to make sure it is accurately reflects the specific concept the researcher is attempting to measure. In other words the test should be valid. There has been done little research in this area which makes it harder to create a valid and reliable test.

\section{Observation}
Observation is used especially within social anthropology. This type of observation is a set of methods where the researcher participates, openly or hidden, in people's daily lives in a given period of time\cite{tjora2012kvalitative}. In the same way observation can be used in a certain context. The goal of observation is often to see with your own eyes what happens to a person or group in a given situations or under certain circumstances. According to \cite{thagaard2003systematikk} observation is particularly suited to give information about people's behavior and how they interact with each other, thus it is a good option to measure cooperation. 

\bigskip\noindent
Observation can be used as the only method in an experiment, but can be, and is often, used as a supplement to other methods. We distinguish between open and hidden observation and spectator, partially participating or completely participating observation. In hidden observation the researcher is present and observing, without the observee being informed of this. The advantage of this type of observation is that the participants don't get influenced and change their behavior as a result of being observed. The downside may be ethical issues and considerations. The difference between non-participating and participating observation is well illustrated by the difference in observing a handball match as a player or spectator. 

\section{Interview}
Interview is used in situations where you want to study meaning, attitude and experience\cite{tjora2012kvalitative}. According to \cite{thagaard2003systematikk} interview is particularly well used to give information about a person's experience, point of view and self understanding. In an interview situation the researcher often tries to create a relaxed atmosphere, so that the interview resembles a normal conversation revolving around some chosen topics. During the conversation the researcher leads the conversation by asking open or closed questions. The goal is often to get the interview object to talk as freely as possible. In this way it becomes clear what the interview subject means. Their attitudes and experiences comes forward. 

\bigskip\noindent
An advantage of using interview is the possibility to ask follow up questions and to come close to the interview object.
A disadvantage is that it takes a lot of time and the subject might feel less anonymous.

\bigskip\noindent
Students might have difficulties with the way questions are written, there can for example be language barriers, and with a short explanation they might be able to answer questions in an oral test that they would have answered incorrectly in a written test. In an oral test there are also opportunities to argue why they approached the way they did and how they solved the task in greater detail. We get to understand their thought process and understand what they have actually misunderstood and understood and not just misinterpreted.  

\section{Questionnaire}
Questionnaires are in general considered a quantitative method. \cite{ringdal2001enhet} writes that a questionnaire is a systematic method to collect data from a sample of people. This is done to give a statistical description of the population the sample is collected from. The questionnaire is standardized, which means that everyone get the same questions asked in the same way. A questionnaire is often used to find overall connections and tendencies by pinpointing special factors that are the cause of a given tendency \cite{tjora2012kvalitative}. 

\bigskip\noindent
An advantage of using a questionnaire is that you can perform it on many students at the same time. You also obtain data that you can use for statistical testing and draw conclusions about correlations on. 

\bigskip\noindent
One disadvantage is that there is no possibility for follow up questions. It might be possible to find connections and correlations between questions, but it is hard to understand why without any other methods as supplements.

\section{Experiment}
Experiment is the classical scientific design in science, but it is also used in medicine and in social science under different names. In our case we call it experiment, in other cases it is called trial or intervention \cite{ringdal2001enhet}. In an experiment the researcher controls the difference between two treatments, X and Y, by deciding when the experimental group should be exposed to X. Control of other factors are secured through randomization. This gives a good foundation for explaining cause and effect. In social science, experiments cannot be performed in such a controlled way. It is impossible to control all the factors that can influence a person or group. But still some sort of experiment, trial or intervention can give valuable knowledge which is difficult to attain through other methods.

\bigskip\noindent
The main advantage of an experiment is the possibility  to test out something new on real people and see how it works.
A disadvantage is the disturbance of the trial population. We intervene with their daily activities. One possible solution is that the teacher runs the experiment. Else wise, not only is the activity new, but new people are introduces to the students.

\section{Choice}
The choice of methods are selected based on the research questions and different practical considerations. We chose a mix of quantitative and qualitative methods. The choice was influenced by our research questions, the prestudy, the teacher and time limitations. The quantitative methods was meant to aid in our attempt to answer the research question: is there a learning difference in students understanding of angles. The qualitative methods should aid in pinpointing strengths and weaknesses in the robot compared to a simulator as well as supplementing the quantitative data findings. 

\bigskip\noindent
For our first research question ``Can using robotics in school help students understand what an angle is and aid in their angle estimation skills?'', we wanted to use interview, test and observation. This decision was based on related work discovered through the prestudy, but also our own beliefs that an individual construct their own knowledge within their social environment by building mental models which they can use to think with. We would have liked to use interviews because of its ability to ask followup questions and get a deeper understanding of how the robotics changed the way students think. However there was not enough time to do an interview because the experiment was performed near the end of the school year. The experiment was performed so late in the semester because it was hard for us to get in contact with a school that was willing to let us run the experiment. The students schedules were full and would therefore miss something important if they had to be taken out of the class to be interviewed. A test can be administered by the teacher and all the students can do it at the same time. An interview would have taken longer, as we would have needed to interview one student at a time. We tried to compensate for this lack in data by including questions where the students have to elaborate on how they got their answers on the test. We also wanted to observe the class. It is hard to observe without intervening and still understand what the students are thinking, but through watching how they used their protractors we could identify which angle they were measuring, and that shows whether they have understood it or not. We chose testing because this is how students are used to getting tested.

\bigskip\noindent
For the second research question ``What are the strength and weaknesses of using a robot compared to a simulated environment?'',  we wanted to use observation, interview and questionnaire. We wanted to use observation to supplement the quantitative results, but also to assess the cooperation and motivation, even though this is hard to observe. We want to test factors found through the prestudy, that might contribute to the robot activity being better than the simulator activity. We will test this through a questionnaire and in-class observations and focus on the important factors of mental models, motivation, cooperation and problem solving approach. We also gather qualitative data about the tasks by asking students to answer how they thought when attempting different tasks in the test. With these results, we will try to amplify the quantitative results and provide another basis for us to draw our conclusions. 



\chapter{Participants}
To analyze our research questions, we tested the robots with a group of fifth-grade students from the international school in Trondheim(THIS). A group of 15 students (4 boys and 11 girls) was selected from a group of 20 to work with the Robotics activity. 3 of the 20 students was on vacation during the experiment period. 2 of the 20 did a full score on the pretest and their approach to the problems, as described in the depth questions, were perfect. We therefore concluded that our activities could not teach them anything new. The teacher suggested that we don't include these 2 students. She meant that if they were in the group their ideas and help would dominate the learning gains of the students. Since we want to test the robot activity's ability to teach students the material, these students might interfere with the results. It turned out that if we included them, there would not be enough tablets for every group anyways, so these students were dropped from the study as well, leaving a total of 15 students. All the students were in the same class and either 10 or 11 years old. They had been taught about angles one month prior to our experiment, using a traditional blackboard based teaching scheme. Thus every student had prior knowledge of angles and angle estimation. The next thing they were going to learn was about reflex angles and area of shapes. None of the students had worked in this experimental manner before or worked with this specific kind of technology before.

\bigskip\noindent
We originally wanted to include only students who knew nothing about angles. This class was chosen because it was nearing the end of the school year for THIS and this was the only available class. It would have been a perfect choice if the experiment had been performed one month earlier, when they were first introduced to the concepts of angles and degrees. Another alternative would be to do this experiment after the summer vacation, so the knowledge they gained would have had time to sink in and not get different ideas confused or altered by our experiment. 
\chapter{Experimental design}
The selected students were divided into two groups by the teacher in advance. The teacher had divided the group by trying to divide the skill level equally between groups, she also wanted to avoid any conflict that might arise between students. Then we randomly assigned one group to the simulator activity and one to the robot activity. The robot or experimental group (7 students: 3 boys and 4 girls, where one boy could not attend the first session) was appointed to work with the physical robot activity. Considering that simulators have shown to be an efficient learning tool \cite{mitnik2009collaborative}, and \cite{papert1980mindstorms} mention that there is not an obvious difference between mental model building in robots and simulators, the control group (8 students: 1 boy and 7 girls, where one girl could not attend the first session) worked with an on-screen simulation of the robot, seen from a top down perspective on the tablet. Other than this, the activity remained the same. The choice of control group activity was also influenced by our findings in the prestudy, stating that most studies use a control group that has not done any alternative activities, which we feel is a problem. Each group either had a tablet or android phone. We altered which group started first because, as the teacher pointed out, when teaching, you always do best the first time, so we wanted to equalize this effect. The materials required for this experiment were the robots (one robot per group) for the robot group, the tablet or android phone (one tablet per group) and protractors. Experimental and control groups had no interaction during the activity sessions, however some of the other simulator group members came over and watched the robotic activity sometimes. We were sharing the classroom with the teacher and the other group, because of legal issues. The students were free to choose their team mates once the activity started. They then stayed in these groups throughout both sessions. The tasks were the same in both groups.

\bigskip\noindent
Groups of two members were preferred. This decision was based on our findings in section \ref{sec:whatAdvantage}. When we discussed the experimental design with the teachers, they agreed there should be 2 students per group based on their experience. However to avoid the case of 1 student working alone and considering our lack of tablets, some groups had to contain 3 students.

\chapter{Procedure}
The study was divided into 2 sessions held on a friday and the following monday. The first session lasted 40 minutes and the second session lasted 45 minutes. The first session was delayed because we were waiting for an extra tablet which never showed up, forcing us to use a phone for one of the groups instead of a tablet. Each session consisted of a short review of what they are supposed to learn, followed by a set of 10 difficulty increasing robot exercises. In the first session there was also a short introduction of about 4 minutes explaining the software. During the exercise phase the students worked on their own, while we acted as tutors, solving doubts regarding software use and the activity itself. We also acted a bit as 'teachers', yet only when required by the students themselves or if they were stuck on a problem for far too long. We never gave them the answers or any code, we merely said it might help if you walk the route, or which angle are you measuring now? etc. Other than this the students received no external help or instruction. The exercises of the first session were relatively easy, since their main purpose was for the students to familiarize with the technology, and everyone had to be able to do something, but no one should be able to do everything. They were only paths where you don't end up where you started. In the second session every exercise was a closed shape. The students were free to choose their group members. Groups of two were prefered, however some groups had three members to avoid the case of one student working alone. 

\section{First lecture}
We arrived at the school 9:00. Then the students were already in the classroom ready to start, but the tablet we were going to borrow had not been brought to the school so we had to improvise and use one of our phones instead of the tablet. Therefore one group in each session used a phone instead of a tablet. The only difference is a smaller screen and it did not seem to make a difference, at least not on the robot group as they don't use the simulator there anyways. 

We started with the robot group. The first 10 minutes ca was supposed to be exactly the same. First a small introduction of how the application works, then a short introduction to supplementary angles and reflex angles. However everything is performed better the second time around and the introduction went faster and with fewer questions in the second group. We had already agreed to start with the simulation group first the next session because we had anticipated this.

Everything we said about supplementary angles was:

---------- 
Does anyone know how many degrees there are in a straight line?
 - 180

Yes what about if we insert a line here
\
   \
     \
----------
which is 45 degrees. What is the other angle then?
 - 135 
 
Yes. These two angles are called supplementary angles because together they create 180 degrees (a straight line). 

The reason you need to know this is because the robot turns in degrees. But you need to keep supplementary angles in mind if you run into problems. This is the main way we can calculate how the robot should turn. 

Everything we said about reflex angles was:

When two lines comes together at a point (draw 45 deg) there will always be two angles. One here (mark 45 deg) and one here (mark 315). Does anyone know how big this is? (the 315 deg)... Got all kinds of answers but waited till someone said 315. Yes that is correct. Do you know why? etc.

In a circle there are 360 degrees. If we cut out 90 degrees of it anywhere.. illustrate
Then the rest will be 360-90. 

The reason you need to know this is that we have some questions where we ask you to create alternative programs to make the same path. Keep this in mind then! 



We mentioned that it might help if you move your body as the program tells you. And they should cooperate. They should also try to understand why the program did not work before trying a new solution. 



They chose their own groups. The main groups was taken out by the teacher. 


Next followed a 32 minutes session where they experimented with the robot / simulator and tried to solve some sample tasks that we had printed out before the lesson. 

During the experimentation phase they were allowed to use protractors and pens. 

\section{Second lecture}
The second lecture started with a small recap on what we did last time with simple examples like last time and then a short talk introducing regular polygons mentioning that if you are to complete a shape then you need to turn all the way around during the run. If you turn all the way around that is 360 degrees, keep this in mind. The recap focused on supplementary and reflex angles.

Both groups had the small intro then followed 45 minutes of robot / simulator activity. 
\chapter{Data collection}
In order to assess the learning gains accomplished by the students in angle and turn measurement estimation, a pretest-posttest design was used. The pretest was administered one week before the first session, and the posttest was administered an hour after the second session. The tests were individual and the students were not allowed to cooperate. Both tests were administered by the teacher. Since we could not find a standardized test that measure angle skills with well defined reliability and internal consistency we had to make our own. We got inspiration from Clements' research in LOGO. In his research he has created several tests to measure geometry learning gains. These tests have been used in several projects before us. We looked at a subset of these that specifically target angles and turn measurement. We then expanded and improved(focused for our group) these questions in a session with the teacher, with regards to what the students already knew and what they are supposed to learn next. We also thought back to when we were at school ourselves for help. The finished test consist of 24 questions, divided into 7 categories. The score range from 0 to 24 and was designed to be completed in 30 minutes. We used the same tasks for both the pre and post test so we can look at differences. The test does not include questions specific to LOGO or the robots, but are general questions to measure angle understanding. To aid the reader the questions are included in the results section when we go through each category. The factors identified in the pre-study was measured based on qualitative observations and quantitative results obtained by means of a post activity survey. Each of the survey's questions had five possible answers; --, -, 0, + and ++; "`--"' meant "`very little"' while "`++"' ment "`very much"'.

\bigskip\noindent
Regarding collaboration, the in-site observations were focused on the following four factors that have shown to foster effectiveness in collaborative learning settings (people from a graph teaching).

\bigskip\noindent
Observation were also focused on the other sustainable learning factors. We also focused on how the students seemed to measure the angles with the protractors and how they solved the tasks in general. In addition we looked for potential problems and ideas of what might be better explained etc. We did this because we are doing an intro master and this is supposed to be a platform to build on. Because of this future work is necessary to focus on. 

\bigskip\noindent
Finally, in order to determine whether the learning outcomes of these activities were dependant of the students' previous knowledge, the difference between the posttest and pretest results were analyzed in relation to the pretest scores.

\section{Test}
As mentioned we had to create our own test to assess learning gains in the topic angle and turn estimation~(Appendix~\ref{appendix:pretest}). Here we will go through step by step some of the choices taken when creating the test and what they are supposed to measure. We will also mention how these questions were scored for use in statistical analysis. No partial scores were given.

\bigskip\noindent
Question 1 was an open ended question meant to assess students understanding of what an angle is and their ability to explain it. There were given no points for this question.

\bigskip\noindent
Question 2 and 3 asked the students to draw an angle and then to draw a bigger angle. Each worth a single point. They were then asked to describe why the second angle is bigger. There were given no points 
for this answer, it was meant as a supplement to draw conclusions on.

\bigskip\noindent
Question 4 was aimed at assessing students ability to differentiate between angles and identify which is bigger in each pair. One point was given for each correct choice. There are 3 normal, or inner, angles and 2 angles which has highlighted the outer angle. These 2 outer angles were added because of the teachers advice, that this is what the students are supposed to learn next. In the first pair, one of the angles has longer "`legs"' and seeks to confuse students who don't understand that angle size is measured according to how wide it is, not how much space it occupies. In the second pair, one angle is turned upside down to appear smaller than it actually is and will test students who think orientation matter. In the third pair the angles are very small and will test the students general estimation skills, in case they cannot answer correctly on the two previous pairs. The 4th and 5th pair are outer angles. The 4th pair has an orientation change to confuse the students while the 5th pair is straight forward to test if they understand outer angles. 

\bigskip\noindent
Question 5 was aimed at angle calculation skills. There are a certain set of rules the students are supposed to be familiar with already, which we think will be reinforced by our experiment. Each sub task is worth 1 point each. In addition they are asked to explain how they got their answer, this is not worth any points.

\bigskip\noindent
Question 6 tests the students estimation skills. 2 of the questions have a multiple choice format, while in the 2 last the students need to write an estimate on their own. Each correct choice is worth 1 point. In the tasks where they have to write their own estimate we have an error margin of plus minus 15 degrees. The first task is a straight angle which they should be very familiar with. The next has its orientation changed, so does the 3rd. In the 4th angle estimation task the angle is very wide compared to the previous tasks. 

\bigskip\noindent
Question 7 is the same as question 6, but with outer angles. This will test their ability to measure the outer angle and understanding of what the circle drawn on the angles mean. Each correct choice is worth 1 point and there is an error margin of plus minus 15 degrees. The angles are the same as in question 6, but have had their orientation changed and their order changed. 

\bigskip\noindent
Question 8 tests the students ability to orient. They must imagine themselves in the position and with the direction of the arrow and decide which direction they want to turn. Then they need to estimate the amount of degrees to turn in order to face the black dot. One point is given if the direction and amount of degrees is correct. There is an error margin of plus minus 15 degrees. 

\bigskip\noindent
Question 9 and 10 are also turn estimation tasks. The students must keep track of how much the robot has turned at every step and in which direction it is currently facing. In question 9 one point is given if the answer is 12. In question 10 one point is given if the amount of degrees are 50. 

\section{Observation}
We had focus on teamwork during the observation, also problem solving approach. 
\chapter{Reliability and validity}
Reliability and Validity are important topics within all types of research. 
These topics provides our discussion for whether we are measuring what we think we are measuring and whether these measures can be viewed as  consistent and valid. 
Reliability and valididy comes in many different shapes and form, all measuring different aspect regarding how the study was conducted.

\section{Reliability}
	When talking about reliability we usually talk about the four key concepts, \textit{equivalence reliability}, \textit{stability reliability}, \textit{internal consistency} and  \textit{interrater reliability}. 
	
	\subsection{Equivalence reliability}
	Equivalence reliability is the extent to which two items measure identical concepts at an identical level of difficulty~\cite{colostateReliability}. This is often measured as the correlation coefficient, measuring the strength of the correlation between the dependent variable and the independent variabel. 
	
	\subsection{Stability reliability}
	Stability reliability is a measure of the instruments stability. To test an answer to this, one would usually repeat a given test to see if it gives the same results.
	
	\subsection{Internal consistency}
	Internal consistency is a measure of how well an instructument measures the same underlying concepts. A common way of measuring this is to use Cronbach's alpha.
	
	\subsection{Interrater reliability}
	Interrater reliability is the extent to which raters agree, and is used as a measure of the rating system. 
	
\section{Validity}
	Validity assesses the degree of which an experiment and design measures the concept that the researchers intended to measure.
	When talking about reliability, we usually talk about \textit{internal validity} and \textit{external validity}, each with its own subcategories like \textit{face validity}, \textit{construct validity} and \textit{content validity}.
	Internal validity refers to how the study was design, organized and conducted. While external validity looks at how a study can be generalized and transferable.
	
	\subsection{Internal validity}
	Common threats to internal validity include \textit{testing effects}, \textit{statistical regression}, \textit{selection bias}, \textit{experimental mortality}, \textit{diffusion between groups} and so on. 
	
	\subsection{External validity}
	External validity suffers from some of the same threats as internal validity (selection bias etc.), but selection bias internally and externally may infact be two very different threats when conducting a study.
	Some of the more unique threats to external validity is the \textit{"`real-world"' versus "`experimental world"'} and \textit{"`faulty construts"'} threats, where the first refers to when participants are aware that they are part of a study and may therefore alter their behaviour because of this. The latter can be a bit more subtle, but refers to how well construct have been narrowed down from concepts, and how these constructs are measured.

Our test might suck! We should address this here

Ideally we might have wanted to split the groups randomly. 
However the teacher said that as she knew the skill level of everyone and to avoid conflicts we should let her assign the groups. 
We saw no major problem with this, and as we have no experience with teaching and are not even familiar with the kids we thought it best to grant her wish.

The tests reliability and internal consistency should be determined by 'KR-20' coefficient maybe. 

Possible problems
The teacher teaches the other group in the same classroom as we are doing the tests because of legal issues. 
Maybe they overhear things etc. 

Maybe we should have picked random student groups, however we saw no reason to do this. 
Particularly because everyone knows each other and everyone matched up good. 

We left out the top scoring students from the study because they have nothing to learn and the teacher was concerned that if they were included that they would help the other students a lot and we want to test the robot mainly, not how they help. 
