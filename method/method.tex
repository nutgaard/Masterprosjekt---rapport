The inspiration behind the study was a wish to test the robot as an educational tool in an experimental classroom setup. We wished to increase students' understanding of angles and their angle estimation skills. We also wished to increase the students' cooperation and motivation. As we mentioned in chapter~\ref{chap:constructionism}, the pedagogic theory we rely on is constructionism, which is quite different from the blackboard and task driven mathematics education, which according to \tcite{alseth2003endringer} happens in most of the classrooms in Norway, including the classrooms at THIS. %When we talked to the teacher at THIS, it seemed like this is the prevalent method there as well. 
\chapter{Choice of (testing) methods}
Most research methods can be placed in one of two categories: quantitative and qualitative methods. Qualitative research is often used to get an insight into how students perceive something, and their thoughts and feelings about the subject. It is also used to understand their behavior. Quantitative methods have a focus on finding the general correlations, and discover tendencies that give interesting insights for future research or can be supported through other methods. This is done by pointing out special factors that are the cause of a given tendency, like smoking is a factor in the tendency of lung cancer [Tjora]. If you want a broad/superficial understanding of a phenomenon you should use quantitative methods. Testing a hypothesis statistically is an example of quantitative research methods. 

\section{Test}
Tests are commonly used  in schools to find out what the students know and what they have learned. All the papers presented in the pre study applied tests to support their findings. This is due to one of the literature study inclusion criteria. We wanted statistical results and therefore we ended up with only test using papers. 

\bigskip\noindent
One advantage of using a test is that students are used to being tested in this manner. It is relatively easy and efficient to get an overview of what the students know, or don’t know and what they have learned.

\bigskip\noindent
Using tests provide many disadvantages or rather lacks of what it can accomplish alone. It is hard to test all kinds of knowledge. It becomes an absolute questions do the students know this or don’t they know it? With a little help the students could possibly have managed the task, if they for example did not understand the question. With this in mind the question about tests are: does this test really reflect what they understand and what is outside their understanding [vygotsky]. A way to work around this would be to include several items in the test that measures knowledge in the underlying same way. Interview or an oral test can also be used in this case to get better knowledge of their understanding.

\bigskip\noindent
A test should be built on a solid knowledge within the given area to make sure it is accurately reflects the specific concept the researcher is attempting to measure. In other words the test should be valid. There has been done little research in this area which makes it harder to create a valid and reliable test.

\section{Observation}
Observation is used especially within social anthropology. This type of observation is a set of methods where the researcher participates, openly or hidden, in people’s daily lives in a given period of time[Tjora]. In the same way observation can be used in a certain context. The goal of observation is often to see with your own eyes what happens to a person or group in a given situations or under certain circumstances. According to [Thagaard] observation is particularly suited to give information about people’s behaviour and how they interact with each other. 

\bigskip\noindent
Observation can be used as the only method in an experiment, but can be, and is often, used as a supplement to other methods. We distinguish between open and hidden observation and spectator, partially participating or completely participating observation. In hidden observation the researcher is present and observing, without the observee being informed of this. The advantage of this type of observation is that the participants don’t get influenced and change their behavior as a result of being observed. The downside may be ethical issues and considerations. The difference between non-participating and participating observation is well illustrated by the difference in observing a handball match as a player or spectator. 
\section{Intervju}
Interview is used in situations where you want to study meaning, attitude and experience[Tjora]. According to [Thagaard] interview is particularly well used to give information about a person’s experience, point of view and self understanding. In the interview situation the researcher often tries to create a relaxed atmosphere, so that the interview resembles a normal conversation revolving around some chosen topics. During the conversation the researcher leads the conversation by asking open or closed questions. The goal is often to get the interview object to talk as freely as possible. In this way it becomes clearer what the interview subject means, their attitudes and experiences comes forward. 

\bigskip\noindent
The advantage of interview  is the possibility to ask follow up questions and to come close in on the interview object.

\bigskip\noindent
The disadvantage is that it takes a lot of time and the subject might feel less anonymous.

\bigskip\noindent
Students might have difficulty with the way questions are written (have problems with the language, the teacher mentioned at least 1 that might struggle) and with a short explanation they might have been able to answer it in an oral test. In an oral test they could have argued why they approached the way they did and how they solved it as well. We get to know their thoughtprocess and easier understand what they have misunderstood and understood.  
\section{Survey}
Questionnaires are in general considered as a quantitative method. [Ringdal] says that a questionnaire is a systematic method to collect data from a sample of people. This is done to give a statistical description of the population the sample is collected from. The questionnaire is standardized, which means that everyone get the same questions asked in the same way. Questionnaire is often used to find overall connections and tendencies by pointing at special factors that are the cause of a given tendency[Tjora]. 

\bigskip\noindent
The advantage of using a survey is that you can ask many students at the same time, to save time. You also obtain data that you can use for statistical testing and draw conclusions about correlations on. 

\bigskip\noindent
One disadvantage is that there is no possibility for follow up questions. It might be possible to find connections and correlations between questions etc, but it is hard to understand why without any other methods as supplements.
\section{Experiment}
Experiment is the classical scientific design in science, but used also in medicine and in social science with different names for it, in our case experiment, in other cases trial or intervention etc [Ringdal]. In experiment the order is controlled between X and Y by the researcher deciding when the experiment group should be exposed to X. Control for other factors are secured through randomization. This gives a good foundation for explaining cause and effect. In social science experiments cannot be performed in such a controlled way. It is impossible to control all the factors that can influence a person or group, because they keep living their lives. But still some sort of experiment, trial or at least an intervention can give valuable knowledge which is difficult to attain through other methods. In this way we can test a teaching plan which it would be difficult to say something about how it works without some sort of trial. 

\bigskip\noindent
The advantage of an experiment is the possibility  to test out something new on real people and see how it works

\bigskip\noindent
The bad thing is the disturbance of trial people/class. We fuck with their daily activities. It is a abnormal setting. If the teacher just tries it it might be more real? Not only is it something new but some random people are coming and think they know everything. 
\section{Choice}
The choice of methods are selected based on the research questions and different practical considerations. We chose a mix of quantitative and qualitative methods. The choice was influenced by our research questions, the prestudy, the teacher and time limitations. The quantitative methods was meant to aid in our attempt to answer the research question: if there is a learning gain in angles in students. The qualitative should aid in pointing out strengths and weaknesses in the robot vs simulator as well as supplementing the quantitative data findings through observation of how the students solved the tasks. 

\bigskip\noindent
For our first research question which talks about learning gains in students angle understanding, we wanted to use interview, test and observation. This decision was based on the other research from the prestudy, but also our own beliefs that an individual construct their own knowledge within their social environment by building mental models which they can use to think with. We would have liked to use interviews because of its ability to ask followup questions and get a deeper understanding of how the robotics changed the way students think. However there was not enough time to do an interview because it was was the end of the school year. It was the end of the school year because it was hard for us to get in contact with the school. They were very busy at the moment. The students schedules were full and would therefore miss something else if they had to be taken out of the class to be interviewed. A test can be administered by the teacher and everyone can do it at the same time. An interview would have taken longer. We tried to address this lack of interview by including questions where the students have to elaborate on how they got the answers on the pre-post test. We also wanted to observe the class, focusing on how students solved their tasks to see if any progress were made. It is hard to observe without intervening and still understand what they are thinking, but we can watch how they use their protractors to get an idea of what they are measuring or if they are thinking along the right lines so to speak etc. We chose testing because this is how they are used to getting tested, because teachers don’t have enough time to do an oral test either. It is almost only oral exams that are oral testing at all. 

\bigskip\noindent
We are using tests with well defined scores to check the learning gain. Then we can run statistical tests on it. We want to test the learning gains of robotics and the simulator. An advantage of tests considering the time issue is that everyone can take a test at the same time. It can alsobe administered by the teacher whenever there is about 30 minutes of spare time in the class schedule.

\bigskip\noindent
For the second research question which looks at differences between robots and simulators,  we wanted to use observation, interview and survey. We wanted to observe to supplement the quantitative results, but also to try to assess the cooperation and motivation, even though this is hard to observe. We also wanted to watch how the students work, how do they use the protractor, do they use it in the “right way” or what? We want to test factors that might contribute to the robot activity being better than the simulator activity based on factors pointed out in the prestudy. We will test this through the survey and in-class observations and focus on the important factors of mental models, fun, motivation and cooperation. We also gather qualitative data about the tasks by asking students to answer how they thought when attempting different tasks in the test. With these results, we will try to amplify the quantitative results and provide another basis for us to draw our conclusions. 



\chapter{Participants}
To analyze our research questions, we tested the robots with a group of fifth-grade students from the international school in Trondheim(THIS). A group of 15 students (4 boys and 11 girls) was selected from a group of 20 to work with the Robotics activity. 3 of the 20 students was on vacation during the experiment period. 2 of the 20 did a full score on the pretest and their approach to the problems, as described in the depth questions, were perfect. We therefore concluded that our activities could not teach them anything new. The teacher suggested that we don't include these 2 students. She meant that if they were in the group their ideas and help would dominate the learning gains of the students. Since we want to test the robot activity's ability to teach students the material, these students might interfere with the results. 
%It turned out that if we included them, there would not be enough tablets for every group anyways, so these students were dropped from the study as well, leaving a total of 15 students. 
All the students were in the same class and either 10 or 11 years old. They had been taught about angles one month prior to our experiment, using a traditional blackboard based teaching scheme. Thus every student had prior knowledge of angles and angle estimation. The next thing they were going to learn was about reflex angles and area of shapes. None of the students had worked in this experimental manner before or worked with this specific kind of technology before.

\bigskip\noindent
We originally wanted to include only students who knew nothing about angles. This class was chosen because it was nearing the end of the school year for THIS and this was the only available class. It would have been a perfect choice if the experiment had been performed one month earlier, when they were first introduced to the concepts of angles and degrees. Another alternative would be to do this experiment after the summer vacation, so the knowledge they gained would have had time to sink in and not get different ideas confused or altered by our experiment. 
\chapter{Experimental design}
The selected students were divided into two groups. The robot or experimental group (7 students: 3 boys and 4 girls, where one boy could not attend the first session) was appointed to work with the physical robot activity. Considering that simulators have shown to be an efficient learning tool (graph potting activity We should also write much more about the choice of simulator for the control group) the control group (8 students: 1 boy and 7 girls, where one girl could not attend the first session) worked with an on-screen simulation of the robot seen from a top down perspective on the tablet. Other than this, the activity remained the same. Each group either had a tablet or android phone. The tasks were the same in both groups. We altered which group started first because as the teacher pointed out, you always do best the first time, so we wanted to equalize this effect. The materials required for this experiment were the robots (one robot per group) for the robot group, the tablet or android phone (one tablet per group) and protractors. Experimental and control groups had no interaction during the activity sessions, however some of the other groups came over and watched the other group sometimes. We were sharing the classroom with the teacher because of legal issues. The set of daily exercises was common for both groups. The students were free to choose their team mates once the activity started. They then stayed in these groups throughout both sessions. The teacher decided who was gonna be in what group by trying to divide the skill level equally between groups, she also wanted to avoid any conflict that might arise between students. Then we randomly assigned one group to the simulator activity and one to the robot activity. 

Groups of two members were preferred. This is consistent with other research of diminishing cooperation effect of groups more than 3 students but 2 specifically was chosen on the basis of what the teachers thought best through their experience with the students. However to avoid the case of 1 student working alone some groups had to be 3 students. There was also the case of not enough tablets to work with anyways.

We chose to use a quantitative analysis of the angles and turn measurement skills because of several factors. First off we could not do interviews as this would take too much time from the teacher but also, A qualitative data collection method might have been prefered in order to dig deeper into the students brains, but there was not enough time left in the school year for interviewing students. The advantage of quantitative is that we can do statistic analysis on it, even though it is a small sample size. We also included some qualitative questions in the test, and therefore we have a little bit of both. There were barely enough time to do this project. 

\section{Procedure}
The study was divided into two sessions, one held on a Friday and one on the following Monday. The first session lasted forty minutes and the second session lasted fortyfive minutes. 
One of the groups had to use a phone instead of a tablet due to unavailability of one tablet. 
The only difference of our application when used on a phone compared to a tablet is a smaller screen, and it did not seem to make a difference, at least not on the robot group, as they don't use the simulator, which is the part requiring a larger screen. Each session consisted of a short review of what they were supposed to learn, followed by a set of 10 difficulty increasing robot exercises. In the first session there was also a short introduction of about four minutes explaining how the the software worked. During the exercise phase the students worked on their own, while we acted as tutors, solving doubts regarding software and the activity itself. If groups were stuck on a single task without making progress we provided subtle hints to help them along. 
%We also gave hints about the tasks, yet only when required by the students themselves or if they were stuck on a problem for far too long. 
We never gave them the answers or any code, but explained that visualizing the program block by block and think about the effects on the robot. If it was clear that they measured the wrong angle we helped them get on the right track by having them explain how the robot moved. 
%We never gave them the answers or any code, we merely said it might help if they go through the program block by block and think about what will happen to the robot, or asked them to show us which angle they were measuring and why they were measuring that angle. 
Other than this the students received no external help or instruction. The exercises of the first session were relatively easy, since their main purpose was for the students to familiarize with the technology. The exercises was designed so that everyone should be able to do something, but no one should be able to do everything. Only open polygons, or paths, where you don't end up where you started, were used in the first lecture exercises. This was done so that students would not get caught up in the precision of the robot or the length of movement. In the second session every exercise was a closed shape, and the students had be careful of how far the robot moved in forward or backward block.  

However when teaching one always does better the second time. We had anticipated this. Therefore we the control group did the activities first in the second lecture.

\paragraph{First lecture}
The experimental group started. The robotic activity was the first learning activity that the students in the experimental group did on that day. 
%This may have a positive effect on their learning. 
Meanwhile the control group did other activities, not related to our work, with the teacher.

\bigskip\noindent
The content in the first ten minutes was exactly the same. First an introduction of how the application works, then an introduction to supplementary angles and reflex angles. 

\bigskip\noindent
During the introduction we explained that they had to cooperate and let everyone try at some point. They should also try to understand why the program did not work before trying a new solution to avoid an trial and error approach.

\bigskip\noindent
Next followed a 32 minutes session where the students experimented with the robot / simulator and tried to solve some sample tasks. The tasks are included as appendix~\ref{appendix:experiment1}.
The tasks met our expectations, as everyone managed at least four tasks and the best group finished the last task right before the session ended. When solving these tasks, the students were allowed to use protractors and pens. 

\paragraph{Second lecture}
To keep things consistent with the first lecture we used one phone as substitute for the missing tablet.

\bigskip\noindent
The session was the first learning activity that the students in the simulator group did on that day. 
%This may have a positive effect on their learning. 
The roboitcs group had physical education before we swapped groups.

\bigskip\noindent
The second lecture started with a small recap on what we did last time. Then followed a short talk introducing regular polygons. We mentioned that, in order to complete a shape and make the robot end up where it started, it needs to turn all the way around during one program run. The students understood that this meant turning $360\,^{\circ}$. The recap focused on supplementary and reflex angles. Then followed fortyfive minutes of robot / simulator task solving activity. The tasks are included as appendix~\ref{appendix:experiment2}.
As in the first lecture the tasks we created for the second lecture also met our expectations, as everyone managed at least four tasks and the best group finished the last task right before the session ended. 
\section{Data collection}
In order to assess the learning gains accomplished by the students in angle and turn measurement estimation, a pretest-posttest design was used. The pretest was administered one week before the first session, and the posttest was administered an hour after the second session. The tests were individual and the students were not allowed to cooperate. Both tests were administered by the teacher. The other advantages and disadvantages of robots identified in section \ref{sec:whatAdvantage} was measured based on qualitative observations and quantitative results obtained by means of a post activity questionnaire.

\subsection*{Test}
Since we could not find a standardized test that measure angle skills with well defined reliability and internal consistency we had to make our own. The test is also a part of our contribution. Creating a test that uncover what we are looking for is hard, especially when we are looking at a new area of investigation. Pedagogical insight is needed to plan a good pretest and posttest, and we cooperated with the teacher of the experiment class to create our test.

\bigskip\noindent
We got inspiration from \citeauthor{clements2001logo}\cite{clements2001logo}. They tested similar concepts, and the programming language they used was LOGO. In this research they have created several tests to measure geometry learning gains. These tests have been used in several projects before us~\cite{clements1990effects,clements1993research,clements1996development,clements2001logo}. 
A subset of these tests specifically targeting angles and turn measurements were selected.
%We looked at a subset of these that specifically target angles and turn measurement. 
We then expanded and improved these questions in a session with the teacher, with regards to what the students already knew and what they are supposed to learn next. 
%We also thought back to when we were elementary school students ourselves for help. 
The finished test consist of twenty questions, divided into six categories (table~\ref{table:testCategoriesDef}). The score range from 0 to 24 and was designed to be completed in thirty minutes. We used the same tasks for both the pre and post test so we could look at differences. The test does not include questions specific to LOGO or the robots, but are general questions to measure angle understanding. 
Assessments of the tests were conducted after the experiment ended. Each correct answer were given one point, whereas incorrect answers gave zero points. No partial scores were given. Questions added to investigate the though process of the participants were not graded (e.g. question 1, "`how did you solve this?"' and question 11). 

\smalltable{Categories}{table:testCategoriesDef}{
	\begin{tabular}{lll}
		\textbf{Category \#} & \textbf{Description}\\\hline
		1 & General angle understanding\\
		2 & Normal(inner) angles\\
		3 & Reflex(outer) angles\\
		4 & Complementary and angles in shapes\\
		5 & Orientation and estimation\\
		6 & Calculation with angles\\\hline
	\end{tabular}
}

\bigskip\noindent
Below the test is described step by step and we discuss the decision behind each task and how they were scored. The test can be found in appendix \ref{appendix:pretest}.

\bigskip\noindent
\textbf{Question 1} was an open ended question meant to assess students understanding of what an angle is and their ability to explain it. There were given no points for this question.

\bigskip\noindent
\textbf{Question 2 and 3} asked the students to draw an angle and then to draw a bigger angle. 
One point was given for each correct drawing, with a maximum score of 2 points.
They were then asked to describe why the second angle is bigger. 
There were given no points for this answer, it was meant as a qualitative supplement to get a deeper understanding of their though process.

\bigskip\noindent
\textbf{Question 4} was aimed at assessing students ability to differentiate between angles and identify which is bigger in each pair. 
This question was constructed of 5 tasks, 3 normal, or inner, angles and 2 angles which has highlighted the outer angle. Where each induvidual task were worth one point. These 2 outer angles were added based on the teachers advice, that this is what the students are supposed to learn next. In the first pair, one of the angles has longer sides than the other and seeks to confuse students who don't understand that angle size is measured according to how wide it is, not how much space it occupies. In the second pair, one angle is turned upside down to appear smaller than it actually is and will test students orientation ability. In the third pair the angles are very small and will test the students general estimation skills, in case they cannot answer correctly on the two previous pairs. The 4th and 5th pair are outer angles. The 4th pair has an orientation change to confuse the students while the 5th pair is straight forward to test if they understand outer angles. One point was given for each correct choice, with a maximum score of 5 points.

\bigskip\noindent
\textbf{Question 5} was aimed at angle calculation skills. There are a certain set of geometric properties that the students are supposed to be familiar with already, which we predict will be reinforced by our experiment. One point was given for each correct angle, with a maximum score of 3 points.In addition they are asked to explain how they got their answer, this is not scored.

\bigskip\noindent
\textbf{Question 6} tests the students estimation skills. 2 of the questions have a multiple choice format, while in the 2 last the students need to write an estimate on their own. One point was given for each correct choice or estimate, with a maximum score of 4 points. In the tasks where they have to write their own estimate we have an error margin of $\pm 15^{\circ}$. The first task is a straight angle which they should be very familiar with. The next has its orientation changed, so does the 3rd. In the 4th angle estimation task the angle is very wide compared to the previous tasks. 

\bigskip\noindent
\textbf{Question 7} is the same as question 6, but with outer angles. This will test their ability to measure the outer angle and understanding of what the circle drawn on the angles mean. One point was given for each correct choice or estimate, with a maximum score of 4 points. There is an error margin of $\pm 15^{\circ}$. The angles are the same as in question 6, but have had their orientation changed and their order changed. 

\bigskip\noindent
\textbf{Question 8} tests the students ability to orient. They must imagine themselves in the position of the arrow and with the direction of the arrow and decide which direction they want to turn. Then they need to estimate the amount of degrees to turn in order to face the black dot. One point is given if both the direction and amount of degrees is correct, with a maximum score of 4 points. There is an error margin of $\pm 15^{\circ}$. 

\bigskip\noindent
\textbf{Question 9 and 10} are also turn estimation tasks. The students must keep track of how much the robot has turned at every step and in which direction it is currently facing. In question 9 one point is given if the answer is 12. In question 10 one point is given if the amount of degrees are 50. The maximum score is 2 points.

\subsection*{Observation}
During the experiment we observed the students. We used hidden observation and acted as spectators. We had a focus on teamwork, motivation and problem solving approach during the observation. We got inspiration from \tcite{mitnik2009collaborative} and focused on the 4 factors that foster collaboration. Individual responsibility, mutual support, positive interdependence and social face-to-face interactions. To observe problem solving approach we had a focus on how the students used their protractors and if they discussed the problem before starting to program it. As both the simulator group and the robotics group consisted of more teams than there were observers, we confined their working area in such a way that both observers always could see and head every team.

\subsection*{Questionnaire}
The questionnaire can be found at the end of the posttest (Appendix \ref{appendix:posttest}). It was created to test the difference between robots and simulators regarding the factors identified on section \ref{sec:whatAdvantage} Each of the questionnaire's questions had five possible answers; \texttt{--}, \texttt{-}, \texttt{0}, \texttt{+} and \texttt{++}; \texttt{--} meant ``very little'' while \texttt{++} meant ``very much''.
\chapter{Reliability and validity}
Reliability and Validity are important topics within research. These topics provides our discussion for whether we are studying what we think we are studying and whether the measures are consistent. 
Reliability and validity are important in the quality assessment of scientific investigations. Reliability means that the data is as reliable as possible. Reliability is high if independent measurements give the same results. It aims to have as few as possible error sources in the test. 

Our test might suck! We should address this here

Ideally we might have wanted to split the groups randomly. However the teacher said that as she knew the skill level of everyone and to avoid conflicts we should let her assign the groups. We saw no major problem with this, and as we have no experience with teaching and are not even familiar with the kids we thought it best to grant her wish.

The tests reliability and internal consistency should be determined by 'KR-20' coefficient maybe. 

Possible problems
The teacher teaches the other group in the same classroom as we are doing the tests because of legal issues. Maybe they overhear things etc. 

Maybe we should have picked random student groups, however we saw no reason to do this. Particularly because everyone knows each other and everyone matched up good. 

We left out the top scoring students from the study because they have nothing to learn and the teacher was concerned that if they were included that they would help the other students a lot and we want to test the robot mainly, not how they help. 
