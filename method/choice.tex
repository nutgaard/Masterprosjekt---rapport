\chapter{Choice of methods}
Most research methods can be placed in one of two categories: quantitative and qualitative methods. Qualitative research is often used to get an insight into how students perceive something, and their thoughts and feelings about the subject. It is also used to understand their behavior. Quantitative methods have a focus on finding the general correlations, and discover tendencies that give interesting insights for future research or can be supported through other methods. This is done by pointing out special factors that are the cause of a given tendency, like smoking is a factor in the tendency of lung cancer [Tjora]. If you want a broad/superficial understanding of a phenomenon you should use quantitative methods. Testing a hypothesis statistically is an example of quantitative research methods. 

\section{Test}
Tests are commonly used  in schools to find out what the students know and what they have learned. All the papers presented in the pre study applied tests to support their findings. This is due to one of the literature study inclusion criteria. We wanted statistical results and therefore we ended up with only test using papers. 

\bigskip\noindent
One advantage of using a test is that students are used to being tested in this manner. It is relatively easy and efficient to get an overview of what the students know, or don’t know and what they have learned.

\bigskip\noindent
Using tests provide many disadvantages or rather lacks of what it can accomplish alone. It is hard to test all kinds of knowledge. It becomes an absolute questions do the students know this or don’t they know it? With a little help the students could possibly have managed the task, if they for example did not understand the question. With this in mind the question about tests are: does this test really reflect what they understand and what is outside their understanding [vygotsky]. A way to work around this would be to include several items in the test that measures knowledge in the underlying same way. Interview or an oral test can also be used in this case to get better knowledge of their understanding.

\bigskip\noindent
A test should be built on a solid knowledge within the given area to make sure it is accurately reflects the specific concept the researcher is attempting to measure. In other words the test should be valid. There has been done little research in this area which makes it harder to create a valid and reliable test.

\section{Observation}
Observation is used especially within social anthropology. This type of observation is a set of methods where the researcher participates, openly or hidden, in people’s daily lives in a given period of time[Tjora]. In the same way observation can be used in a certain context. The goal of observation is often to see with your own eyes what happens to a person or group in a given situations or under certain circumstances. According to [Thagaard] observation is particularly suited to give information about people’s behaviour and how they interact with each other. 

\bigskip\noindent
Observation can be used as the only method in an experiment, but can be, and is often, used as a supplement to other methods. We distinguish between open and hidden observation and spectator, partially participating or completely participating observation. In hidden observation the researcher is present and observing, without the observee being informed of this. The advantage of this type of observation is that the participants don’t get influenced and change their behavior as a result of being observed. The downside may be ethical issues and considerations. The difference between non-participating and participating observation is well illustrated by the difference in observing a handball match as a player or spectator. 
\section{Intervju}
Interview is used in situations where you want to study meaning, attitude and experience[Tjora]. According to [Thagaard] interview is particularly well used to give information about a person’s experience, point of view and self understanding. In the interview situation the researcher often tries to create a relaxed atmosphere, so that the interview resembles a normal conversation revolving around some chosen topics. During the conversation the researcher leads the conversation by asking open or closed questions. The goal is often to get the interview object to talk as freely as possible. In this way it becomes clearer what the interview subject means, their attitudes and experiences comes forward. 

\bigskip\noindent
The advantage of interview  is the possibility to ask follow up questions and to come close in on the interview object.

\bigskip\noindent
The disadvantage is that it takes a lot of time and the subject might feel less anonymous.

\bigskip\noindent
Students might have difficulty with the way questions are written (have problems with the language, the teacher mentioned at least 1 that might struggle) and with a short explanation they might have been able to answer it in an oral test. In an oral test they could have argued why they approached the way they did and how they solved it as well. We get to know their thoughtprocess and easier understand what they have misunderstood and understood.  
\section{Survey}
Questionnaires are in general considered as a quantitative method. [Ringdal] says that a questionnaire is a systematic method to collect data from a sample of people. This is done to give a statistical description of the population the sample is collected from. The questionnaire is standardized, which means that everyone get the same questions asked in the same way. Questionnaire is often used to find overall connections and tendencies by pointing at special factors that are the cause of a given tendency[Tjora]. 

\bigskip\noindent
The advantage of using a survey is that you can ask many students at the same time, to save time. You also obtain data that you can use for statistical testing and draw conclusions about correlations on. 

\bigskip\noindent
One disadvantage is that there is no possibility for follow up questions. It might be possible to find connections and correlations between questions etc, but it is hard to understand why without any other methods as supplements.
\section{Experiment}
Experiment is the classical scientific design in science, but used also in medicine and in social science with different names for it, in our case experiment, in other cases trial or intervention etc [Ringdal]. In experiment the order is controlled between X and Y by the researcher deciding when the experiment group should be exposed to X. Control for other factors are secured through randomization. This gives a good foundation for explaining cause and effect. In social science experiments cannot be performed in such a controlled way. It is impossible to control all the factors that can influence a person or group, because they keep living their lives. But still some sort of experiment, trial or at least an intervention can give valuable knowledge which is difficult to attain through other methods. In this way we can test a teaching plan which it would be difficult to say something about how it works without some sort of trial. 

\bigskip\noindent
The advantage of an experiment is the possibility  to test out something new on real people and see how it works

\bigskip\noindent
The bad thing is the disturbance of trial people/class. We fuck with their daily activities. It is a abnormal setting. If the teacher just tries it it might be more real? Not only is it something new but some random people are coming and think they know everything. 
\section{Choice}
The choice of methods are selected based on the research questions and different practical considerations. We chose a mix of quantitative and qualitative methods. The choice was influenced by our research questions, the prestudy, the teacher and time limitations. The quantitative methods was meant to aid in our attempt to answer the research question: if there is a learning gain in angles in students. The qualitative should aid in pointing out strengths and weaknesses in the robot vs simulator as well as supplementing the quantitative data findings through observation of how the students solved the tasks. 

\bigskip\noindent
For our first research question which talks about learning gains in students angle understanding, we wanted to use interview, test and observation. This decision was based on the other research from the prestudy, but also our own beliefs that an individual construct their own knowledge within their social environment by building mental models which they can use to think with. We would have liked to use interviews because of its ability to ask followup questions and get a deeper understanding of how the robotics changed the way students think. However there was not enough time to do an interview because it was was the end of the school year. It was the end of the school year because it was hard for us to get in contact with the school. They were very busy at the moment. The students schedules were full and would therefore miss something else if they had to be taken out of the class to be interviewed. A test can be administered by the teacher and everyone can do it at the same time. An interview would have taken longer. We tried to address this lack of interview by including questions where the students have to elaborate on how they got the answers on the pre-post test. We also wanted to observe the class, focusing on how students solved their tasks to see if any progress were made. It is hard to observe without intervening and still understand what they are thinking, but we can watch how they use their protractors to get an idea of what they are measuring or if they are thinking along the right lines so to speak etc. We chose testing because this is how they are used to getting tested, because teachers don’t have enough time to do an oral test either. It is almost only oral exams that are oral testing at all. 

\bigskip\noindent
We are using tests with well defined scores to check the learning gain. Then we can run statistical tests on it. We want to test the learning gains of robotics and the simulator. An advantage of tests considering the time issue is that everyone can take a test at the same time. It can alsobe administered by the teacher whenever there is about 30 minutes of spare time in the class schedule.

\bigskip\noindent
For the second research question which looks at differences between robots and simulators,  we wanted to use observation, interview and survey. We wanted to observe to supplement the quantitative results, but also to try to assess the cooperation and motivation, even though this is hard to observe. We also wanted to watch how the students work, how do they use the protractor, do they use it in the “right way” or what? We want to test factors that might contribute to the robot activity being better than the simulator activity based on factors pointed out in the prestudy. We will test this through the survey and in-class observations and focus on the important factors of mental models, fun, motivation and cooperation. We also gather qualitative data about the tasks by asking students to answer how they thought when attempting different tasks in the test. With these results, we will try to amplify the quantitative results and provide another basis for us to draw our conclusions. 


