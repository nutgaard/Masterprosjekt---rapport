\section{Choice of methods}
In the end we ended up with a mix of qualitative and quantitative methods. This was due to our research questions and pratical considerations like time limitations. 
%The choice of methods are selected based on the research questions and different practical considerations. We chose a mix of quantitative and qualitative methods. The choice was influenced by our research questions, the prestudy, the teacher and time limitations. 
The quantitative methods was meant to aid in our attempt to answer the research question: is there a learning difference in students understanding of angles. The qualitative methods should aid in pinpointing strengths and weaknesses in the robot compared to a simulator as well as supplementing the quantitative data findings. 

\bigskip\noindent
For our first research question ``Can using robotics in school help students understand what an angle is and aid in their angle estimation skills?'', we wanted to use interview, test and observation. This decision was based on related work discovered through the prestudy, but also our own beliefs that an individual construct their own knowledge within their social environment by building mental models which they can use to think with. We would have liked to use interviews because of its ability to ask followup questions and get a deeper understanding of how the robotics changed the way students think. 
However, due to the time limitations we were not able to use this method. 
%However there was not enough time to do an interview because the experiment was performed near the end of the school year. 
The experiment was preformed at the end of a semester as getting in contact with the school, designing the experiment, and designing the prototype system all took a long time.
%The experiment was performed so late in the semester because it was hard for us to get in contact with a school that was willing to let us run the experiment.
At the time the students schedules were full and would therefore miss something important if they had to be taken out of the class to be interviewed. A test could be administered by the teacher and all the students could do it at the same time. An interview would have taken longer, as we would have needed to interview one student at a time. We tried to compensate for this lack in data by including questions where the students have to elaborate on how they got their answers on the test. We also wanted to observe the class. It is hard to observe without intervening and still understand what the students are thinking, but through watching how they used their protractors we could identify which angle they were measuring, and that shows whether they have understood it or not. We chose testing because this is how students are used to getting tested.

\bigskip\noindent
For the second research question ``What are the strength and weaknesses of using a robot compared to a simulated environment?'',  we wanted to use observation, interview and questionnaire. We wanted to use observation to supplement the quantitative results, but also to assess the cooperation and motivation, even though this is hard to observe. We want to test factors found through the prestudy, that might contribute to the robot activity being better than the simulator activity. We will test this through a questionnaire and in-class observations and focus on the important factors of mental models, motivation, cooperation and problem solving approach. We also gather qualitative data about the tasks by asking students to answer how they thought when attempting different tasks in the test. With these results, we will try to amplify the quantitative results and provide another basis for us to draw our conclusions. 


