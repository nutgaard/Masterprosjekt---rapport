\chapter{Choice of (testing) methods}
In these 3 first subchapters we should write about the different methods available to us. We discuss the choice between these first in the choice subchapter. Even though we do not use intervju etc we should write about it so we can discuss and say that we wanted to use it, but there would simply not be enough time and so forth.

Most research methods can be placed in one of two categories, quantitative and qualitative methods. Quantitative methods are used when one seeks a broader understanding of a phenomenon. Testing a hypothesis statistically or testing a causal connection are examples of quantitative research methods. 

\section{Test}
A reason to not use a questionaire is that it has to build on a solid knowledge within the given area. There has not been done a lot of research in the given area. 

\section{Observation}
\section{Intervju}
\section{Survey}
\section{Choice}
Here we could write about what we chose in the end and why we chose these. We have chosen to use a mix of quantitative and qualitative research methods.

The choice of methods are selected based on the research questions and different practical considerations. We worked with THIS and it is nearing the end of the school year. Even though we would have liked to intervju the children there would simply not be enough time to do this. Our choices then are a prepost test and observation. Based on the other research and such from our point of view an individual construct their own knowledge within their social environment.

The quantitative was meant to aid in our attempt to answer if there is a learning gain while the qualitative should aid in pointing out strengths and weaknesses in the robot vs simulator. The qualitative includes written questions on the pre-post test on how the students got their answers as well as observation in the classroom. 

We want to test factors that might contribute to robot being better than the simulator based on factors pointed out in other research. We will test this through the survey and in-class observations and focus on the important factors of mental models, fun, motivation and cooperation. 

We also gather qualitative data about the tasks by asking students to answer how they thought when attempting different tasks. These results will try to amplify the quantitative results and provide another basis for us to draw our conclusions. 

