\chapter{Choice of methods}
Most research methods can be placed in one of two categories: quantitative or qualitative methods. Qualitative research is often used to get an insight into how students perceive something, and their thoughts and feelings about the subject. It is also used to understand their behavior. Quantitative methods have a focus on finding the general correlations, and discover tendencies that give interesting insights for future research or can be supported through other methods. This is done by pointing out special factors that are the cause of a given tendency \cite{tjora2012kvalitative}. If you want a broad understanding of a phenomenon you should use quantitative methods. Testing a hypothesis statistically is an example of quantitative research methods. 

\section{Test}
Tests are commonly used  in schools to find out what the students know and what they have learned. All the papers presented in the pre study applied a pretest and a posttest to support their findings. This is due to one of the literature study inclusion criteria. We wanted statistically significant results to support our decisions. 

\bigskip\noindent
One advantage of using a test is that students are used to being tested in this manner. It is relatively easy and efficient to get an overview of what the students know, or don't know and what they have learned.

\bigskip\noindent
Using tests alone has limitations to what can be measured. It is hard to test for deeper kinds of knowledge, as test questions often have only one right answer. It only test whether students know the answer or not, even though they might have managed the task with a little help. For example if a student did not understand the question due to language limitations. With this in mind the question about tests are: does this test really reflect what they understand and what is outside their understanding \cite{andersen2009evaluering}. A way to work around this would be to include several items in the test that measures the same underlying knowledge. Interview or an oral test can also be used in this case to get better knowledge of their understanding.

\bigskip\noindent
A test should be built on a solid knowledge within the given area to make sure it is accurately reflects the specific concept the researcher is attempting to measure. In other words the test should be valid. There has been done little research in this area which makes it harder to create a valid and reliable test.

\section{Observation}
Observation is used especially within social anthropology. This type of observation is a set of methods where the researcher participates, openly or hidden, in people's daily lives in a given period of time\cite{tjora2012kvalitative}. In the same way observation can be used in a certain context. The goal of observation is often to see with your own eyes what happens to a person or group in a given situations or under certain circumstances. According to \cite{thagaard2003systematikk} observation is particularly suited to give information about people's behavior and how they interact with each other, thus it is a good option to measure cooperation. 

\bigskip\noindent
Observation can be used as the only method in an experiment, but can be, and is often, used as a supplement to other methods. We distinguish between open and hidden observation and spectator, partially participating or completely participating observation. In hidden observation the researcher is present and observing, without the observee being informed of this. The advantage of this type of observation is that the participants don't get influenced and change their behavior as a result of being observed. The downside may be ethical issues and considerations. The difference between non-participating and participating observation is well illustrated by the difference in observing a handball match as a player or spectator. 

\section{Interview}
Interview is used in situations where you want to study meaning, attitude and experience\cite{tjora2012kvalitative}. According to \cite{thagaard2003systematikk} interview is particularly well used to give information about a person's experience, point of view and self understanding. In an interview situation the researcher often tries to create a relaxed atmosphere, so that the interview resembles a normal conversation revolving around some chosen topics. During the conversation the researcher leads the conversation by asking open or closed questions. The goal is often to get the interview object to talk as freely as possible. In this way it becomes clear what the interview subject means. Their attitudes and experiences comes forward. 

\bigskip\noindent
An advantage of using interview is the possibility to ask follow up questions and to come close to the interview object.
A disadvantage is that it takes a lot of time and the subject might feel less anonymous.

\bigskip\noindent
Students might have difficulties with the way questions are written, there can for example be language barriers, and with a short explanation they might be able to answer questions in an oral test that they would have answered incorrectly in a written test. In an oral test there are also opportunities to argue why they approached the way they did and how they solved the task in greater detail. We get to understand their thought process and understand what they have actually misunderstood and understood and not just misinterpreted.  

\section{Questionnaire}
Questionnaires are in general considered a quantitative method. \cite{ringdal2001enhet} writes that a questionnaire is a systematic method to collect data from a sample of people. This is done to give a statistical description of the population the sample is collected from. The questionnaire is standardized, which means that everyone get the same questions asked in the same way. A questionnaire is often used to find overall connections and tendencies by pinpointing special factors that are the cause of a given tendency \cite{tjora2012kvalitative}. 

\bigskip\noindent
An advantage of using a questionnaire is that you can perform it on many students at the same time. You also obtain data that you can use for statistical testing and draw conclusions about correlations on. 

\bigskip\noindent
One disadvantage is that there is no possibility for follow up questions. It might be possible to find connections and correlations between questions, but it is hard to understand why without any other methods as supplements.

\section{Experiment}
Experiment is the classical scientific design in science, but it is also used in medicine and in social science under different names. In our case we call it experiment, in other cases it is called trial or intervention \cite{ringdal2001enhet}. In an experiment the researcher controls the difference between two treatments, X and Y, by deciding when the experimental group should be exposed to X. Control of other factors are secured through randomization. This gives a good foundation for explaining cause and effect. In social science, experiments cannot be performed in such a controlled way. It is impossible to control all the factors that can influence a person or group. But still some sort of experiment, trial or intervention can give valuable knowledge which is difficult to attain through other methods.

\bigskip\noindent
The main advantage of an experiment is the possibility  to test out something new on real people and see how it works.
A disadvantage is the disturbance of the trial population. We intervene with their daily activities. One possible solution is that the teacher runs the experiment. Else wise, not only is the activity new, but new people are introduces to the students.

\section{Choice}
The choice of methods are selected based on the research questions and different practical considerations. We chose a mix of quantitative and qualitative methods. The choice was influenced by our research questions, the prestudy, the teacher and time limitations. The quantitative methods was meant to aid in our attempt to answer the research question: is there a learning difference in students understanding of angles. The qualitative methods should aid in pinpointing strengths and weaknesses in the robot compared to a simulator as well as supplementing the quantitative data findings. 

\bigskip\noindent
For our first research question ``Can using robotics in school help students understand what an angle is and aid in their angle estimation skills?'', we wanted to use interview, test and observation. This decision was based on related work discovered through the prestudy, but also our own beliefs that an individual construct their own knowledge within their social environment by building mental models which they can use to think with. We would have liked to use interviews because of its ability to ask followup questions and get a deeper understanding of how the robotics changed the way students think. However there was not enough time to do an interview because the experiment was performed near the end of the school year. The experiment was performed so late in the semester because it was hard for us to get in contact with a school that was willing to let us run the experiment. The students schedules were full and would therefore miss something important if they had to be taken out of the class to be interviewed. A test can be administered by the teacher and all the students can do it at the same time. An interview would have taken longer, as we would have needed to interview one student at a time. We tried to compensate for this lack in data by including questions where the students have to elaborate on how they got their answers on the test. We also wanted to observe the class. It is hard to observe without intervening and still understand what the students are thinking, but through watching how they used their protractors we could identify which angle they were measuring, and that shows whether they have understood it or not. We chose testing because this is how students are used to getting tested.

\bigskip\noindent
For the second research question ``What are the strength and weaknesses of using a robot compared to a simulated environment?'',  we wanted to use observation, interview and questionnaire. We wanted to use observation to supplement the quantitative results, but also to assess the cooperation and motivation, even though this is hard to observe. We want to test factors found through the prestudy, that might contribute to the robot activity being better than the simulator activity. We will test this through a questionnaire and in-class observations and focus on the important factors of mental models, motivation, cooperation and problem solving approach. We also gather qualitative data about the tasks by asking students to answer how they thought when attempting different tasks in the test. With these results, we will try to amplify the quantitative results and provide another basis for us to draw our conclusions. 


