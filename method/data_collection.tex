\chapter{Data collection}
In order to assess the learning gains accomplished by the students in angle and turn measurement estimation, a pretest-posttest design was used. The pretest was administered one week before the first session, and the posttest was administered an hour after the second session. The tests were individual and the students were not allowed to cooperate. Both tests were administered by the teacher. Since we could not find a standardized test that measure angle skills with well defined reliability and internal consistency we had to make our own. We got inspiration from Clements' research in LOGO. In his research he has created several tests to measure geometry learning gains. These tests have been used in several projects before us. We looked at a subset of these that specifically target angles and turn measurement. We then expanded and improved(focused for our group) these questions in a session with the teacher, with regards to what the students already knew and what they are supposed to learn next. We also thought back to when we were at school ourselves for help. The finished test consist of 24 questions, divided into 7 categories. The score range from 0 to 24 and was designed to be completed in 30 minutes. We used the same tasks for both the pre and post test so we can look at differences. The test does not include questions specific to LOGO or the robots, but are general questions to measure angle understanding. To aid the reader the questions are included in the results section when we go through each category. The factors identified in the pre-study was measured based on qualitative observations and quantitative results obtained by means of a post activity survey. Each of the survey's questions had five possible answers; --, -, 0, + and ++; "`--"' meant "`very little"' while "`++"' ment "`very much"'.

\bigskip\noindent
Regarding collaboration, the in-site observations were focused on the following four factors that have shown to foster effectiveness in collaborative learning settings (people from a graph teaching).

\bigskip\noindent
Observation were also focused on the other sustainable learning factors. We also focused on how the students seemed to measure the angles with the protractors and how they solved the tasks in general. In addition we looked for potential problems and ideas of what might be better explained etc. We did this because we are doing an intro master and this is supposed to be a platform to build on. Because of this future work is necessary to focus on. 

\bigskip\noindent
Finally, in order to determine whether the learning outcomes of these activities were dependant of the students' previous knowledge, the difference between the posttest and pretest results were analyzed in relation to the pretest scores.

\section{Test}
As mentioned we had to create our own test to assess learning gains in the topic angle and turn estimation~(Appendix~\ref{appendix:pretest}). Here we will go through step by step some of the choices taken when creating the test and what they are supposed to measure. We will also mention how these questions were scored for use in statistical analysis. 

\bigskip\noindent
Question 1 was an open ended question meant to assess students understanding of what an angle is and their ability to explain it. There were given no points for this question.

\bigskip\noindent
Question 2 and 3 asked the students to draw an angle and then to draw a bigger angle. Each worth a single point. They were then asked to describe why the second angle is bigger. There were given no points 
for this answer, it was meant as a supplement to draw conclusions on.

\bigskip\noindent
Question 4 was aimed at assessing students ability to differentiate between angles and identify which is bigger in each pair. One point was given for each correct choice. There are 3 normal, or inner, angles and 2 angles which has highlighted the outer angle. These 2 outer angles were added because of the teachers advice, that this is what the students are supposed to learn next. In the first pair, one of the angles has longer "`legs"' and seeks to confuse students who don't understand that angle size is measured according to how wide it is, not how much space it occupies. In the second pair, one angle is turned upside down to appear smaller than it actually is and will test students who think orientation matter. In the third pair the angles are very small and will test the students general estimation skills, in case they cannot answer correctly on the two previous pairs. The 4th and 5th pair are outer angles. The 4th pair has an orientation change to confuse the students while the 5th pair is straight forward to test if they understand outer angles. 

\bigskip\noindent
Question 5 was aimed at angle calculation skills. There are a certain set of rules the students are supposed to be familiar with already, which we think will be reinforced by our experiment. Each sub task is worth 1 point each. In addition they are asked to explain how they got their answer, this is not worth any points.

\bigskip\noindent
Question 6 tests the students estimation skills. 2 of the questions have a multiple choice format, while in the 2 last the students need to write an estimate on their own. Each correct choice is worth 1 point. In the tasks where they have to write their own estimate we have an error margin of plus minus 15 degrees. The first task is a straight angle which they should be very familiar with. The next has its orientation changed, so does the 3rd. In the 4th angle estimation task the angle is very wide compared to the previous tasks. 

\bigskip\noindent
Question 7 is the same as question 6, but with outer angles. This will test their ability to measure the outer angle and understanding of what the circle drawn on the angles mean. Each correct choice is worth 1 point and there is an error margin of plus minus 15 degrees. The angles are the same as in question 6, but have had their orientation changed and their order changed. 

\bigskip\noindent
Question 8 tests the students ability to orient. They must imagine themselves in the position and with the direction of the arrow and decide which direction they want to turn. Then they need to estimate the amount of degrees to turn in order to face the black dot. One point is given if the direction and amount of degrees is correct. There is an error margin of plus minus 15 degrees. 

\bigskip\noindent
Question 9 and 10 are also turn estimation tasks. The students must keep track of how much the robot has turned at every step and in which direction it is currently facing. In question 9 one point is given if the answer is 12. In question 10 one point is given if the amount of degrees are 50. 