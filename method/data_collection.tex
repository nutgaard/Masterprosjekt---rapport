\section{Data collection}
In order to assess the learning gains accomplished by the students in angle and turn measurement estimation, a pretest-posttest design was used. The pretest was administered one week before the first session, and the posttest was administered an hour after the second session. The tests were individual and the students were not allowed to cooperate. Both tests were administered by the teacher. The other advantages and disadvantages of robots identified in section \ref{sec:whatAdvantage} was measured based on qualitative observations and quantitative results obtained by means of a post activity questionnaire.

\subsection*{Test}
Since we could not find a standardized test that measure angle skills with well defined reliability and internal consistency we had to make our own. The test is also a part of our contribution. Creating a test that uncover what we are looking for is hard, especially when we are looking at a new area of investigation. Pedagogical insight is needed to plan a good pretest and posttest, and we cooperated with the teacher of the experiment class to create our test.

\bigskip\noindent
We got inspiration from \citeauthor{clements2001logo}\cite{clements2001logo}. They tested similar concepts, and the programming language they used was LOGO. In this research they have created several tests to measure geometry learning gains. These tests have been used in several projects before us~\cite{clements1990effects,clements1993research,clements1996development,clements2001logo}. 
A subset of these tests specifically targeting angles and turn measurements were selected.
%We looked at a subset of these that specifically target angles and turn measurement. 
We then expanded and improved these questions in a session with the teacher, with regards to what the students already knew and what they are supposed to learn next. 
%We also thought back to when we were elementary school students ourselves for help. 
The finished test consist of twenty questions, divided into six categories (table~\ref{table:testCategoriesDef}). The score range from 0 to 24 and was designed to be completed in thirty minutes. We used the same tasks for both the pre and post test so we could look at differences. The test does not include questions specific to LOGO or the robots, but are general questions to measure angle understanding. 
Assessments of the tests were conducted after the experiment ended. Each correct answer were given one point, whereas incorrect answers gave zero points. No partial scores were given. Questions added to investigate the though process of the participants were not graded (e.g. question 1, "`how did you solve this?"' and question 11). 

\smalltable{Categories}{table:testCategoriesDef}{
	\begin{tabular}{lll}
		\textbf{Category \#} & \textbf{Description}\\\hline
		1 & General angle understanding\\
		2 & Normal(inner) angles\\
		3 & Reflex(outer) angles\\
		4 & Complementary and angles in shapes\\
		5 & Orientation and estimation\\
		6 & Calculation with angles\\\hline
	\end{tabular}
}

\bigskip\noindent
Below the test is described step by step and we discuss the decision behind each task and how they were scored. The test can be found in appendix \ref{appendix:pretest}.

\bigskip\noindent
\textbf{Question 1} was an open ended question meant to assess students understanding of what an angle is and their ability to explain it. There were given no points for this question.

\bigskip\noindent
\textbf{Question 2 and 3} asked the students to draw an angle and then to draw a bigger angle. 
One point was given for each correct drawing, with a maximum score of 2 points.
They were then asked to describe why the second angle is bigger. 
There were given no points for this answer, it was meant as a qualitative supplement to get a deeper understanding of their though process.

\bigskip\noindent
\textbf{Question 4} was aimed at assessing students ability to differentiate between angles and identify which is bigger in each pair. 
This question was constructed of 5 tasks, 3 normal, or inner, angles and 2 angles which has highlighted the outer angle. Where each induvidual task were worth one point. These 2 outer angles were added based on the teachers advice, that this is what the students are supposed to learn next. In the first pair, one of the angles has longer sides than the other and seeks to confuse students who don't understand that angle size is measured according to how wide it is, not how much space it occupies. In the second pair, one angle is turned upside down to appear smaller than it actually is and will test students orientation ability. In the third pair the angles are very small and will test the students general estimation skills, in case they cannot answer correctly on the two previous pairs. The 4th and 5th pair are outer angles. The 4th pair has an orientation change to confuse the students while the 5th pair is straight forward to test if they understand outer angles. One point was given for each correct choice, with a maximum score of 5 points.

\bigskip\noindent
\textbf{Question 5} was aimed at angle calculation skills. There are a certain set of geometric properties that the students are supposed to be familiar with already, which we predict will be reinforced by our experiment. One point was given for each correct angle, with a maximum score of 3 points.In addition they are asked to explain how they got their answer, this is not scored.

\bigskip\noindent
\textbf{Question 6} tests the students estimation skills. 2 of the questions have a multiple choice format, while in the 2 last the students need to write an estimate on their own. One point was given for each correct choice or estimate, with a maximum score of 4 points. In the tasks where they have to write their own estimate we have an error margin of $\pm 15^{\circ}$. The first task is a straight angle which they should be very familiar with. The next has its orientation changed, so does the 3rd. In the 4th angle estimation task the angle is very wide compared to the previous tasks. 

\bigskip\noindent
\textbf{Question 7} is the same as question 6, but with outer angles. This will test their ability to measure the outer angle and understanding of what the circle drawn on the angles mean. One point was given for each correct choice or estimate, with a maximum score of 4 points. There is an error margin of $\pm 15^{\circ}$. The angles are the same as in question 6, but have had their orientation changed and their order changed. 

\bigskip\noindent
\textbf{Question 8} tests the students ability to orient. They must imagine themselves in the position of the arrow and with the direction of the arrow and decide which direction they want to turn. Then they need to estimate the amount of degrees to turn in order to face the black dot. One point is given if both the direction and amount of degrees is correct, with a maximum score of 4 points. There is an error margin of $\pm 15^{\circ}$. 

\bigskip\noindent
\textbf{Question 9 and 10} are also turn estimation tasks. The students must keep track of how much the robot has turned at every step and in which direction it is currently facing. In question 9 one point is given if the answer is 12. In question 10 one point is given if the amount of degrees are 50. The maximum score is 2 points.

\subsection*{Observation}
During the experiment we observed the students. We used hidden observation and acted as spectators. We had a focus on teamwork, motivation and problem solving approach during the observation. We got inspiration from \tcite{mitnik2009collaborative} and focused on the 4 factors that foster collaboration. Individual responsibility, mutual support, positive interdependence and social face-to-face interactions. To observe problem solving approach we had a focus on how the students used their protractors and if they discussed the problem before starting to program it. As both the simulator group and the robotics group consisted of more teams than there were observers, we confined their working area in such a way that both observers always could see and head every team.

\subsection*{Questionnaire}
The questionnaire can be found at the end of the posttest (Appendix \ref{appendix:posttest}). It was created to test the difference between robots and simulators regarding the factors identified on section \ref{sec:whatAdvantage} Each of the questionnaire's questions had five possible answers; \texttt{--}, \texttt{-}, \texttt{0}, \texttt{+} and \texttt{++}; \texttt{--} meant ``very little'' while \texttt{++} meant ``very much''.