\chapter{Data collection}
In order to assess the learning gains accomplished by the students in angle and turn measurement estimation, a pretest-posttest design was used. The pretest was administered one week before the first session, and the posttest was administered an hour after the second session. The tests were individual and the students were not allowed to cooperate. Both tests were administered by the teacher. Since we could not find a standardized test that measure angle skills with well defined reliability and internal consistency we had to make our own. We took inspiration from Clements� research in LOGO. In his research he has created several tests to measure geometry learning gains. These tests have been used in several projects before us. We looked at a subset of these that specifically target angles and turn measurement. We then expanded and improved(focused for our group) on these questions in a session with the teacher, with regards to what the students already knew and what they are supposed to learn next. We also thought back to when we were at school ourselves for help. The finished test consist of 24 questions, divided into 7 categories. The score range from 0 to 24 and was designed to be completed in 30 minutes. We used the same tasks for both the pre and post test so we can look at differences. The test does not include questions specific to LOGO or the robots, but are general questions to measure angle understanding. To aid the reader the questions are included in the results section when we go through each category. The factors identified in the pre-study was measured based on qualitative observations and quantitative results obtained by means of a post activity survey. Each of the survey�s questions had five possible answers; --, -, 0, + and ++; �--� meant �very little� while �++� ment �very much�. The questions of the post-activity survey were the following.

Regarding collaboration, the in-site observations were focused on the following four factors that have shown to foster effectiveness in collaborative learning settings (people from a graph teaching).

Observation were also focused on the other sustainable learning factors. We also looked for potential problems and ideas of what might be better explained etc. We did this because we are doing an intro master and this is supposed to be a platform to build on. Because of this future work is important. 

Write about the observation. What did we plan to look for in advance? Hvordan gikk vi fram?

Finally, in order to determine whether the learning outcomes of these activities were dependant of the students� previous knowledge, the difference between the posttest and pretest results were analyzed in relation to the pretest scores.
