\chapter{Participants}
To analyse our research questions, we tested the robots with a group of fifth-grade students from the international school in trondheim(THIS). The study was carried out over two sessions with a weekend in between. A group of 15 students (4 boys and 11 girls) was selected from a group of 20 to work with the Robotics activity. 3 of the 20 students went on vacation during the experiment period. 2 of the 20 did a full score on the pretest and explained how everything worked very well in the ``how did you solve this'' tasks, and thus did not have much to learn. The teacher suggested that we don't include these. She meant that if they were in the group their ideas and help would dominate the learning gains of the students. Since we want to test the robot activity's ability to teach students the material these good students might interfere with the results. It turned out that we could not acquire enough tablets for that many groups anyways without having many groups of three, so because of these issues, they were dropped as well. All the students were in the same class and either 10 or 11 years old. They had been taught about angles one month prior to our arrival using a traditional blackboard based teaching scheme. Thus every student had prior knowledge of angles and angle estimation. The next thing they were going to learn was about reflex angles and area of shapes. None of the students had worked in this experimental manner before or worked with this specific kind of technology before.

\bigskip\noindent
We originally wanted students who knew nothing or little, or at least it had been a while since learned about angles. But this class was chosen because since it was nearing the end of the school year there was very little time left, and this class was the only one available. It would have been a perfect choice if we had gotten there one month earlier when they were learning this in the first place or maybe after the summer when their knowledge was not so fresh and able to be confused or altered by our experiment, or ready to be built upon further. 