\section{Participants}
To analyze our research questions, we tested the robots with a group of fifth-grade students from the international school in Trondheim(THIS). A group of fifteen students (four boys and eleven girls) was selected from a class of twenty to participate in the experiment. Three of the twenty students was on vacation during the experiment period. Two of the twenty did a full score on the pretest and their approach to the problems, as described in the depth questions, were perfect. We therefore concluded that our activities could not teach them anything new. The teacher also suggested that we don't include these two students. She meant that if they were in the group their ideas and help would dominate the learning gains of the students. Since we want to test the robot activity's ability to teach students the material, these students might interfere with the results. From these fifteen only eleven were able to participate on both sessions. 
%It turned out that if we included them, there would not be enough tablets for every group anyways, so these students were dropped from the study as well, leaving a total of 15 students. 
All the students were in the same class and either ten or eleven years old. They had been taught about angles one month prior to our experiment, using a traditional blackboard based teaching scheme. Thus every student had prior knowledge of angles and angle estimation. The next thing they were going to learn was about reflex angles and area of shapes. None of the students had worked in this experimental manner before or worked with this specific kind of technology before.

\bigskip\noindent
We originally wanted to include only students who knew nothing about angles. This class was chosen because it was nearing the end of the school year for THIS and this was the only available class. It would have been a perfect choice if the experiment had been performed one month earlier, when they were first introduced to the concepts of angles and degrees. Another alternative would be to do this experiment after the summer vacation, so the knowledge they gained would have had time to sink in and not get different ideas confused or altered by our experiment. 