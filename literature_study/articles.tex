\subsection{Articles and studytype}
For the classification of different study types we borrow from Donnelly and Trochim's \textit{Research methods knowledge base}\cite{donnelly2007research}. 
The study type representation is defined as:

\bigskip\noindent
\begin{center}
	\begin{tabular}{ll}
		\textbf{R:} random assignment & \textbf{N:} nonrandom assignment\\
		\textbf{O:} measures/evidence & \textbf{X:} robotics intervention
		\label{tab:typedesc}
	\end{tabular}
\end{center}

\bigskip\noindent
If two lines is used for the classification then the first line is the timeline for the robotics group, whereas the second line will show the timeline for the control group.
In the case of \tcite{hussain2006effect} you may interpret the classification as two randomly assigned group given the same pre/post-test scheme, with the only difference being that the robotics group was exposed to robotics between the pretest and posttest. In some cases the \textbf{X} may be suffixed by a number, in these cases the different groups were exposed to different robotics experiences.


\setlength\LTleft{0px}
\setlength\LTright{0px}
\begin{longtable}{@{\extracolsep{\fill}}p{0.18\textwidth}p{0.6\textwidth}p{0.14\textwidth}}
	\hline \multicolumn{1}{l}{\textbf{Author}} & \multicolumn{1}{l}{\textbf{Article description}} & \multicolumn{1}{c}{\textbf{Study type	}} \\ \hline\hline
	\endfirsthead
	
	\hline
	\multicolumn{1}{l}{\textbf{Author}} & \multicolumn{1}{l}{\textbf{Article description}} & \multicolumn{1}{c}{\textbf{Study type	}} \\ \hline\hline
	\endhead
	
	\hline
	\caption{Articles and studytype}
	\label{tab:type}
	\endlastfoot
	\tcite{hussain2006effect} & This study aim at investigating the effect of one year regular robotics traingin on students performance. & R O X O R O O\\\hline
	\tcite{lindh2007does} & To investigate the effect of regular robotics training on pupils performance. & R O X O R O O\\\hline
	\tcite{barker2007robotics}& Paper reports on a pilot program aimed at increasing the achievement scores of young people. & N O X O N O O\\\hline
	\tcite{nugent2009use} & The goals of the program were to prepare youth for the workplace by providing them an opportunity to learn STEM concepts and foster positive attitudes towards STEM. & O X O\\\hline
	\tcite{mitnik2008autonomous} & Presents a novel application of robotics to education, where they use robotics to teach non-robotic related subjects. & N O X1 O N O X2 O\\\hline
	\tcite{nugent2008effect} & Study of 4-H program to increase STEM achievement and interest using robotics and geospatial technologies. & N O N O N O O\\\hline
	\tcite{williams2007acquisition} & Study the impact of a summer robotics camp on middle school students physics content knowledge.  & O X O\\\hline
	\tcite{mitnik2009collaborative} & This study aims at developing graph construction and graph interpretations skills in students by graphing the movements of a robot in an interactive system..  & N O X1 O N O X2 O\\\hline
	\tcite{norton2004using} & This study examines students learning ratio concepts while engaged in designing, constructing and evaluating simple machines that use cogs and pulleys.  & O X O\\\hline
	\tcite{silk2011resources}Study1 & Observes a formal classroom unit using robots to learn math concepts. Part of an introductory experience to learning robotics in a step-by-step manner.  & O X O\\\hline
	\tcite{silk2011resources}Study3 & Observes a robot competition. The students had to use math to solve the problems, where other teams could use whatever they wanted. & O X O\\\hline
	\tcite{silk2011resources}Study4 & Students have to synchronize two robots with different wheel size by explicitly using math. Another group uses math explicitly to enter a robot competition. & N O X1 O N O X2 O\\\hline
\end{longtable}
