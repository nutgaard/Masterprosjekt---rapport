\subsection*{State of the art}
The main question is, is robotics an effective teaching tool?

\bigskip\noindent
There is nothing robots do that can't be illustrated by a simulator, flash game or app. What drives researchers is the belief that robots awaken a tremendous source of energy and motivation in children. STEM topics in particular is also very abstract and robots give a concrete idea to connect these abstract ideas to. The pedagogical believes behind robotics is constructivism. 

\bigskip\noindent
In the literature search there are some areas of lacking research. Namely research on students aged 11-12 and lack of empirical research involving the use of low cost robots in education. In general there is also a lack of research with good experimental design with a larger sample. In general the lack of research can partly be blamed on the cost of robots. The cheapest robot seems to be Lego Mindstorm. This robot construction kit costs 3499 DKR. 

\bigskip\noindent
In educational robotics we differentiate between academic performance and other skills. Academic performance concerns how school curriculum can be tough by using robots, while other skills are skills outside the curriculum. These are skills that you learn because of working with the robots. Often academic performance is the main goal when introducing robots to students while other skills are merely a bonus that is not even taken into account most of the time. 

\subsubsection{Academic performance}

Topics that are taught with robotics as teaching aid are mostly within the STEM (science, technology, engineering and mathematics) category. Specifically Newton's Laws of Motion, distances, angles, kinematics, graph construction and interpretation, fractions, ratios and geospatial concepts is mentioned often. In the systematic review carried out in \cite{Benitti2012978} 80\% of the papers focus on these topics. The two remaining papers discuss basic evolution and teaching basic social skills to kids with autism and asperger syndrome. 

\bigskip\noindent
Most of the research done provides promising results. None have discovered that it worsens learning, but there are examples of it not making any difference compared to traditional methods. It is hard to pinpoint the factors that generates positive results. Researchers and schools are reluctant too implement robotics as a learning tool because of this lack of empirical evidence of it being helpful. As mentioned earlier robots are expensive and the school could rather use this money on more effective learning aids. Many think that this interest in robots is just a passing fad. 

\subsubsection{Other skills}
These skills are not measured, as the research focus is on the curriculum, but these skills may have important benefits later in school and life.

\bigskip\noindent
Technical skills that are often taught when working with robotics are problem solving, logic and scientific inquiry. 

\bigskip\noindent
Mind tools are thinking skills, problem-solving skills and teamwork skills. \cite{Benitti2012978} Suggests these as main topics for future research. Many mention that skills such as these were improved when introducing robotics in education but more research is also needed to figure out how to train the specific skills separately. 