\section*{Method}
This review has been completed as several phases, based upon the Kitchenham's \cite{kitchenham2007guidelines} and Khan's~\cite{khan2001undertaking} guide for writing systematic reviews. The initial steps of identifying a need for the review and commissioning a review have however been omitted here. The steps followed is written down below.
\begin{description}
	\item[Phase 1: ] Planning
		\begin{enumerate}
			\item Specifying the research question(s)
			\item Developing a review protocol
			\item Evaluating the review protocol
		\end{enumerate}
	\item[Phase 2: ] Conducting the review
		\begin{enumerate}
			\item Identification of research
			\item Selection of primary studies
			\item Study quality assessment
			\item Data extaction and monitoring
			\item Data synthesis
		\end{enumerate}
	\item[Phase 3: ] Reporting the review
		\begin{enumerate}
			\item Communicating the results through a report.
		\end{enumerate}
\end{description}

\subsection*{Planning the review}
\label{sec:questions}
Initially, we preformed an inital general search into educational robotics, in order to obtain some fundemental knowledge regarding this subject. In this search we stumbled upon another review written in 2011 by Fabiane Barreto Vavassori Benitti\cite{Benitti2012978}, which to a very large extend covers the same topic as initially planned by this review. There is however some minor differences between our research question, papers relevant to Benitti will to a large extend be relevant for this review as well. Benitti asked general questions like "`\textit{What topics are taught through robotics in school}s?"', "`\textit{is robotics an effective tool for teaching? What do the studies show?}"', and "`\textit{How is student learning evaluated?}"'. While this review is going to use more narrow question, limiting the research to math and how math can be taught in schools using robotics.
 
\begin{description}
	\item[Question 1: ] Which topics within math are taught through robotics in schools?
	\item[Question 2: ] How effective has these inquires been?
	\item[Question 3: ] Which, if any, secondary skills (teamwork, scientific inquiry etc) may also be improved through the utilization of robotics in education?
\end{description}

\bigskip\noindent
The review was done in February and March of 2014, with paper retrieved from all the major bibligraphic databases. These include, but not limited to, CiteSeer, ACM Digital Library, SpringerLink, ERIC, IEEE XPLORE, Wiley Inter Science, and ScienceDirect. In addition to these known the search query was run through the 	google scholars search engine to ensure that every study was found.

\bigskip\noindent
The general search query was created using groups of synonyms, concatenated by the \texttt{and}/\texttt{or} operators before adjustments to each unique database was done(e.g making sure the search query was compatible with the search engine at any given site). The search query used in this review was: \texttt{(math or stem or mathematics) and (education or learn or learning or educational or teach or teaching) and (robot or robotics or robots) and (school or k-12)}. 

\bigskip\noindent
In order to prune the search result into a managable amount of papers we identified several inclusion and exclusion criteria. 
\begin{description}
	\item[IC1] The purpose of the paper is to investigate the usage of robotics in school, where the goal is not to teach about robotics itself.
	\item[IC2] The paper should contain some sort of assessment, quantitative or qualitative, of the learning outcome and/or experiences from the study. 
	\item[IC3] The assessment must address the development of math skills. 
	\item[IC4] The study should be done in an elementary, middle or highschool context.
	\item[IC5] The study should involve the use of physical robots.
\end{description}
These criteria diverge from Benitti's review in that qualitative assessments also are included. 
We justify this by acknowledging the fact that non-immediate returns of educational robotics may be equally important to immediate curricular related returns,
and to reflect and investigate this we allow qualitative research to take part of this review. 	

\bigskip\noindent
By negating the inclusion criteria above we get a hold of the exclusion criteria used for this review. 
The only criteria which does not have any clear negated form is IC2, we therefore define EC2 to be "`does not include any form of assessment in the form of a study"'. 
