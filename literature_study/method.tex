\section*{Method}
This review has been completed as several phases, all based upon the Kitchenham's guide for writing systematic reviews\cite{kitchenham2007guidelines}. The initial steps of identifying a need for the review and commissioning a review have however been omitted here. The steps followed is written down below.
\begin{description}
	\item[Phase 1: ] Planning
		\begin{enumerate}
			\item Specifying the research question(s)
			\item Developing a review protocol
			\item Evaluating the review protocol
		\end{enumerate}
	\item[Phase 2: ] Conducting the review
		\begin{enumerate}
			\item Identification of research
			\item Selection of primary studies
			\item Study quality assessment
			\item Data extaction and monitoring
			\item Data synthesis
		\end{enumerate}
	\item[Phase 3: ] Reporting the review
		\begin{enumerate}
			\item Communicating the results through a report.
		\end{enumerate}
\end{description}

\subsection*{Planning the review}
Initially, we preformed an inital general search into educational robotics, in order to obtain some fundemental knowledge regarding this subject. In this search we stumbled upon another review written in 2011 by Fabiane Barreto Vavassori Benitti\cite{Benitti2012978}, which to a very large extend covers the same topic as initially planned by this review. There is however some minor differences between our research question, papers relevant to Benitti will to a large extend be relevant for this review as well. Benitti asked general questions like "`\textit{What topics are taught through robotics in school}s?"', "`\textit{is robotics an effective tool for teaching? What do the studies show?}"', and "`\textit{How is student learning evaluated?}"'. While this review is going to use more narrow question, limiting the research to math and how math can be taught in schools using robotics. 

\begin{description}
	\item[Question 1: ] How can math be taught in school with the help of robotics?
	\item[Question 2: ] Is robotics an effective tool for teaching children math?
	\item[Question 3: ] Which, if any, secondary skills (teamwork, scientific inquiry etc) may also be improved through the utilization of robotics in education?
\end{description}

