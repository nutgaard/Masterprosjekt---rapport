\section*{Discussion}
In this section we discuss the results obtained in an attempt to answer the three questions presented in section~\ref{sec:questions}.

\subsection*{Which topics within math are taught through robotics in schools?}

Almost all math concepts present in elementary and middle school can be taught in some way or another through robotics. Some papers run experiments over a year and have lot of different topics. Others like the ones listed below focus on one specific topic.

\begin{description}
	\item [\tcite{mitnik2009collaborative}] graph construction and interpretation skills.
	\item [\tcite{norton2004using}] Ratio concepts. 
	\item [\tcite{silk2011resources}] Proportional reasoning.
\end{description}

\subsection*{How effective has these inquires been?}
Just because the math is present in an activity, it doesn't mean that students will learn math \tcite{silk2011resources}. This dissertation looks mostly at how the lessons have to be designed to generalize the knowledge students attain. Several problems were encountered and solutions were implemented gradually with increasing success.

\bigskip\noindent
Students using robots achieve a significant increase in their graph interpreting skills. It is twice as effective as an alternative simulation activity \tcite{mitnik2009collaborative}. This seems like a promising result and should be tested further. 

\bigskip\noindent
Main factors that is often mentioned

\bigskip\noindent
The link between activities and the underlying math must be made very explicit. 

\bigskip\noindent
Mostly everyone proposes classrooms to change, allowing students to be active learners and discover things and creating their own knowledge.

\subsection*{Which, if any, secondary skills (teamwork, scientific inquiry etc) may also be improved through the utilization of robotics in education?}

\bigskip\noindent
Teamwork \cite{mitnik2009collaborative, }

