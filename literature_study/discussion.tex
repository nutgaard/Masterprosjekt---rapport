\section{Discussion}
In this section we discuss the results obtained in an attempt to answer the three questions presented in section~\ref{sec:questions}.

\subsection{Which concepts within math are taught through robotics in schools?}
Almost all math concepts present in elementary and middle school can be taught in some way or another through robotics, something the diversity of the studies presented shows. 
This broad applicability of robotics within math also gives room for some of the bigger studies presented, which have been conducted over the course of 
a full school year\cite{hussain2006effect,lindh2007does}. 
The results from these long term studies are very important as they may, to a better extent, measure the long term effects of educational robotics. 
Most studies are however conducted over a much shorter time span, often just an intensive week of robotics tutoring. 
Thus it may be harder to measure anything other than changes in content knowledge alone.

\bigskip\noindent
A summery of the different math concepts investigated can be seen in table~\ref{tab:concepts}. 

\setlength\LTleft{0px}
\setlength\LTright{0px}
\begin{longtable}{@{\extracolsep{\fill}}p{0.38\textwidth}p{0.62\textwidth}}
	\hline \multicolumn{1}{l}{\textbf{Article}} & \multicolumn{1}{l}{\textbf{Math concepts}} \\ \hline\hline
	\endfirsthead
	
	\hline
	\hline \multicolumn{1}{l}{\textbf{Article}} & \multicolumn{1}{l}{\textbf{Math concepts}} \\ \hline\hline
	\endhead
	
	\hline
	\caption{Articles and concepts}
	\label{tab:concepts}
	\endlastfoot
	\tcite{barker2007robotics} & Decimals and geometry\\
	\tcite{nugent2008effect} & Geospatial and GPS concepts\\
	\tcite{nugent2009use} & Geospatial and GPS concepts\\
	\tcite{williams2007acquisition} & Physics\\
	\tcite{mitnik2008autonomous} & Distance, angles, kinematics, and graph construction\\
	\tcite{mitnik2009collaborative} & Graph construction and interpretation skills.\\
	\tcite{norton2004using} & Ratio concepts.\\
	\tcite{silk2011resources} & Proportional reasoning.\\
\end{longtable}

\subsection{How effective has these inquires been?}

\bigskip\noindent
10 / 12 report an increase in math skills

\bigskip\noindent
Students using robots achieve a significant increase in their graph interpreting skills. It is twice as effective as an alternative simulation activity \tcite{mitnik2009collaborative}. This seems like a promising result and should be tested further. 

\bigskip\noindent
Just because the math is present in an activity, it doesn't mean that students will learn math \tcite{silk2011resources}. This dissertation looks mostly at how the lessons have to be designed to generalize the knowledge students attain. Several problems were encountered and solutions were implemented gradually with increasing success.

\bigskip\noindent
This concern is mentioned several times in other papers as well and a common suggestion is to make the link between activities and the underlying math very explicit. Also mostly everyone proposes a classrooms dynamic change, allowing students to be active learners and create their own knowledge and mental models, as is the main idea behind constructionism \cite{papert1980mindstorms}.

\subsection{Which, if any, secondary skills (teamwork, scientific inquiry etc) may also be improved through the utilization of robotics in education?}

The main secondary skill is teamwork \cite{mitnik2009collaborative, }. When working with robots students tend to get a greater sense of community and start helping each other instead of competing. Students are also eager to help other groups and want to explain how they got their solution.

