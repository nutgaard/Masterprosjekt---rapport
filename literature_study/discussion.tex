\section{Discussion}
In this section we discuss the results obtained in an attempt to answer the three questions presented in section~\ref{sec:questions}.

\subsection{Which concepts within math are taught through robotics in schools?}
Almost all math concepts present in elementary and middle school can be taught in some way or another through robotics, something the diversity of the studies presented shows. 
This broad applicability of robotics within math also gives room for some of the bigger studies presented, which have been conducted over the course of 
a full school year~\cite{hussain2006effect,lindh2007does}. 
The results from these long term studies are very important as they may, to a better extent, measure the long term effects of educational robotics. 
Most studies are however conducted over a much shorter time span, often just an intensive week of robotics tutoring. 
Thus it may be harder to measure anything other than changes in content knowledge alone.

\bigskip\noindent
A summery of the different math concepts investigated can be seen in table~\ref{tab:concepts}. 

\setlength\LTleft{0px}
\setlength\LTright{0px}
\begin{longtable}{@{\extracolsep{\fill}}p{0.38\textwidth}p{0.62\textwidth}}
	\hline \multicolumn{1}{l}{\textbf{Article}} & \multicolumn{1}{l}{\textbf{Math concepts}} \\ \hline\hline
	\endfirsthead
	
	\hline
	\hline \multicolumn{1}{l}{\textbf{Article}} & \multicolumn{1}{l}{\textbf{Math concepts}} \\ \hline\hline
	\endhead
	
	\hline
	\caption{Articles and concepts}
	\label{tab:concepts}
	\endlastfoot
	\tcite{barker2007robotics} & Decimals and geometry\\
	\tcite{nugent2008effect} & Geospatial and GPS concepts\\
	\tcite{nugent2009use} & Geospatial and GPS concepts\\
	\tcite{williams2007acquisition} & Physics\\
	\tcite{mitnik2008autonomous} & Distance, angles, kinematics, and graph construction\\
	\tcite{mitnik2009collaborative} & Graph construction and interpretation skills.\\
	\tcite{norton2004using} & Ratio concepts.\\
	\tcite{silk2011resources} & Proportional reasoning.\\
\end{longtable}

\subsection{How effective has these inquires been?}
Most of the papers presented, and otherwise seen, throughout this literature review have provide positive evidence that educational robotics
may teach children about math. 
Out of the twelve papers presented in this review we found just two papers that did not provide any evidence of positive returns from using robotics~(\tcite{silk2011resources}, study 1 and 3). 
In \citeauthor{silk2011resources}'s forth study he did however find significant evidence of increased math content knowledge. 

\bigskip\noindent
\citeauthor{silk2011resources} argued that just because math is present in an activity, it does not mean that students will learn math~\cite{silk2011resources}.
His dissertation looks mostly at how the lessons have to be designed to generalize the knowledge students attain. Several problems were encountered and solutions were implemented gradually with increasing success. 
Thus his work provide important knowledge about how to design future endeavors into educational robotics. 

\bigskip\noindent
The concerns around disassociation between robotics and math several times in other papers as well and a common suggestion is to make the link between activities and the underlying math very explicit~\cite{nugent2008effect}. 

\bigskip\noindent
\citeauthor*{lindh2007does} study also provide interesting data regarding the effectiveness of educational robotics. 
They found that not every student may benefit from the use of robotics, and had to initially accepts their null hypothesis. 
Further investigation did however show interesting results. 
Pupils in ninth grade showed a negative \textit{t}-statistic, indicating that they in fact perform worse after partaking in the robotics experiment. 
For low performing and high performing pupils in fourth grade there was no significant difference, while there was positive results for medium performing pupils in fourth grade. 
Some consolation was however found in the correlation between fourth grade scores and fifth grade scores. 
Which, as expected, showed a positive correlation between scores. 
But did show a significantly lower correlation for the robotics group, indicating a weakening of the relationship between poor performance in forth grade and poor performance in fifth grade. 

%\bigskip\noindent
%Students using robots achieve a significant increase in their graph interpreting skills. It is twice as effective as an alternative simulation activity \tcite{mitnik2009collaborative}. This seems like a promising result and should be tested further. 

%Also mostly everyone proposes a classrooms dynamic change, allowing students to be active learners and create their own knowledge and mental models, as is the main idea behind constructionism \cite{papert1980mindstorms}.




\subsection{Which, if any, secondary skills (teamwork, scientific inquiry etc) may also be improved through the utilization of robotics in education?}
With regards to secondary skills there is a lot greater gap between the results, 
universally mentioned is however teamwork including social interactions and communication~\cite{mitnik2008autonomous,mitnik2009collaborative,nugent2009use,owens2008lego}.
When working with robots students tend to get a greater sense of community and start helping each other instead of competing. Students are also eager to help other groups and want to explain how they got their solution.

\bigskip\noindent
When testing for other secondary skills the results are to a large extent inconclusive or negative. 
\citeauthor{hussain2006effect} and \citeauthor{lindh2007does} identifies an insignificant increase  in problem-solving, \citeauthor{hussain2006effect} also identifies an insignificant positive attitude change towards LEGO~\cite{hussain2006effect,lindh2007does}. 
For scientific inquiry \citeauthor{williams2007acquisition} found no significant difference when comparing the pretest and post test~\cite{williams2007acquisition}. 
Though they argue that scientific inquiry may be a process to be learned through long exposure and that their study was to short.
\tcite{nugent2008effect} identified an increase in interest and motivation, where pupils working with robots expressed their wish to continue working with robots. Whereas the control group would do the opposite~\cite{nugent2008effect}. 
