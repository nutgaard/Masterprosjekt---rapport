\section*{Literature Search}
\subsection*{Systematic review}
A lot of work regarding a systematic review on this subject have already been done by Benitti\cite{Benitti2012978} in 2010. His work will therefore be included in this summary as well as any additional papers found. In his review he tries to answer three crucial question, namely: 

\begin{enumerate}
  \item What topics are taught through robotics in schools?
  \item How is student learning evaluated?
  \item Is robotics an effective tool for teaching?
\end{enumerate}

\bigskip\noindent
As a part of Benitti's review he identified four exclusion criteria (EC) of studies, which we also adopt.

\begin{description}
  \item[EC1] Aimed at teaching of robotics, i.e. robotics is the subject of the learning and not a teaching tool
  \item[EC2] Article does not provide a quantitative assessment of learning. If an article presented only interviews, observation and motivating
analysis, then it was excluded.
  \item[EC3] It did not show the use of robots, involving automated equipment or simulation environments with robots
	\item[EC4] The article was considered out of context, addressing undergraduate education (the focus of study is elementary, middle and high
school), or it reports the design of robots, among other aspects.
\end{description}

\bigskip\noindent
These exclusion criteria does, in other terms, limit the papers to those that look at teaching something besides robotics itself, to the lower part of the 
school system, and provide quantitative assessments of the study's effect. 
Benitti's review ends up with a total of ten studies after applying these criteria.
Details of these studies will not be included in this review, as we would rather encourage the reader to read Benittis original review. 

\bigskip\noindent
Other papers have however been located by reading the papers that has cited this paper and using google scholar with keywords education and robot. 

\bigskip\noindent
Write more about the other papers we selected here.

