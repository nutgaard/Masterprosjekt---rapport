\subsection*{Articles and findings}
\setlength\LTleft{0px}
\setlength\LTright{0px}
\begin{longtable}{@{\extracolsep{\fill}}p{0.2\textwidth}p{0.4\textwidth}@{\hspace{10pt}}p{0.4\textwidth}}
	\hline \multicolumn{1}{l}{\textbf{Article}} & \multicolumn{1}{l}{\textbf{How}} & \multicolumn{1}{l}{\textbf{What}} \\ \hline\hline
	\endfirsthead
	\hline \multicolumn{1}{l}{\textbf{Article}} & \multicolumn{1}{l}{\textbf{How}} & \multicolumn{1}{l}{\textbf{What}} \\ \hline\hline
	\endhead
	\hline \caption{Articles and findings}\endlastfoot
	\tcite{hussain2006effect} & 322 pupils in robotics group, 374 pupils as control group. \textbf{Quantitative}: pre/post-test compared with control group using a generalized linear model. \textbf{Qualitative}: observation, interview, and inquiry. & Positive returns for 5th graders math knowledge, while no returns in 9th grade. No significant returns regarding problem-solving skills was found in 5th or 9th grade. \\\hline
	
	\tcite{lindh2007does} & 322 pupils in robotics group, 374 pupils as control group. \textbf{Quantitative}: different tests in mathematics and problem solving using ANOVA test. \textbf{Qualitative}: observation, interview, and inquiry. &  No statistical evidence that the average pupil benefits from robotics. Further analysis showed that the medium scoring students did however benefit from the tutoring. \\\hline
	
	\tcite{barker2007robotics} & 14 pupils in robotics group, 18 pupils as control group. \textbf{Quantitative}: pre/post-test of non-random assigned pupils. & Increase of mean scores from pretest to posttest for robotics group. \\\hline
	\tcite{nugent2009use} & 147 pupils in robotics group, 141 pupils as control group. \textbf{Quantitative}: quasi-experimental pre/post-test of children at summer camp. Analyzed using ANCOVA/split plot ANOVA. & Significant pre/post-test increase for robotics group within programming, mathematics, geospatial concepts. In addition they found an increase in interest towards STEM.  \\\hline
	
	\tcite{mitnik2008autonomous} & Three different pairs of groups, with 12, 14, and 12 as the robotics group. And 6, 15, and 11 as the control groups. \textbf{Quantitative}: pre/post-test scheme of non-random assignment. \textbf{Qualitative}: observations  & Significant increase in the assessment of teaching distances and angles, kinematics, and graph construction. In addition they saw that students in general was more motivated to continue learning when working with robotics, whereas the control group expressed their boredom.  \\\hline
	
	\tcite{nugent2008effect} & 38 pupils attending a summer camp, no control group. \textbf{Quantitative}: pre/post-test using the same assessment in both cases. Analysis done using t-tests. & Increased test scores within four content areas; mathematics, geospatial concepts, programming and engineering. The math tutoring focused on fractions, proportions, distancerelated formulas and geometry. The authors also provide a head-up warning that youth may not see the direct connections between robotics and STEM content learning.\hline
	
	\tcite{williams2007acquisition} & 21 pupils attending a summer camp, no control group. \textbf{Quantitative}: pre/post-test using the same assessment in both cases.  & Significant impact on students gains in physics conent knowledge related to diameter of the wheels, friction, energy flow etc. In addition they observed that the scientific inquiry skills did not increase. \\\hline
	\tcite{mitnik2009collaborative} & 16-year-old students. 12 in the robot group, 11 in the simulator group. 4 60min sessions over one week. \textbf{Quantitative}: pre/post-test using the same assessment in both cases. Collaboration and motivation post activity survey. \textbf{Qualitative}: in-site motivation and collaboration observations & Fostered learning in both implementations. However the real-robot activities worked twice as well. Motivation and collaboration was far better in the real-robot group. \\\hline
	\tcite{norton2004using} & 56 year 7 students. 10 weeks of lectures at a school. \textbf{Quantitative}: Paired t pre/post-test.  & Most students improved their ability to explain mathematics concepts on pencil and paper tests. Believe that motivation is the main factor in these gains. Think that the link between the activities and the math concepts should be made more explicit. \\\hline
\end{longtable}
