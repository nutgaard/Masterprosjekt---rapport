\chapter{Concluding thoughts / Why use educational robotics?}\label{ch:literatureConclusion}
Robotics foster teamwork, motivation, interest in STEM and give students an increased understanding of the usefulness of mathematics. Robots are real, manipulable, inspectable objects, which presents students with an opportunity to develop mental models closely aligned with their preexisting knowledge of the world. In most of the studies seen the robot only works as a feedback provider to the children. They do however not utilize all the benefits of the robot and many of the tasks could be fulfilled by the use of a simulator, flash game or mobile application. Which in all cases would be cheaper and easier to acquire for the schools given the robotics landscape seen today. Another benefit is the connection between abstract concepts and physical representation given by robots, this could be especially beneficial within STEM topics as they are usually very abstract and robots may therefore help by giving a concrete understanding of these concepts.

\bigskip\noindent
Robotics does not seem to develop students' problem solving approaches. This is tested in several of our selected studies under different names. Problem solving, scientific inquiry, problem approach, students' use of a plan to guide their problem solving process etc.. None of the studies did however report a statistical significant increase in these skills. 

\bigskip\noindent
A big concern for us is the lack of relevant control groups. Even though some studies are placed in schools, the activities are extra curricular, so it does not in any way address if it is better than regular school or other alternative approaches. In other studies there are too many facilitators, making the study irrelevant for school contexts. In addition the studies that did perform it in school contexts found it helpful with more than 1 teacher in the classroom, which might indicate that robotics are too chaotic for regular classrooms. 

\bigskip\noindent
Most of the research done provided promising results. None have discovered that it worsens learning, but there are examples of it not making any difference for the average pupil compared to traditional methods~\cite{lindh2007does}. It is hard to pinpoint the factors that generates positive results, as the art of teaching and learning is extremely complex and different for each individual. The studies that focused on STEM in general contributed little to our discussion on maths in robotics. Even though they showed good results they did not present their work on mathematics separately from the other STEM concepts. The one study on physics is promising; however it is not math specific. In the mathematics specific studies, robotics was shown to be effective for graph construction and graph interpretations skills, ratios and proportional reasoning. 

%\bigskip\noindent
%We can make a contribution through adding a population to the mix. We cannot do a large scale test, which is needed the most. We cannot do a long term study either, so finding out if it fosters STEM interest is out of the question. We will add to the discussion about robot vs simulator. 

\bigskip\noindent
The literature review indentified several areas where research is needed, though many of them were not feasible to investigate during this project. 
The main areas being research utilizing low cost robots, research conducted in a close-to-real school settings, and longtitudal research with large populations. 
It is our opinion that research within these areas would greatly benefit educational robotics as a whole. 
%In the literature search there are some areas of lacking research. 
%In general there is also a lack of research with good experimental design with a larger sample, often below thirty pupils. 
%In general the lack of research can partly be blamed on the cost of robots. The most popular robot seems to be Lego Mindstorm. This robot construction kit costs 650 USD. 