\chapter{Concluding thoughts}
The current state of educational robotics does seem promising, it does however also include a major obstacle before reaching the mass populous of school children.
In most of the studies seen the robot only works as a feedback provider to the children, and this task could in many circumstances be achieved by the use of 
a simulator, flash game or app. Which in all cases would be cheaper and easier to acquire for the schools given the robotics landscape seen today.
A pedagogical counter to this situation comes from Piaget and Papert in the form of constructionism and constructivism, the latter being the dominant at this point. 
Researchers are in addition to this driven by the belief that robots awaken a tremendous source of energy and motivation in children. 
Another benefit is the connection between abstract concepts and physical representation given by robots, this could be especially beneficial within STEM topics as 
they are usually very abstract and robots may therefore help by giving a concrete understanding of these concepts. 
Thus truly justifying robotics as a superior alternative to simulator etc. 

\bigskip\noindent
In the literature search there are some areas of lacking research. Namely research involving the use of low cost robots in education. In general there is also a lack of research with good experimental design with a larger sample, often below thirty pupils. 
In general the lack of research can partly be blamed on the cost of robots. The most popular robot seems to be Lego Mindstorm. This robot construction kit costs 650 USD. 

\bigskip\noindent
In educational robotics we differentiate between academic performance and secondary skills. Academic performance concerns how school curriculum can be tough by using robots, while secondary skills are skills outside the curriculum. These are skills that you learn because of working with the robots. Often academic performance is the main goal when introducing robots to students while other skills are merely a bonus that is often not taken into account. 

\section{Academic performance}
Topics that are taught with robotics as teaching aid are mostly within the STEM (science, technology, engineering and mathematics) category. Specifically Newton's Laws of Motion, distances, angles, kinematics, graph construction and interpretation, fractions, ratios and geospatial concepts. In the systematic review carried out in \cite{Benitti2012978} 80\% of the papers focus on these topics. The two remaining papers discuss basic evolution and teaching basic social skills to kids with autism and asperger syndrome. 

\bigskip\noindent
Most of the research done provide promising results. None have discovered that it worsens learning, but there are examples of it not making any difference compared to traditional methods. It is hard to pinpoint the factors that generates positive results, as the art of teaching and learning is extremely complex and different for each individual. 
For future research we would like to propose more longitudinal studies, observing the effect of educational robotics over a large time span. 
As this would truly investigate the true effect, and not only the immediate increase in content knowledge. 
More research into this field could also mitigate the concerns about lacking empirical evidence, and possible confirm that educational robotics is
an untapped resource waiting to be utilized. 

\section{Secondary skills}
These skills are often not measured, as the research focus is on content knowledge, but these skills may have important benefits later in school and life.
The skills often include the technical skill as problem-solving, logic, and scientific inquiry. But also include non-technical skills as teamwork, social interactions, collaboration, attitude changes, and motivation.
\citeauthor{Benitti2012978} suggest the latter of these as main topics for future research, as some studies show positive trends within these skills. 
Many mention that skills such as these were improved when introducing robotics in education but more research is also needed to figure out how to train the specific skills separately. 

