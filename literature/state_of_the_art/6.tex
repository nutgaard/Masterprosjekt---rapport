\section{Which, if any, secondary skills (teamwork, scientific inquiry etc) may also be improved through the utilization of robotics in education?}
With regards to secondary skills there is a lot greater gap between the results, 
universally mentioned is however teamwork including social interactions and communication~\cite{mitnik2008autonomous,mitnik2009collaborative,nugent2009use,owens2008lego}.
When working with robots students tend to get a greater sense of community and start helping each other instead of competing. Students are also eager to help other groups and want to explain how they got their solution.

\bigskip\noindent
When testing for other secondary skills the results are to a large extent inconclusive or negative. 
\citeauthor{hussain2006effect} and \citeauthor{lindh2007does} identifies an insignificant increase  in problem-solving, \citeauthor{hussain2006effect} also identifies an insignificant positive attitude change towards LEGO~\cite{hussain2006effect,lindh2007does}. 
For scientific inquiry \citeauthor{williams2007acquisition} found no significant difference when comparing the pretest and post test~\cite{williams2007acquisition}. 
Though they argue that scientific inquiry may be a process to be learned through long exposure and that their study was to short.
\tcite{nugent2008effect} identified an increase in interest and motivation, where pupils working with robots expressed their wish to continue working with robots. Whereas the control group would do the opposite~\cite{nugent2008effect}. 