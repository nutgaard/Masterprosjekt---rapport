\section{Why use robotics to teach math?}
This chapter focuses on why robotics are nice to use in an educational setting. Argumentation for robots instead of a simulator comes after this literature study.
\subsection{Increase interest for STEM}
Many papers write about how interest in STEM can be improved through the use of educational robotics. In addition several robotics competitions like the FIRST robot league are held specifically to improve attitudes toward STEM and encourage students to pursue careers in STEM.

\cite{silk2011resources} says that competitions effectively blend a focus on excitement and fun with an opportunity to engage in an extended and challenging activity from which to learn valuable STEM-related skills. When compared to a matched comparison group from an existing national data set, alumni participants of the high-school level FIRST robotics competition were more likely to attend college, to major in a STEM-related field, and to expect to pursue a STEM-related career. 

One major question is whether and how students initial interest in working with robots might lead to development of more general understanding about the robot context and problem solving strategies for navigating it. Can students acquire more general understandings that can be applied in other similar situations?

Traditional boundaries between disciplines make it unlikely that if a student has an experience doing something with robotics that they will draw on mathematics to help understand, justify, revise or communicate their design ideas. They are more likely to focus on building a working solution that satisfies some criteria, then demonstrating without an explanation. 

\cite{nugent2009use} Research supports the use of educational robotics to increase academic achievement in specific STEM concept areas.

\cite{nugent2009use} Studies show that robotics generates a high degree of youth interest and engagement and promotes interest in math and science careers.

Some argue that this is mainly because the robot league is only about fun and does not focus on learning anything specific except doing "`engineering work"'. Research in what students actually learn during these competitions are inconclusive or bad. There is not a big focus on the math behind the solution and guess and test strategies are the techniques most often applied.

Overall STEM interest has been shown to increase when this is the focus. However when this is talked about it is often the main objective of the setup and what students actually learn is questionable.

\subsection{Control in a real-world context}

\cite{silk2011resources} says this with a reference. the robots are reliable, manipulable, and inspectable. At the same time they are not simply a made-up or imagined entity, but instead they do real things out in the physical world, and so a person's own intuitive knowledge of the physical world with apply to the robots as well. 

\cite{barker2007robotics} Papert believed that children could identify with the robots because they are concrete, physical manifestations of the computer and the computer's programs. other researchers have also identified the concrete nature of robots as being one of their important advantages. 

\cite{mitnik2009collaborative} In effect, the embodied nature of a robot provides a way to depart from the traditional abstract blackboard mased teaching scheme to a teaching model in which the student can learn by experimenting the subjects in the more natural 3D real world. Compared with a simulated environment, a physical robot can attain most of the simulation benefits. Nevertheless, an environment based on a physical element may help students to develop stronger affective bonds with it, also developing more situatinal interest. Thus robotics technologies can include and extend the benefits introduced by simulations. Simulation + 3-dimentionality, mobility, and the presence of a real, explorable and measurable setting. 

Even though this concept is mentioned often no one talks about a simulator as an alternative or mention what the difference might be. 

\subsection{problem solving}
\subsection{experimentation}
\subsection{planning}
\subsection{Motivation}
\subsection{Fun}
\subsection{Cooperative learning}
\subsection{Interest and engagement}
\subsection{Mathematically rich environment}
