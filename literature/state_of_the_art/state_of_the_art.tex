\section{Review results}\label{ch:stateOfArt}
To present the currect state of the art we will go through each of the questions highlighted in chapter~\ref{ch:literatureProcess}, before presenting our conclusion based on the knowledge we gained throughout the project.

\subsection*{How well did the different mathematics concepts get taught with robotics?}
The literature review showed that the majority of math concepts present in the elementary and middle school can be taught with the help of robotics. We have also found several other papers, not included in this study, that teach decimals, geometry, functions, graphs, coordinates, polygons, perimeter and area, symmetry, probability and algebra.
%Almost all math concepts present in elementary and middle school can be taught in some way or another through robotics, something the diversity of the studies presented shows. 
This broad applicability of robotics within math have given room for some of the bigger studies presented, as the studies presented by \citeauthor{hussain2006effect} and the study presented by \citeauthor{lindh2007does}. These studies were conducted over the course of a full school year~\cite{hussain2006effect,lindh2007does}. 
The results from these long term studies are very important as they may, to a better extent, measure the long term effects of educational robotics. 
They do however form the minority in terms on study duration as the majority of the studies in the review had duration comparable to a week. 
%Most studies are however conducted over a much shorter time span, often just an intensive week of robotics tutoring. 
Thus it may be harder to measure anything other than changes in content knowledge alone.

\bigskip\noindent
A summery of the different math concepts investigated can be seen in table~\ref{tab:concepts}. 

\setlength\LTleft{0px}
\setlength\LTright{0px}
\begin{longtable}{@{\extracolsep{\fill}}p{0.38\textwidth}p{0.62\textwidth}}
	\hline \multicolumn{1}{l}{\textbf{Article}} & \multicolumn{1}{l}{\textbf{Math concepts}} \\ \hline\hline
	\endfirsthead
	
	\hline
	\hline \multicolumn{1}{l}{\textbf{Article}} & \multicolumn{1}{l}{\textbf{Math concepts}} \\ \hline\hline
	\endhead
	
	\hline
	\caption{Articles and concepts}
	\label{tab:concepts}
	\endlastfoot
	\tcite{barker2007robotics} & Decimals and geometry\\
	\tcite{nugent2008effect} & Geospatial and GPS concepts\\
	\tcite{nugent2009use} & Geospatial and GPS concepts\\
	\tcite{williams2007acquisition} & Physics\\
	\tcite{mitnik2008autonomous} & Distance, angles, kinematics, and graph construction\\
	\tcite{mitnik2009collaborative} & Graph construction and interpretation skills.\\
	\tcite{norton2004using} & Ratio concepts.\\
	\tcite{silk2011resources} & Proportional reasoning.\\
\end{longtable}

\noindent
Most of the papers presented, and otherwise seen, throughout this literature review provided positive evidence that educational robotics
can teach children about math. 
Out of the twelve selected papers presented in this review we found just two papers that did not provide any evidence of positive returns from using robotics~(\tcite{silk2011resources}, study 1 and 3). 
In \citeauthor{silk2011resources}'s forth study he did however find significant evidence of increased math content knowledge. 

\bigskip\noindent
Even though we focused this literature study on robotics in math education, three of the studies included did not focus on math specifically \cite{barker2007robotics, nugent2008effect, nugent2009use}. They tested broader for SET(science, engineering and technology) or STEM(science, technology, engineering and math) learning gains. Therefore math subjects is only mentioned a sub part in their studies.

\bigskip\noindent
\tcite{barker2007robotics} reports on a pilot study that examined the use of a science and technology curriculum based on robotics to increase the achievement scores of youth. It was measured by a self created test to measure SET knowledge. They conclude that the activities helped teach math. %but that statement is not argued more than that.
We identified some concerns related to their approach. The most concerning was related to their analysis and the questions on their test.
Analyzing their test we found only two questions directly related to mathematics("`what is a ratio?"', and "`what does < mean?"'), and the rest where question related to broader topics or closely related to the software used in their study. Because of this we find it hard to support their conclusion that the study helped teach mathematics. A secondary conclusion in their paper stated that the hands-on experiment helped youth transform abstact science, engineering and technology concepts into concrete real-world understanding. It is our opinion that this conclusion is to a much greater extent supported by their evidence, but we reserve the right to be concerned regarding how generalizable these results are since a large part of their test were closely related to their specific software.\cite{barker2007robotics}

%They did not divide their test into different subjects and analyze them separately. When we analyzed the test they had used we could only see two questions related to math: ``what is a ratio'' and ``what does the symbol < mean?''. All the other tasks are more general and some of the questions are even very specific to the program they used to control the robots, ROBOLAB. Because of this we cannot support their conclusion that the activity helped teach math. They also conclude that through hands-on experimentation, robots helped youth transform abstract science, engineering and technology concepts into concrete real-world understanding. This seems to be better supported by the test, but their test was very specific to ROBOLAB and can not be generalized to other experimental designs or robots.

\bigskip\noindent
\tcite{nugent2008effect} used instructional activities created by 4-H to increase achievement scores and interest in STEM. 4-H is an organization that aims to develop citizenship, leadership, responsibility and life skills of youth through experiential learning programs. The test they used was based on the test created in \cite{barker2007robotics}, which we just discussed. They improved on this test by adding five questions. They claim that there were 9 mathematics questions, which would not be possible based on our previous discussion. They have not discussed the test or included it either, which makes their claim even harder to verify. 
In addition to this concern they have not included any description of the experimental design or activities. 
%There is no description of the experimental design or activities.
Their conclusion was that youth had significant increases in scores in mathematics(including fractions and ratios), programming concepts, and engineering concepts, while there was no increase in their skills related to the GPS. 
Though their analysis of the results may be consistent with their conclusions, it is not possible to investigate the underlying factors positive, or otherwise, which may have influenced their results given the information presented in the paper.
Because of this it is hard to draw any good conclusions regarding this study.\cite{nugent2008effect}
%But as we mentioned, even if it is true that students learned mathematics because of the activity, we cannot learn from the factors that assisted in this when they have not mentioned anything about their design.

\bigskip\noindent
\tcite{nugent2009use} used the 4-H instructional activities, similar to \tcite{nugent2008effect}. The activities include building and programming robots using lego mindstorms. The test they used was a paper-and-pencil, 37-item, multiple choice assessment, covering topic in computer programming, mathematics(including fractions and ratios), geospatial concepts, engineering and robotics. 
A concern regarding this study is that they did not analyze the results of the different concepts separately, and failed to include a good description of the test itself. 
%They did not analyze these different topics separately and they have not written anything specific about the test, leaving us to wonder how many math questions there were about math. 
They reported that the robotics group dramatically increased their scores. But we cannot draw any conclusion regarding mathematics learning gains from these results since we don't know anything about the test specifics.\cite{nugent2009use}

\bigskip\noindent
Overall, the studies which investigated STEM learning gains in general was not very useful for our discussion about leaning gains in mathematics. 
%Overall, the studies which looked at STEM learning gains in general was not very useful for saying anything about whether or not students learned math, even though they concluded so. 
Although the majority of papers concluded with positive results relating to learning mathmatics, we found a severe lack in explanation of experimental design as well as test design. 
%There was a severe lack in the explanation of experimental design as well as test design. 
%As they have not included their experimental design we cannot learn anything from these projects when we are creating our own experiment. 

%\bigskip\noindent
%One paper tried to teach physics with robotics, not STEM in general as %they did above. Since mathematics understanding is closely tied with physics understanding we included this study.

\bigskip\noindent
\tcite{williams2007acquisition} investigated the impact of robotics when teaching physics. Since understanding mathematics and understanding physics is closely related we included this study. 

\bigskip\noindent
\tcite{williams2007acquisition} evaluated the impact of a robotics summer camp on students' physics content knowledge and scientific inquiry skills. To collect data they used mixed methods, because they wanted to not only find out if it was effective, but explore various factors that might have contributed to the impact of the program. That is useful for us and other researchers when designing new experiments. The test consisted of twelve multiple-choice items developed by the team to assess students' understanding of Newton's Laws of Motion. There was a statistically significant difference on the physics content knowledge measure. No statistically significant difference was found when comparing scientific inquiry. The good results in physics might have been influenced by lectures about physics that were held at the summer camp, and not only the robotics activity. However these results are promising in general, the main problem for us is that the context is a summer camp.\cite{williams2007acquisition}

\bigskip\noindent
At this point we've talked about all the papers where mathematics were a considerable part of the study, but not the sole focus of the study. We now turn our attention to studies directly focused on mathematics. 
%The rest of the papers focused on mathematics specifically \cite{mitnik2009collaborative, norton2004using, lindh2007does, silk2011resources}.

\bigskip\noindent
\tcite{mitnik2009collaborative} aimed to develop graph construction and graph interpretations skills. They used a robot for a experimental group and a simulator which acted as the control group. They used the \textit{Test of Understanding Graphs in Kinematics}. This test consists of 21 multiple choice questions. The activities was shown to foster learning in both groups. 
The results showed a that the average increase of the experimental group were almost double that of the control group. 
%The experimental group with the real robots improved more, about twice as much. 
This is a very promising study since one of our main concerns when starting this literature study was if simulators were a just as good, but a cheaper alternative. This study shows that robots are more effective than simulators, at least in their setting.\cite{mitnik2009collaborative}

\bigskip\noindent
\tcite{norton2004using} investigate students' learning of ratio concepts while building robots with cogs and pulleys. The goal is to explore the learning of ratio when students design and make artifacts in which ratio thinking is embedded. In the experiment students got to work with concepts such as velocity, revolutions, linear measurement, circumference, quantification of gear ratios and quantification of pulley mechanisms. In some lessons, ten to fiftheen minutes was spent in whole group discussions of the underlying theories. They only tested for ratio learning gains. They created their own test with eleven explanatory questions. The students had to write down how they would solve certain tasks. Points were given for completeness. Students improved in their ability to explain science and mathematics concepts on pencil and paper tests. There was not much improvement in their ability to use and explain ratio and proportional reasoning in their construction of artifacts, but this statement was only supported by observation. They concluded that Lego is a powerful manipulative for the learning of mathematical ideas. We believe this study shows great promise for the use of robotics in mathematics teaching.\cite{norton2004using}

\bigskip\noindent
\tcite{lindh2007does} provide interesting data regarding the effectiveness of educational robotics. They reported on the impact of one year of lego training on school performance. The experimental group used LEGO mindstorms activities adjusted to the ordinary school activities, while the control group were students from different schools which took no part in any of the activities. The quantitative test they used was similar to national test in mathematics. They have not mentioned anything else about the test. The question they sought to answer was: \textit{will the robotics activities improve students' performance in mathematics and logical problems?}. With the null hypothesis of, lego do not have a positive or negative effect on students ability to solve mathematical or logical problems, was failed to be rejected. So in the end there was no statistical evidence that the average pupil gains from lego training, however when the ANOVA test was performed on subgroups of students based on previous performance the null hypothesis was rejected in some cases, namely for the medium good pupils. Indicating that lego training may be useful for some groups of students, but not all. This study is the most relevant study for us as it has the most participants, the longest duration, and was performed in many different schools in Sweden. This means that many of the external factors will be the same as in Norway, seen as the culture if fairly similar. The study did not have good results in learning gains, except for subgroups of students. However, it is not always practical to divide a school class into subgroups and teach in different ways. The fact that it was performed in schools is very important to us. We are attempting to find out how robotics can be used in schools and none of the previous studies have done so. The other studies have often been set up with small groups of students and many facilitators. This setting is more comparable to a tutor situation, not a school environment. They don't have to face many of the problems that exist in schools, such as conflicts, lack of teachers, many students in one room, chaos and equipment management. The negative results may not be in conflict with the other studies. Almost all the other studies was performed as part of an extracurricular activity. And when they conclude that students learned math compared to the control group this does not imply that they will learn math better through the robotics activities than through regular school activities. \cite{lindh2007does}

\bigskip\noindent
\tcite{silk2011resources} is a doctoral dissertation. 3 of the 6 experiments presented in the dissertation is included in this discussion. Part 2 was not included because there were issues with the reliability and validity of the test. Part 5 and 6 were not included because they were not relevant for this literature study.

\bigskip\noindent
In Part 1 there was a very clear goal to teach students math with robots serving as the context. The activities involved learning how the size of the wheels is related to the distance the robot moves. Four lessons were observed that were part of a larger study. These were chosen because they were the ones highlighting math the most. The test measured the students' ability to solve quantitative problems in robotics and non-robotics contexts that aligned with the instructional goals in the unit, which were proportional reasoning and concepts of measurement. The items were selected from the National Assessment of Educational Progress. There were in total 32 items, 8 items were selected for each concept and 8 items were created for the robotics context by the team for each concept. The test was split in half and one half was given as the pretest and the second as the post test. The test were included and they focus very specifically on math topics, not like the general STEM tasks in the previous studies. There were mixed results for helping students productively engage with and learn about how to control robot movements using math. The students made significant improvement in their problem solving, but the improvement was not in proportional reasoning and was not in the robot context problems either. 

\bigskip\noindent
Part 3 observed a robot competition. The goal was to measure what students actually learn when they participate in such competitions. Only a few teams used math explicitly in their design solutions, the use of math was found to have a highly variable relationship with design success. The test consisted of 10 multiple-choice and short-answer questions. All questions focused on robot motion. The items were adapted from published sources of problems that assess aspects of proportional reasoning, and then modified to focus on the robot motion problems. We looked over the questions and they are very similar to standard math tests. The results of the test showed the presence of a significant increase in students' overall problem solving from the pretest to the posttest. Participating in the competition did have some positive impact on students' overall problem solving.

\bigskip\noindent
Part 4 implemented an activity where the students should make two robots with different wheels move the same distance. This activity was called RSD(Robot Synchronized Dancing). There were two groups. The experimental group did the RSD activities and the control group prepared for a robot competition. The Problem Solving Assessment was identical to the one in \tcite{silk2011resources}. Participation in the RSD activities did have a positive effect on students' overall problem solving, but the effect was not reliably different from participating in the competition environment, so it is not clear whether it was the activities or just working with the robots in general that promoted the increase in problem solving. The implementations of the RSD unit helped students improve their understanding of the way the robots work. 

\bigskip\noindent
Part 1-4 built and improved on the previous studies ending up with part 4 where there were positive effects on students' overall problem solving. One of the conclusions is that just because math is present in an activity, it does not mean that students will learn math. The main reason behind their success in study 4 was the focus on immediately aligning students' activity with the type of thinking that was desired.

\bigskip\noindent
We have had some problems in describing why different approaches worked well or not. This is mainly because of the lacking documentations in these studies. Only a few talk about factors that might influence the results. Only a brief overview is given about group size and environment settings and it is hard to draw any conclusions except the ones they chose to put in the report, and even then it is hard to validate these results. Overall the results are quite weak and it is clear that further studies are needed.

\subsection*{What advantages or disadvantages except learning gains are there?}\label{sec:whatAdvantage}

With regards to secondary skills there is a lot greater gap between the results, 
universally mentioned is however teamwork including social interactions and communication~\cite{mitnik2008autonomous,mitnik2009collaborative,nugent2009use,owens2008lego}.
When working with robots students tend to get a greater sense of community and start helping each other instead of competing. Students are also eager to help other groups and want to explain how they got their solution.
In \tcite{lindh2007does} pupils' learning of cooperating in working groups was of interest. In interviews pupils often said the activities had enhanced the feeling of a community. This is the most relevant study for us and it is promising that collaboration was increased also here, in a school context. \tcite{mitnik2009collaborative} had a robotics experimental group and a simulator group acting as the control group. Collaboration was measured by an attitudinal questionnaire and by observation with a focus on four factors shown to foster collaboration is several other studies. The robotics group scored higher on the collaboration questionnaire and a greater amount of collaborative interactions were observed in the experimental group students compared to the simulation group students. In \tcite{nugent2008effect} results showed positive effects on students' perceptions of the value of teamwork as measured by questions probing their attitudes in working with others and listening to others while problem solving. \tcite{nugent2009use} reported non-significant results from a teamwork questionnaire. This study has not described their experimental setup, so it is hard to draw any conclusions on what might have caused these negative results. However they mention that activities happened in robotics summer camps. They speculate that peer relationships within the camp settings can influence the experience of a particular youth.

\bigskip\noindent
Students motivation is also affected either positive or negative. In \tcite{mitnik2009collaborative} both groups were motivated, but the simulator was based on newness, while motivation in the experimental group was seen to be supported by the students need and possibility to immerse into the activity. Motivation in the experimental group lasted throughout the experiment, while motivation in the control group decreased after two sessions. \tcite{nugent2008effect} identified an increase in interest and motivation, where pupils working with robots expressed their wish to continue working with robots. Whereas the control group would do the opposite~\cite{nugent2008effect}. 

\bigskip\noindent
Many papers write about how interest in STEM can be improved through the use of educational robotics. In addition several robotics competitions like the FIRST robot league are held specifically to improve attitudes toward STEM and encourage students to pursue careers in STEM. \tcite{silk2011resources} states that competitions effectively blend a focus on excitement and fun with an opportunity to engage in an extended and challenging activity from which to learn valuable STEM-related skills. When compared to a matched comparison group from an existing national data set, alumni participants of the high-school level FIRST robotics competition were more likely to attend college, to major in a STEM-related field, and to expect to pursue a STEM-related career. \tcite{nugent2008effect} The results from this study suggest that students do not directly perceive connections between robotics and STEM concepts. When working they become excited about robotics, but not recognize that STEM learning is being integrated into the activities. Middle school students, in particular, seem to need specific guidance on how they relate. Changing their STEM attitude may be hard if they do not perceive the activity as a STEM activity. The students interest in technology increased but it decreased in science.

\bigskip\noindent
Control in a real-world context is a factor mentioned in several different ways throughout the literature. \tcite{silk2011resources} argues that the robots are reliable, manipulable, and inspectable. At the same time they are not simply a made-up or imagined entity, but instead they do real things out in the physical world, and so a person's own intuitive knowledge of the physical world with apply to the robots as well. Papert believed that children could identify with the robots because they are concrete, physical manifestations of the computer and the computer's programs \cite{papert1980mindstorms}. In effect, the embodied nature of a robot provides a way to depart from the traditional abstract blackboard based teaching scheme to a teaching model in which the student can learn by experimenting the subjects in the more natural 3D real world \cite{mitnik2009collaborative}. Compared with a simulated environment, a physical robot has all the same benefits as the simulator. In addition, an environment based on a physical element may help students to develop stronger affective bonds with it, also developing more situational interest. Thus robotics technologies can include and extend the benefits introduced by simulations. The extended functionality introduced by robots include 3-dimentionality, mobility, and the presence of a real, explorable and measurable setting. 

\bigskip\noindent
Students get a better sense of the value of mathematics. It is not uncommon for students to feel that what they learn in mathematics class is useless in the real world. In several of the studies they tested specifically for this \cite{silk2011resources, nugent2009use}. The results were positive. The results might just reflect that students now feel math is useful when working with robotics, but at least it got them thinking about one topic where math is necessary in order to succeed. 

\bigskip\noindent
When testing for other secondary skills the results are to a large extent inconclusive or negative. 
\citeauthor{hussain2006effect} and \citeauthor{lindh2007does} identifies an insignificant increase  in problem-solving, \citeauthor{hussain2006effect} also identifies an insignificant positive attitude change towards LEGO ~\cite{hussain2006effect,lindh2007does}. 
For scientific inquiry \citeauthor{williams2007acquisition} found no significant difference when comparing the pretest and post test~\cite{williams2007acquisition}. 
Though they argue that scientific inquiry may be a process to be learned through long exposure and that their study was to short.
\tcite{silk2011resources} reported a decrease in robotics interest from pretest to posttest. 
\tcite{nugent2009use} had non-significant results for perceived values of GPS/GIS and problem approach. On an attitudinal assessment there was a significant increase in science and robotics task value and self-efficacy in robotics and GPS. 
\tcite{nugent2008effect} Positively impacted students' use of a plan to guide their problem solving process.

\subsection*{Were any suggested improvements to the experiments identified?}
Almost all papers used a control group that did not do anything. So their scores were, as expected, very stable between tests. 
This is a concern in our opinion as it does not provide any evidence regarding the effectiveness of educational robotics compared to ordinary tutoring. 
%This is a concern in our minds. How can they reliably test the value of robotics for example in a summer camp, when the control group has summer vacation and does not learn in any way. 
Generally all the studies' activities were extra curricular activities. The exceptions are \tcite{lindh2007does} which had robotics in schools 2 hours a week instead of traditional activities, this became robotics versus school. \tcite{mitnik2009collaborative} compared the robotics to a simulator. And in \tcite{silk2011resources} the robotics activity were matched up against a robotics competition, to see if the learning gains came from their test setup and not just generally working with robots. 

\bigskip\noindent
The learning model used was largely experimental based on constructionism. \tcite{barker2007robotics} used an experimental learning model based on other research where they have 5 steps. Experience, share, process, generalize and apply. They provided the children an opportunity to learn before being told or shown how and then share what they did. Then they considered what was important and tried to generalize the experience. Then they applied this to new situations.

\bigskip\noindent
Most of the experiments had students working in small groups. \tcite{nugent2008effect} does not mention how many students there were per group. The rest state how many students there were per group but there is not much discussion around the choice of group size. In general there was 2-3 students per group\cite{mitnik2009collaborative, norton2004using, lindh2007does, silk2011resources, nugent2009use, williams2007acquisition}. \tcite{barker2007robotics} had as many as 4-5 students per group, but they have not discussed the impact of this decision and there was one teacher per group so they could easily keep control and make sure everyone participated. \tcite{mitnik2009collaborative} used groups of three because a previous study found that groups of 4 or more provide collaborative drawbacks. In the robot competition observed by Silk \cite{silk2011resources} the teams were quite big. The average group size was 7 but the minimum allowed was 1 and maximum allowed was 10. No analysis was done to check if the team size had an impact on learning or collaboration, but that would have been interesting. \tcite{lindh2007does}, which is the most interesting paper for us, used 3-4 pupils per group. Through the study they identified several factors that could be improved and one was that groups should not be too big, maximum 2-3 pupils per group. If we exclude the irrelevant studies that had a huge amounts of teachers or were in a special context, we find that 2-3 pupils per group is ideal.

\bigskip\noindent
Another point that we have mentioned earlier is the fact that many of the studies had one facilitator per group. This is completely unrealistic in a school setting where there might be as many as 30 pupils per teacher. Therefore many of the activities described in these studies have reduced applicability in a school context. The context in these studies looks more like a tutoring session, and this is not addressed in any of the studies that does this. \tcite{lindh2007does} conducted their research in a real school context. They found that having 2 teachers in the room during the robotics activities helped a lot, because then one teacher could help while the other maintained classroom control.

\bigskip\noindent
There were several factors to consider when creating tasks. \tcite{lindh2007does} found that the tasks must be concrete, relevant and realistic  to solve. And that it is important that the pupils can relate the materials to their ordinary school work and their different subjects. \tcite{williams2007acquisition} reported that their tasks might have been too easy for the students to require or appreciate the process of  scientific inquiry. Almost none of the studies have included their tasks so it is hard to decide how we should design our own. In \tcite{williams2007acquisition} the students had to move the robot straight forward, so no wonder that it was too easy.

\bigskip\noindent
The purpose of the teacher is mostly one where teachers act as facilitators. They solve doubts about the software and problems when students ask. At least one teachers is therefore needed that can freely be able to go assist one group. It is hard to maintain classroom control at the same time, this is an issue found in a school context and therefore not addressed in most of the studies included here, since most were small experiment with often multiple people involved. \tcite{lindh2007does} Did report from school contexts and found that the role of a teacher as a mediator of knowledge and skills was crucial for coping with problems related to this kind of technology. Teachers also acted as resolver of conflicts. Even though groups collaborated well sometimes conflicts occurred. 

\bigskip\noindent
The time period varies greatly from 1 week to 1 year. With 1 week periods being dominant. However in terms of hours most of the studies were about 40 hours. Even the study that lasted over 1 year only resulted in about 96 hours in total. So the differences in studies is really about how intensive the exposure was. 

\bigskip\noindent
The sample size is very different from study to study. The by far biggest is \tcite{lindh2007does}'s study with 322 students at different schools in Sweden. \tcite{nugent2009use}'s had 147 students in total, they were spread on 6 different robotics summer camps. The rest had on average 30 students each.

\subsection*{What lacks of research are there in the literature?}

The areas that lack research is partially found from studies' future work chapters, but also from our own analysis. The biggest hole in the literature at this moment is the lack of big sample sizes, as mentioned by \tcite{mitnik2009collaborative} and \tcite{lindh2007does}. 
\tcite{lindh2007does} proposes a test run over 2-3 years in school with 3500+ students. 
%This is not realistic for us but nevertheless, it is the main lack in the literature at this moment. 

\bigskip\noindent
\tcite{barker2007robotics} mentioned the need for research in different populations. \tcite{lindh2007does} was performed in Sweden and is the only similar population to Norwegian schools. %Here we can support and add to the literature. 

\bigskip\noindent
Long term studies aimed at discovering whether the activities help foster SET as a career and school subject choices are needed. Research is needed to examine whether the program helps foster positive attitudes towards SET in school and as a career \tcite{barker2007robotics}. More research is needed on whether the program fosters SET careers \cite{barker2007robotics}. It takes commitment and a good followup system to track down students up to 10 years after a study to find out whether they chose to become engineers, or have careers within STEM. %This is not realistic for us.

\bigskip\noindent
We have found a problem with the control groups' alternative activities. In most cases the control group does nothing, while the experimental group does extracurricular activities. We want to integrate with the school system and look at how a robot can be used in an educational setting. As such we would need information on how a robot compares against regular school activities. 
%Even though a robot is nice and can teach students mathematics, are they necessary? 
\tcite{nugent2009use} have addressed the fact that it is not in a school context: ``The use of intensive summer camps offer a chance for youth to become more deeply involved in STEM activities than what might be possible in more formal educational setting''. When considering if the robot is necessary at all another question comes to mind: Why should we use robots instead of a simulator? A simulator has many advantages over robotics, such as costs and changeability. %This issue could be interesting for us to test. 

\subsection*{Conclusion}\label{ch:literatureConclusion}
Robotics foster teamwork, motivation, interest in STEM and give students an increased understanding of the usefulness of mathematics. Robots are real, manipulable, inspectable objects, which presents students with an opportunity to develop mental models closely aligned with their preexisting knowledge of the world. In most of the studies seen, the robot only works as a feedback provider to the children. They do however not utilize all the benefits of the robot and many of the tasks could be fulfilled by the use of a simulator, flash game or mobile application. Which in all cases would be cheaper and easier to acquire for the schools given the robotics landscape seen today. Another benefit is the connection between abstract concepts and physical representation given by robots, this could be especially beneficial within STEM topics as they are usually very abstract and robots may therefore help by giving a concrete understanding of these concepts.

\bigskip\noindent
Robotics does not seem to develop students' problem solving approaches. This is tested in several of our selected studies under different names. Problem solving, scientific inquiry, problem approach, students' use of a plan to guide their problem solving process etc.. None of the studies did however report a statistical significant increase in these skills. 

\bigskip\noindent
A big concern for us is the lack of relevant control groups. Even though some studies are placed in schools, the activities are extra curricular, so it does not in any way address if it is better than regular school or other alternative approaches. In other studies there are too many facilitators, making the study irrelevant for school contexts. In addition the studies that did perform it in school contexts found it helpful with more than 1 teacher in the classroom, which might indicate that robotics are too chaotic for regular classrooms. 

\bigskip\noindent
Most of the research done provided promising results. None have discovered that it worsens learning, but there are examples of it not making any difference for the average pupil compared to traditional methods~\cite{lindh2007does}. It is hard to pinpoint the factors that generates positive results, as the art of teaching and learning is extremely complex and different for each individual. The studies that focused on STEM in general contributed little to our discussion on maths in robotics. Even though they showed good results they did not present their work on mathematics separately from the other STEM concepts. The one study on physics is promising; however it is not math specific. In the mathematics specific studies, robotics was shown to be effective for graph construction and graph interpretations skills, ratios and proportional reasoning. 

\bigskip\noindent
Even though the literature review showed great promise for educational robotics, few of the papers have collaborated with schools and pedagogs in an attempt to integrate or check how their ideas will integrate into the school curriculum, and this is a concern to us. Schools often have limitations regarding both time and resources and having multiple teachers per classroom, as seen in many studies is not an option.
One major denominator in the current research is the lack of details in publications. While general trends and statistics are included in the majority of publications, we still found that detailed discussions related to validity and reliability often were excluded. This made our process of determining the current state of educational robotics extremely difficult. 

%\bigskip\noindent
%We can make a contribution through adding a population to the mix. We cannot do a large scale test, which is needed the most. We cannot do a long term study either, so finding out if it fosters STEM interest is out of the question. We will add to the discussion about robot vs simulator. 

\bigskip\noindent
The literature review indentified several areas where research is needed, though many of them were not feasible to investigate during this project. 
The main areas being research utilizing low cost robots, research conducted in a close-to-real school settings, and longtitudal research with large populations. 
It is our opinion that research within these areas would greatly benefit educational robotics as a whole. 
%In the literature search there are some areas of lacking research. 
%In general there is also a lack of research with good experimental design with a larger sample, often below thirty pupils. 
%In general the lack of research can partly be blamed on the cost of robots. The most popular robot seems to be Lego Mindstorm. This robot construction kit costs 650 USD.
%Many mention that educational robotics requires a change in the traditional curriculum in order to be more consistent with the constructionism theories. This does, in our opinion, pose one of the greatest barriers for educational robotics.
%However calling for a huge change in schools is probably not the easiest way to introduce robotics to schools if this require lots of alterations. 
