\chapter{State of the art}
We present a state of the art as a series of answers to our questions stated in the research approach.

%\section{Why use robotics to teach math?}
This chapter focuses on why robotics are nice to use in an educational setting. Argumentation for robots instead of a simulator comes after this literature study.
\subsection{Increase interest for STEM}
Many papers write about how interest in STEM can be improved through the use of educational robotics. In addition several robotics competitions like the FIRST robot league are held specifically to improve attitudes toward STEM and encourage students to pursue careers in STEM.

\cite{silk2011resources} says that competitions effectively blend a focus on excitement and fun with an opportunity to engage in an extended and challenging activity from which to learn valuable STEM-related skills. When compared to a matched comparison group from an existing national data set, alumni participants of the high-school level FIRST robotics competition were more likely to attend college, to major in a STEM-related field, and to expect to pursue a STEM-related career. 

One major question is whether and how students initial interest in working with robots might lead to development of more general understanding about the robot context and problem solving strategies for navigating it. Can students acquire more general understandings that can be applied in other similar situations?

Traditional boundaries between disciplines make it unlikely that if a student has an experience doing something with robotics that they will draw on mathematics to help understand, justify, revise or communicate their design ideas. They are more likely to focus on building a working solution that satisfies some criteria, then demonstrating without an explanation. 

\cite{nugent2009use} Research supports the use of educational robotics to increase academic achievement in specific STEM concept areas.

\cite{nugent2009use} Studies show that robotics generates a high degree of youth interest and engagement and promotes interest in math and science careers.

Some argue that this is mainly because the robot league is only about fun and does not focus on learning anything specific except doing "`engineering work"'. Research in what students actually learn during these competitions are inconclusive or bad. There is not a big focus on the math behind the solution and guess and test strategies are the techniques most often applied.

Overall STEM interest has been shown to increase when this is the focus. However when this is talked about it is often the main objective of the setup and what students actually learn is questionable.

\subsection{Control in a real-world context}

\cite{silk2011resources} says this with a reference. the robots are reliable, manipulable, and inspectable. At the same time they are not simply a made-up or imagined entity, but instead they do real things out in the physical world, and so a person's own intuitive knowledge of the physical world with apply to the robots as well. 

\cite{barker2007robotics} Papert believed that children could identify with the robots because they are concrete, physical manifestations of the computer and the computer's programs. other researchers have also identified the concrete nature of robots as being one of their important advantages. 

\cite{mitnik2009collaborative} In effect, the embodied nature of a robot provides a way to depart from the traditional abstract blackboard mased teaching scheme to a teaching model in which the student can learn by experimenting the subjects in the more natural 3D real world. Compared with a simulated environment, a physical robot can attain most of the simulation benefits. Nevertheless, an environment based on a physical element may help students to develop stronger affective bonds with it, also developing more situatinal interest. Thus robotics technologies can include and extend the benefits introduced by simulations. Simulation + 3-dimentionality, mobility, and the presence of a real, explorable and measurable setting. 

Even though this concept is mentioned often no one talks about a simulator as an alternative or mention what the difference might be. 

\subsection{problem solving}
\subsection{experimentation}
\subsection{planning}
\subsection{Motivation}
\subsection{Fun}
\subsection{Cooperative learning}
\subsection{Interest and engagement}
\subsection{Mathematically rich environment}

%\section{Which questions remain to be answered?}
Not many propose future work that does not include the same setup but with larger scale testing. Our concerns are:

Why should we use robots and not a simulator?
Does a physical robot provide a better basis for intellectual model building?
How should the effectiveness of educational robotics be tested in a specific area within mathematics? not many papers include their tests or the reasoning behind them. No pedagogical expertise is taken into account either.  

%\section{Approaches / How should we go about using robotics in schools?}
activities which are carefully designed to favor strategies that include math as a central rather than supplemental part of the activity have the best chance for achieving learning gains while sustaining engagement[Silk preface]. 

[does lego training stimulate] several researchers including Becker[1987] have stressed that the main evidence showing that LOGO can procude measurable learning when used in "`discovery"' classes has been obtained in situations close to individual tutoring. In normal-sized classes, the evidence clearly shows the need for direct instruction in the concepts and skills to be learned from LOGo, as well as further direct instruction to enable students to generalize what they have learned for transfer to other situations. This is in complete opposition to papert's conception of the discovery approach to LOGO.

[does lego training stimulate] There needs to be a large space for the pupils to work, they must be able to spread the LEGO material on the ground, play around, and test different kind of solutions for each kind of project they face. The working groups should not be too big (max 2-3 per group). the task given to the pupils must be concrete, relevant and realistic to solve. it is very important that the pupils can relate the material to their ordinary school work and their different subjects. This test was done in a regular school and therefore addressed these problems. Many other studies does not attempt to do this and therefore is not that relevant in the question: how can we integrate this technology into schools. 

[does lego training stimulate] the advantage of two teachers working parallel in the classroom. This made it possible for one teacher to aid a specific group while the other maintained control over the classroom. 

[acquisition] Provides 4 principles for design and implementation of robotics programs. 

1: Just-in-time resources such as lessons, tutorials and examples should be embedded to support scientific inquiry and acquisition of content knowledge. The finding of the study is consistent with the literature. many researchers emphasize the importance of providing resources such as case examples, informative resources and various scaffolding tools to support student-centered learning. Future studies are needed to develop and research various types of resources that are important in robotics programs. 

2: Students should be encouraged to explain their design by citing related scientific concepts and principles during debriefings. Must ask students to use ecientific concepts to explain the design of their robots and the strategies that worked or did not work for them. 

3: A robotics program should provide opportunities for students to explore the learning environment but at the sam etime encourage them to follow the process of scientific inquiry to complete design challenges. Future research may be needed to examine how to maintain the balance between free play and problem solving structured by the scientific inquiry process. 

It is a challenge to designers to maintain the interest while also bridgning that interest into a depper, more conceptually productive form of engagement. 

Challenges

Focused content, how do they learn math not just play

Motivated activity, how do you get students to care

Accessible problems, how can students understand what they are supposed to do?

Useful resources, how to you provide the resources they need to solve it?

%\section{Which concepts within math are taught through robotics in schools?}
Almost all math concepts present in elementary and middle school can be taught in some way or another through robotics, something the diversity of the studies presented shows. 
This broad applicability of robotics within math also gives room for some of the bigger studies presented, which have been conducted over the course of 
a full school year~\cite{hussain2006effect,lindh2007does}. 
The results from these long term studies are very important as they may, to a better extent, measure the long term effects of educational robotics. 
Most studies are however conducted over a much shorter time span, often just an intensive week of robotics tutoring. 
Thus it may be harder to measure anything other than changes in content knowledge alone.

\bigskip\noindent
A summery of the different math concepts investigated can be seen in table~\ref{tab:concepts}. 

\setlength\LTleft{0px}
\setlength\LTright{0px}
\begin{longtable}{@{\extracolsep{\fill}}p{0.38\textwidth}p{0.62\textwidth}}
	\hline \multicolumn{1}{l}{\textbf{Article}} & \multicolumn{1}{l}{\textbf{Math concepts}} \\ \hline\hline
	\endfirsthead
	
	\hline
	\hline \multicolumn{1}{l}{\textbf{Article}} & \multicolumn{1}{l}{\textbf{Math concepts}} \\ \hline\hline
	\endhead
	
	\hline
	\caption{Articles and concepts}
	\label{tab:concepts}
	\endlastfoot
	\tcite{barker2007robotics} & Decimals and geometry\\
	\tcite{nugent2008effect} & Geospatial and GPS concepts\\
	\tcite{nugent2009use} & Geospatial and GPS concepts\\
	\tcite{williams2007acquisition} & Physics\\
	\tcite{mitnik2008autonomous} & Distance, angles, kinematics, and graph construction\\
	\tcite{mitnik2009collaborative} & Graph construction and interpretation skills.\\
	\tcite{norton2004using} & Ratio concepts.\\
	\tcite{silk2011resources} & Proportional reasoning.\\
\end{longtable}
%\section{How effective has these inquires been?}
Most of the papers presented, and otherwise seen, throughout this literature review have provide positive evidence that educational robotics
may teach children about math. 
Out of the twelve papers presented in this review we found just two papers that did not provide any evidence of positive returns from using robotics~(\tcite{silk2011resources}, study 1 and 3). 
In \citeauthor{silk2011resources}'s forth study he did however find significant evidence of increased math content knowledge. 

\bigskip\noindent
\citeauthor{silk2011resources} argued that just because math is present in an activity, it does not mean that students will learn math~\cite{silk2011resources}.
His dissertation looks mostly at how the lessons have to be designed to generalize the knowledge students attain. Several problems were encountered and solutions were implemented gradually with increasing success. 
Thus his work provide important knowledge about how to design future endeavors into educational robotics. 

\bigskip\noindent
The concerns around disassociation between robotics and math several times in other papers as well and a common suggestion is to make the link between activities and the underlying math very explicit~\cite{nugent2008effect}. 

\bigskip\noindent
\citeauthor*{lindh2007does} study also provide interesting data regarding the effectiveness of educational robotics. 
They found that not every student may benefit from the use of robotics, and had to initially accepts their null hypothesis. 
Further investigation did however show interesting results. 
Pupils in ninth grade showed a negative \textit{t}-statistic, indicating that they in fact perform worse after partaking in the robotics experiment. 
For low performing and high performing pupils in fourth grade there was no significant difference, while there was positive results for medium performing pupils in fourth grade. 
Some consolation was however found in the correlation between fourth grade scores and fifth grade scores. 
Which, as expected, showed a positive correlation between scores. 
But did show a significantly lower correlation for the robotics group, indicating a weakening of the relationship between poor performance in forth grade and poor performance in fifth grade. 

%\bigskip\noindent
%Students using robots achieve a significant increase in their graph interpreting skills. It is twice as effective as an alternative simulation activity \tcite{mitnik2009collaborative}. This seems like a promising result and should be tested further. 

%Also mostly everyone proposes a classrooms dynamic change, allowing students to be active learners and create their own knowledge and mental models, as is the main idea behind constructionism \cite{papert1980mindstorms}.

%\section{Which, if any, secondary skills (teamwork, scientific inquiry etc) may also be improved through the utilization of robotics in education?}
With regards to secondary skills there is a lot greater gap between the results, 
universally mentioned is however teamwork including social interactions and communication~\cite{mitnik2008autonomous,mitnik2009collaborative,nugent2009use,owens2008lego}.
When working with robots students tend to get a greater sense of community and start helping each other instead of competing. Students are also eager to help other groups and want to explain how they got their solution.

\bigskip\noindent
When testing for other secondary skills the results are to a large extent inconclusive or negative. 
\citeauthor{hussain2006effect} and \citeauthor{lindh2007does} identifies an insignificant increase  in problem-solving, \citeauthor{hussain2006effect} also identifies an insignificant positive attitude change towards LEGO~\cite{hussain2006effect,lindh2007does}. 
For scientific inquiry \citeauthor{williams2007acquisition} found no significant difference when comparing the pretest and post test~\cite{williams2007acquisition}. 
Though they argue that scientific inquiry may be a process to be learned through long exposure and that their study was to short.
\tcite{nugent2008effect} identified an increase in interest and motivation, where pupils working with robots expressed their wish to continue working with robots. Whereas the control group would do the opposite~\cite{nugent2008effect}. 

\section{How well did the different mathematics concepts get taught with robotics?}
The different studies have focused on different mathematical concepts. 

\bigskip\noindent
Even though we focused this literature study on robotics in math education, three of the studies included did not focus on math specifically\cite{barker2007robotics, nugent2008effect, nugent2009use}. They tested broader for SET(science, engineering and technology) or STEM(science, technology, engineering and math) learning gains. Therefore math subjects is only mentioned a sub part in their studies.

\bigskip\noindent
\cite{barker2007robotics} Reports on a pilot study that examined the use of a science and technology curriculum based on robotics to increase the achievement scores of youth. It was measured by a self created test to measure SET knowledge. They conclude that the activities helped teach math, but that statement is not argued more than that. They did not divide their test into different subjects and analyse them separately. When we analysed the test they had used we could only see two questions related to math: ``what is a ratio'' and ``what does the symbol < mean?''. All the other tasks are more general and some of the questions are even very specific to the program they programmed the robots in, ROBOLAB. Because of this we cannot support their conclusion that the activity helped teach math. They also conclude that through hands-on experimentation, robots helped youth transform abstract science, engineering and technology concepts into concrete real-world understanding. This seems to be better supported by the test, but their test was very specific to ROBOLAB and can not be generalized to other experimental designs or robots.

\bigskip\noindent
\cite{barker2007robotics} Do not mention how they approached, what groups? why?
\cite{nugent2008effect} A clear lack in experimental setup and their decision making process. It is hard to copy when we have to think of all these, we will do an introduction to the field and explain every choice to aid future workers. 

\bigskip\noindent
\cite{nugent2008effect} Used instructional activities created by 4-H to increase achievement scores and interest in STEM. 4-H is an organization that aims to develop citizenship, leadership, responsibility and life skills of youth through experiential learning programs. The test they used was based on the test created in \cite{barker2007robotics}, which we just discussed. They improved on this test by adding 5 questions. They claim that there were 9 mathematics questions, which would not be possible based on our previous discussion. They have not discussed the test or included it either. There is no description of the experimental design or activities. Because of this it is hard to draw any good conclusions from this study. Their conclusion was that youth had significant increases in scores in mathematics(including fractions and ratios), programming concepts, and engineering concepts, while there was no increase in GPS skills.

\bigskip\noindent
\cite{nugent2008effect} robots and gps, watching robots move in real time on a gps plot

\bigskip\noindent
\cite{nugent2009use} Used the 4-H instructional activities, similar to \cite{nugent2008effect}. The activities include building and programming robots using lego mindstorms. The test they used was a paper-and-pencil, 37-item, multiple choice assessment, covering topic in computer programming, mathematics(including fractions and ratios), geospatial concepts, engineering and robotics. They did not analyse these different topics separately and they have not written anything specific about the test, leaving us to wonder how many math questions there were about math. They reported that the robotics group dramatically increased their scores. But we cannot draw any conclusion regarding mathematics learning gains from these results since we don’t know anything about the test specifics.

\bigskip\noindent
Overall the studies which looked at STEM learning gains in general was not very useful for saying anything about whether or not students learned math, even though they concluded so. There was a severe lack in the explanation of experimental design as well as test design. As they have not included their experimental design we cannot learn anything from these projects when we are creating our own experiment. 

\bigskip\noindent
One paper taught physics specifically, not just STEM in general as they did above. Since mathematics understanding is closely tied with physics understanding we included this study.

\bigskip\noindent
\cite{williams2007acquisition} Evaluated the impact of a robotics summer camp on students’ physics content knowledge and scientific inquiry skills. To collect data they used mixed methods, because they wanted to not only find out if it was effective, but explore various factors that might have contributed to the impact of the program. That is useful for us and other researchers when designing new experiments. The test consisted of twelve multiple-choice items developed by the team to assess students’ understanding of Newton’s Laws of Motion. There was a statistically significant difference on the physics content knowledge measure. No statistically significant difference was found when comparing scientific inquiry. The good results in physics might have been influenced by lectures about physics that were held at the summer camp, and not only the robotics activity. 

\bigskip\noindent
\cite{williams2007acquisition} there were short lessons and tutorials and debriefings embedded in the problem solving. this might have helped. Although they might have contributed facikitators reported that students generally showed less interest in lessons compared to robotics. 

\bigskip\noindent
The rest of the papers focused on mathematics specifically\cite{mitnik2009collaborative, norton2004using, lindh2007does, silk2011resources}.

\bigskip\noindent
\cite{mitnik2009collaborative} Aimed to develop graph construction and graph interpretations skills. They used a robot for a experimental group and a simulator which acted as the control group. They used the Test of Understanding Graphs in Kinematics. This test consists of 21 multiple choice questions. The activities was shown to foster learning in both groups. The experimental group with the real robots improved more, about twice as much. This is a very promising study since one of our main concerns when starting this literature study was if simulators were a just as good, but a cheaper alternative. This study shows that robots are more effective than simulators, at least in this setting. 

\bigskip\noindent
\cite{norton2004using} Investigate students learning of ratio concepts while building robots with cogs and pulleys. The goal is to explore the learning of ratio when students design and make artefacts in which ratio thinking is embedded. In the experiment students got to work with concepts such as velocity, revolutions, linear measurement, circumference, quantification of gear ratios and quantification of pulley mechanisms. However they only tested ratio learning gains. They created their own test with 11 explanational questions. The students had to write down how they would solve certain tasks, points given for completeness. Students improved in their ability to explain science and mathematics concepts on pencil and paper tests. There was not much improvement in their ability to use and explain ratio and proportional reasoning in their construction of artefacts, but this statement is only supported by observation. They concluded that Lego is a powerful manipulative for the  learning of mathematical ideas. We believe this study shows great promise for the use of robotics in mathematics teaching.

\bigskip\noindent
\cite{norton2004using}in some lessons, 10 to 15 minutes was spent in whole group discussions of the underlying theoires. 

\bigskip\noindent
\cite{lindh2007does} The study reports on the impact of one year of lego training on school performance. The experimental group used LEGO mindstorms activities adjusted to the ordinary school activities, while the control group were students from different schools which took no part in any of the activities. The quantitative test they used was similar to national test in mathematics. They have not mentioned anything else about the test. The question they sought to answer was: will the robotics activities improve students’ performance in mathematics and logical problems? Their null hypothesis, lego do not have a positive or negative effect on students ability to solve mathematical or logical problems, was accepted. So in the end there was no statistical evidence that the average pupil gains from lego training, however when the anova test was performed on subgroups of students the null hypothesis was rejected in some cases, namely for the medium good pupils. Indicating that lego training may be useful for some groups of students but not all. This study is the most relevant study for us. It has the most participants and the longest duration. It was performed in many different schools in Sweden, so many of the external factors will be the same as in Norway. The study did not have good results in learning gains, except for subgroups of students. However, it is not always practical to divide a school class into subgroups and teach in different ways. The fact that it was performed in schools is very important to us. We are attempting to find out how robotics can be used in schools and none of the previous studies have done so. The other studies have often been set up with small groups of students and many facilitators. This setting is more comparable to a tutor situation, not a school environment. They don’t have to face many of the problems that exist in schools, such as conflicts, lack of teachers, many students in one room, chaos and equipment management. 

\bigskip\noindent
\cite{silk2011resources} Is a doctoral dissertation. 3 of the 6 experiments presented in the dissertation is included in this discussion. Part 2 was not included because there were issues with the reliability and validity of the test. Part 5 and 6 were not included because they were not relevant for this literature study.

\bigskip\noindent
In Part 1 there was a very clear goal to teach students math with robots serving as the context. The activities involved learning how the size of the wheels is related to the distance the robot moves. 4 lessons were observed that were part of a larger study. These were chosen because they were the ones highlighting math the most. The test measured the students’ ability to solve quantitative problems in robotics and non-robotics contexts that aligned with the instructional goals in the unit, which were proportional reasoning and concepts of measurement. The items were selected from the National Assessment of Educational Progress. There were in total 32 items, 8 items were selected for each concept and 8 items were created for the robotics context by the team for each concept. The test was split in half and one half was given as the pretest and the second as the post test. The test were included and they focus very specifically on math topics, not like the general STEM tasks in the previous studies. There were mixed results for helping students productively engage with and learn about how to control robot movements using math. The students made significant improvement in their problem solving, but the improvement was not in proportional reasoning and was not in the robot context problems either. 

\bigskip\noindent
Part 3 Observed a robot competition. The goal was to measured what students actually learn when they participate in such competitions. Only a few teams used math explicitly in their design solutions, the use of math was found to have a highly variable relationship with design success. Both highest and lowest team used math. The test consisted of 10 multiple-choice and short-answer questions. All questions focused on robot motion. The items were adapted from published sources of problems that assess aspects of proportional reasoning, and then they were modified to focus on the robot motion problems. We looked over the questions and they are very similar to standard math tests. There was a significant increase in students’ overall problem solving from pre to post. Participating in the competition did have some positive impact on students’ overall problem solving.

\bigskip\noindent
Part 4 Implemented an activity where the students should make two robots with different wheels move the same distance. This activity was called RSD(Robot Synchronized Dancing). There were two groups. The experimental group did the RSD activities and the control group prepared for a robot competition. The Problem Solving Assessment was identical to the one in \cite{silk2011resources}. Participation in the RSD activities did have a positive effect on students’ overall problem solving, but the effect was not reliably different from participating in the competition environment. The implementations of the RSD unit helped students improve their understanding of the way the robots work. 

\bigskip\noindent
Part 1-4 built and improved on the previous studies ending up with part 4 where there were positive effects on students’ overall problem solving. They concluded that the main reason behind this success was the focus on immediately aligning students’ activity with the type of thinking that was desired.

\bigskip\noindent
Our problem in describing why different approaches worked well or not is often in the paper itself. Only a few talk about factors that might influence the results. Only a brief overview is given about group size and environment settings and it is hard to draw any conclusions except the ones they chose to put out there themselves. 

\bigskip\noindent
Overall the results are quite weak and it is clear that further studies are needed.

\bigskip\noindent
Another problem with student choice are that some pick students that have prerequisites. For example some pick out students who have already elected to take math in high school, which might create bias and not be very generalizable to a general student population.

\section{Other}
Robots are said to fosters collaboration. Many papers mentions this in their similar work chapter but we will focus on what is proven in the studies we picked out.
\cite{lindh2007does} Pupils learning of cooperating in working groups was of interest. In interviews pupils often said lego had enhanced the feeling of a community. Different ways to learn, trial and error or cooperative
\cite{mitnik2009collaborative} A greater amount of collaborative interactions were observed in the experimental group students. Compared to simulation
\cite{mitnik2009collaborative} Collaboration in experimental surpassed that of the control group. It increased for both groups.
\cite{nugent2009use} Nonsignificant results for teamwork 
\cite{nugent2008effect} results showed positive effects on students’ perceptions of the value of teamwork as measured by questions probing their attitudes in working with others and listening to others while problem solving. 

\bigskip\noindent
Feeling of value of math in robotics use
\cite{silk2011resources} sence of value of math for doing robotics went up. 
\cite{silk2011resources}Students made no change or were negative about math after participating in the unit. 
\cite{nugent2009use} On the attitudinal assessment the robotics group scored significantly higher for math value.

\bigskip\noindent
Interest for different subjects may be affected
\cite{silk2011resources} Decrease in robotics interest from pre to post. 
\cite{silk2011resources} Students made no change or were negative about robots
\cite{nugent2008effect} The results suggest that students do not directly perceive connections between robotics and STEM concepts. When working they become excited about robotics, but not recognize that STEM learning is being integrated into the activities. Middle school students, in particular, seem to need specific guidance on how they relate. 
\cite{nugent2009use} Nonsignificant results for perceived values of GPS/GIS and problem approach.
\cite{nugent2009use}On the attitudinal assessment the robotics group scored significantly higher as well. Worked for science, robotics task value. Self-efficacy in robotics and gps. 
\cite{nugent2008effect} No increase in attitude survey, increased stem, increased student interest in technology but decreased in science. Positively impacted students’ use of a plan to guide their problem solving process. Changing their STEM attitude may be hard if they do not perceive the activity as a STEM activity.

\bigskip\noindent
Students motivation is also affected either positive or negative.
\cite{mitnik2009collaborative} Both groups were motivated, the simulator was based on newness. Motivation throughout the experimental group was seen to be supported by the students need and possibility to immerse into the activity. Students in the experimental group were not just spectators, but became relevant actors of the activity, developing a greater commitment to their team and towards the activity resolution than the control group. Experimental group maintained the motivation throughout all experiment but the simulation decreased after two sessions. 
\cite{mitnik2009collaborative}Motivation in experimental surpassed that of the control group. Robot motivation was founded on immersion.
Robots can be tied into many different diciplines
\cite{barker2007robotics} Robots tie into a variety of disciplines. A robot is made of motors, sensors and programs. Each of these parts depends on different fields of knowledge, such as engineering, electronics and computer science. 

\bigskip\noindent
\cite{nugent2008effect} There was also a questionnaire on motivation and the use of learning strategies.

\section{What should we think about}
Almost all papers used a control group that did not do anything. So their score was kept constant. This is a concern in our minds. How can they reliably test the value of robotics for example in a summer camp, when the control group has summer vacation and does not learn in any way. In summary generally all the studies’ activities were extra curricular activities. The exceptions are \cite{lindh2007does} which had robotics in schools 2 hours a week instead of traditional activities, this became robotics vs school. \cite{mitnik2009collaborative} Compared the robotics to a simulator. And in \cite{silk2011resources} the robotics activity were matched up against a robotics competition, to see if the learning gains came from their test setup and not just generally working with robots.  

\bigskip\noindent
Many of the papers created their own tests
\cite{barker2007robotics} created their own test to evaluate SET skills. The results indicate that the evaluation used was valid and reliable for this study, but might not be applicable to other scenarios because it was quite specific. They used 2 experts to review the test for validity and relevance. Modifications were made based on their notes. It is not mentioned how they were modified or pitfalls we should avoid when creating tests.
\cite{nugent2008effect} They developed a 29 item pap MC test covering STEM, computer programming, mathematics, geospatial concepts and engineering/robotics. Based on the test from \cite{barker2007robotics}. more experts helped validate. 
\cite{williams2007acquisition} Made their own test, but have not mentioned validity

\bigskip\noindent
The learning model used was largely experimental based on constructionism.
\cite{barker2007robotics} used an experimental learning model based on some other research where they have 5 steps. Experience, share, process, generalize and apply. 
\cite{barker2007robotics} The children are provided an opportunity to learn before being told or shown how and then share what they did. Then they considered what was important and tried to generalize the experience. Then they applied this to new situations. It is based on the 5 steps above. 
\cite{barker2007robotics} Each activity begins with a brief overview of the topic covered, followed by a challenge. After the challenge they are prompted with questions that cause them to reflect and generalize. This might be hard to implement in a school, how are you supposed to ask these questions and get the students to do them?

\bigskip\noindent
However there are multiple papers arguing that an just in time teaching approach might be more helpful, as the mathematics the students are supposed to learn must be explicitly stated before they start working to be generalizable. 

\bigskip\noindent
Most papers used small groups to work in. There is not much discussion around the choice of group size. \cite{nugent2008effect} Does not mention how many students there were per group at all. In general there was 2-3 students per group\cite{mitnik2009collaborative, norton2004using, lindh2007does, silk2011resources, nugent2009use, williams2007acquisition}. One group had as many as 4-5 students per group \cite{barker2007robotics}, but they have not discussed the impact of this decision and there was one teacher per group so they could easily keep control and make sure everyone participated. \cite{mitnik2009collaborative} Used groups of three because a previous study found that groups of 4 or more provide collaborative drawbacks. In the robot competition observed by Silk \cite{silk2011resources} the teams were quite big. The average group size was 7 but the minimum allowed was 1 and maximum allowed was 10. No analysis was done to check if the team size had an impact on learning or collaboration, but that would have been interesting. \cite{lindh2007does} Which is the most interesting paper for us had 3-4 pupils per. Through the study they identified several factors that could be improved and one was that groups should not be too big, maximum 2-3 pupils per robot set. If we exclude the irrelevant studies that had a huge amounts of teachers or were in a special context, we find that 2-3 pupils per group is ideal.

\bigskip\noindent
Another point that we have mentioned earlier is the fact that many of the studies had one facilitator per group. This is completely unrealistic in a school setting where there might be as many as 30 pupils per teacher. Therefore many of the activities described in these studies have reduced applicability in a school context. The context in these studies looks more like a tutoring session, and this is not addressed in any of the studies that does this. \cite{lindh2007does} Conducted their research in a real school context. They found that having 2 teachers in the room during the robotics activities helped a lot, because then one teacher could help while the other maintained classroom control.

\bigskip\noindent
There were several factors to consider when creating tasks. \cite{lindh2007does} Found that the tasks must be concrete, relevant and realistic  to solve. And that it is important that the pupils can relate the materials to their ordinary school work and their different subjects. \cite{williams2007acquisition} Reported that their tasks might have been too easy for the students to require or appreciate the process of  scientific inquiry. Almost none of the studies have included their tasks so it is hard to decide how we should design our own. In \cite{williams2007acquisition} the students had to move the robot straight forward, so no wonder that it was too easy.

\bigskip\noindent
The purpose of the teacher is mostly one where teachers act as facilitators. Solving doubts about the software and problems when students ask, therefore teachers must be able to go assist one group, but maintain classroom control at the same time, this is an issue for normal schools not addressed many places, since it is a small experiment with often multiple people involved. This way it is more like a personal tutoring and not a classroom environment. 
\cite{lindh2007does} The role of a teacher as a mediator of knowledge and skills was crucial for coping with problems related to this kind of technology. 
\cite{lindh2007does} Teachers also acted as resolver of conflicts. even though groups collaborated well sometimes conflicts occurred. 

\bigskip\noindent
The time period varies greatly from 1 week to 1 year. With 1 week period being dominant.
\cite{mitnik2009collaborative} 4 sessions of 60 minutes
\cite{barker2007robotics} twice a week for one hour over six weeks. so 2x6 hours. 
\cite{nugent2008effect} 5 days 40 hours
\cite{nugent2009use} 40 hours in one week summer camp
\cite{williams2007acquisition} 2 weeks 2 ½ hours every day
\cite{norton2004using} most 2 hour lessons over 10 weeks
\cite{lindh2007does} 2 hours a week for 12 months

\bigskip\noindent
The sample size is very different from study to study. The by far biggest is \cite{lindh2007does} with 322 students at different schools in sweden. \cite{nugent2009use} Had 147 students in total, they were spread on 6 different robotics summer camps. The rest had on average 30 students each.

\bigskip\noindent
The participants are collected from different settings, some studies have after school programs while other have summer camps etc, which gets students from all over. The students don’t ordinarily know each other before hand and might not be relevant for an in school setting where everyone knows each other, group setups might be different and so forth. There might even be conflicts between students.
\cite{barker2007robotics} after school program do they mention if they are from the same class or who chose what?
\cite{nugent2009use} robotics summer camps
\cite{williams2007acquisition} robotics summer camp
\cite{lindh2007does} in school one year

\bigskip\noindent
Trial and error 
\cite{williams2007acquisition} Even though the scientific inquiry process was introduced, students predominantly used the trials and error method to solve problems. 
\cite{lindh2007does}more seldom are students read the written instruction from the teacher or book

\bigskip\noindent
Other proposals not mentioned yet
\cite{norton2004using}it was clear that learning of ratio could have been better, with more explicit learning of the mathematical ideas and the construction activities. 
\cite{mitnik2009collaborative} The proposal provides a change in the classroom dynamics, allowing students to be active participants in their experience of learning. 
\cite{silk2011resources} Suggests even though robots are limited, to incorporate well crafted robot challenges that make figuring out simple movements on the robots a more compelling option. 
\cite{lindh2007does} There needs to be a large space for the pupils to work, they must be able to spread the lego material on the ground play around and test different kind of solutions for each kind of project they face.
\cite{silk2011resources} In a discussion about all 4 parts they mention that the most important thing to do when designing a learning environment that help students connect math with robots is about knowing how to fine-tune the tasks such that almost immediately the students are focused and working on that aspect of the problem . 
\cite{silk2011resources} Suggests that broadening the role of math in their activities may help develop a fuller sense of the advantages of mathematics. The qualitative data in the whole class discussions also suggested that the math the students were learning may not have been tightly aligned with their understanding of robot movements.The whole class discussions focused on math ideas that didn’t directly address how to understand the way robots move, and so even when discussions were productive in terms of aligning with some math idea, it would have been unlikely that they helped students in connecting the math to the robot situation specifically. 
\cite{mitnik2009collaborative} Teammate roles are used to force cooperation among them, assuring everyone gets involved in the resolution process. Grapher role, distance measurer role and timer role. The software is difference based on what role they have. 

\section{Future work}
Much of the future work mentioned is based on bigger sample sizes. 
\cite{mitnik2009collaborative} Mentions the need for a bigger sample size.
\cite{lindh2007does} Proposes a test run over 2-3 years in school with 3500+ students. 

\bigskip\noindent
Others mentioned the lack of quantitative data we also have discovered through this literature study
\cite{barker2007robotics} There is a clear lack of quantitative research on how robotics can increase STEM achievement in students. 
\cite{williams2007acquisition} mentions limited empirical evidence

\bigskip\noindent
Some say we need different populations
\cite{barker2007robotics} more research in different populations. 

\bigskip\noindent
And long term studies aimed at discovering whether the activities help foster SET as a career and school subject choices. 
\cite{barker2007robotics} Research is needed to examine whether the program helps foster positive attitudes towards SET in school and as a career.
\cite{barker2007robotics} more research on whether it fosters SET careers. 

\bigskip\noindent
A problem we find is the control groups involvement or alternative activities. In most cases the control group does nothing while the experimental group does some activity on the side. We want to integrate with the school system and look at how a robot can be used in an educational setting. As such we would need information on how a robot compares against regular school activities. Even though a robot is nice and can teach students mathematics, are they necessary?
\cite{nugent2009use} The use of intensive summer camps offer a chance for youth to become more deeply involved in STEM activities than what might be possible in more formal educational setting where typical time constraints make extended involvement with a particular STEM application more difficult.