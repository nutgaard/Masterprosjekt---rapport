\section{How effective has these inquires been?}
Most of the papers presented, and otherwise seen, throughout this literature review have provide positive evidence that educational robotics
may teach children about math. 
Out of the twelve papers presented in this review we found just two papers that did not provide any evidence of positive returns from using robotics~(\tcite{silk2011resources}, study 1 and 3). 
In \citeauthor{silk2011resources}'s forth study he did however find significant evidence of increased math content knowledge. 

\bigskip\noindent
\citeauthor{silk2011resources} argued that just because math is present in an activity, it does not mean that students will learn math~\cite{silk2011resources}.
His dissertation looks mostly at how the lessons have to be designed to generalize the knowledge students attain. Several problems were encountered and solutions were implemented gradually with increasing success. 
Thus his work provide important knowledge about how to design future endeavors into educational robotics. 

\bigskip\noindent
The concerns around disassociation between robotics and math several times in other papers as well and a common suggestion is to make the link between activities and the underlying math very explicit~\cite{nugent2008effect}. 

\bigskip\noindent
\citeauthor*{lindh2007does} study also provide interesting data regarding the effectiveness of educational robotics. 
They found that not every student may benefit from the use of robotics, and had to initially accepts their null hypothesis. 
Further investigation did however show interesting results. 
Pupils in ninth grade showed a negative \textit{t}-statistic, indicating that they in fact perform worse after partaking in the robotics experiment. 
For low performing and high performing pupils in fourth grade there was no significant difference, while there was positive results for medium performing pupils in fourth grade. 
Some consolation was however found in the correlation between fourth grade scores and fifth grade scores. 
Which, as expected, showed a positive correlation between scores. 
But did show a significantly lower correlation for the robotics group, indicating a weakening of the relationship between poor performance in forth grade and poor performance in fifth grade. 

%\bigskip\noindent
%Students using robots achieve a significant increase in their graph interpreting skills. It is twice as effective as an alternative simulation activity \tcite{mitnik2009collaborative}. This seems like a promising result and should be tested further. 

%Also mostly everyone proposes a classrooms dynamic change, allowing students to be active learners and create their own knowledge and mental models, as is the main idea behind constructionism \cite{papert1980mindstorms}.
