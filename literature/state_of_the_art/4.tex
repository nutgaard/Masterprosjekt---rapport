\section{Which concepts within math are taught through robotics in schools?}
Almost all math concepts present in elementary and middle school can be taught in some way or another through robotics, something the diversity of the studies presented shows. 
This broad applicability of robotics within math also gives room for some of the bigger studies presented, which have been conducted over the course of 
a full school year~\cite{hussain2006effect,lindh2007does}. 
The results from these long term studies are very important as they may, to a better extent, measure the long term effects of educational robotics. 
Most studies are however conducted over a much shorter time span, often just an intensive week of robotics tutoring. 
Thus it may be harder to measure anything other than changes in content knowledge alone.

\bigskip\noindent
A summery of the different math concepts investigated can be seen in table~\ref{tab:concepts}. 

\setlength\LTleft{0px}
\setlength\LTright{0px}
\begin{longtable}{@{\extracolsep{\fill}}p{0.38\textwidth}p{0.62\textwidth}}
	\hline \multicolumn{1}{l}{\textbf{Article}} & \multicolumn{1}{l}{\textbf{Math concepts}} \\ \hline\hline
	\endfirsthead
	
	\hline
	\hline \multicolumn{1}{l}{\textbf{Article}} & \multicolumn{1}{l}{\textbf{Math concepts}} \\ \hline\hline
	\endhead
	
	\hline
	\caption{Articles and concepts}
	\label{tab:concepts}
	\endlastfoot
	\tcite{barker2007robotics} & Decimals and geometry\\
	\tcite{nugent2008effect} & Geospatial and GPS concepts\\
	\tcite{nugent2009use} & Geospatial and GPS concepts\\
	\tcite{williams2007acquisition} & Physics\\
	\tcite{mitnik2008autonomous} & Distance, angles, kinematics, and graph construction\\
	\tcite{mitnik2009collaborative} & Graph construction and interpretation skills.\\
	\tcite{norton2004using} & Ratio concepts.\\
	\tcite{silk2011resources} & Proportional reasoning.\\
\end{longtable}