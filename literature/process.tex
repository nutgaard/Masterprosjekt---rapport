\section{Process}\label{ch:literatureProcess}
As a part of defined the goals and research questions for this project we conducted a literature review. The process and planning of the review is documented in this chapter, whereas the concluding thoughs and results of the review is found in chapter~\ref{ch:literatureConclusion}. 
%In order to get proper research questions and state of the art of educational robotics we performed a literature study. 
%Here is how we conducted this. 

%After finding a lot of papers relevant to these questions, we analyzed the content and purpose of each study.
%The reviewed articles suggest that educational robotics usually results in an increase of content knowledge and a more positive attitude towards STEM. 
%Though there were also studies that reported that educational robotics did not yield any tangiable results. 
%The studies that reported negative (no increase in content knowledge) do however form the minorty out of all the studies reviewed. 
	
\subsection{Method}
The review was conducted in line with the guidelines drafted by Kitchenham's and Khan's guide for writing a systematic review~\cite{kitchenham2007guidelines,khan2001undertaking}. 
%The review was completed as several phases, based upon the Kitchenham's \cite{kitchenham2007guidelines} and Khan's~\cite{khan2001undertaking} guide for writing systematic reviews. 
The initial steps of identifying a need for the review and commissioning a review was however omitted. 
The phases used during the review was; 
%The steps followed is written down below:
\begin{description}
	\item[Phase 1: ] Planning
		\begin{enumerate}
			\item Specifying the research question(s)
			\item Developing a review protocol
			\item Evaluating the review protocol
		\end{enumerate}
	\item[Phase 2: ] Conducting the review
		\begin{enumerate}
			\item Identification of research
			\item Selection of primary studies
			\item Study quality assessment
			\item Data extaction and monitoring
			\item Data synthesis
		\end{enumerate}
	\item[Phase 3: ] Reporting the review
		\begin{enumerate}
			\item Communicating the results through a report.
		\end{enumerate}
\end{description}

\subsubsection{Planning the review}\label{sec:questions}
Initially, we preformed an general search into educational robotics, in order to obtain some fundemental knowledge regarding the subject. In this search we stumbled upon another review written in 2011 by Fabiane Barreto Vavassori Benitti\cite{Benitti2012978}, which to a very large extend covers the same topic as initially planned by this review. There was however some minor differences between our research question, but the papers relevant to Benitti proved to a large extend to be relevant for our review as well. 

\bigskip\noindent
Where Benitti asked general questions like "`\textit{What topics are taught through robotics in school}s?"', "`\textit{is robotics an effective tool for teaching? What do the studies show?}"', and "`\textit{How is student learning evaluated?}"'. We focused on narrowing down the scope of the review, limiting the review to research related to math and how math can be taught in schools using robotics.
%While this review is going to use more narrow question, limiting the research to math and how math can be taught in schools using robotics.
The research questions used for this review is stated below; 
\begin{description}
	\item[Question 1: ] How well did the different mathematics concepts get taught with robotics?
	\item[Question 2: ] What advantages or disadvantages except learning gains are there?
	\item[Question 3: ] Were any suggested improvements to the experiments identified?
	\item[Question 4: ] What lacks of research are there in the literature?
\end{description}

\bigskip\noindent
The review was conducted in February and March of 2014, with papers retrieved from all the major bibligraphic databases, including, but not limited to, CiteSeer, ACM Digital Library, SpringerLink, ERIC, IEEE XPLORE, Wiley Inter Science, and ScienceDirect. In addition to these databases the search was applied to the google scholars search engine to ensure that we got most of the relevant papers.

\bigskip\noindent
The general search query was created using groups of synonyms, concatenated by the \texttt{and}/\texttt{or} operators before adjustmentint the query to each unique database (e.g making sure the search query was compatible with the search engine at any given site). The search query used in this review was: \texttt{(math or stem or mathematics) and (education or learn or learning or educational or teach or teaching) and (robot or robotics or robots) and (school or k-12)}. 

\bigskip\noindent
To remove papers that not were of any interest we first applied a set of inclusion criteria, before applying a set of exclusion criteria. In order for a paper to be included in the review it had to pass all inclusion criteria, and not violate any of the exclusion criteria. 

\begin{description}
	\item[IC1] The purpose of the paper is to investigate the usage of robotics in school, where the goal is not to teach about robotics itself.
	\item[IC2] The paper should contain some sort of assessment, quantitative or qualitative, of the learning outcome and/or experiences from the study. 
	\item[IC3] The assessment must address the development of math skills. 
	\item[IC4] The study should be done in an elementary, middle or highschool context.
	\item[IC5] The study should involve the use of physical robots.
	\item[EC1] Article does not address the subject of using robotics to teach maths. 
	\item[EC2] Article does not address, or refer to, any pedagogical foundation. 
\end{description}
These criteria diverge from Benitti's review in that qualitative assessments also are included. 
We justify this by acknowledging the fact that non-immediate returns of educational robotics may be equally important to immediate curricular related returns,
and to reflect and investigate this we allow qualitative research to take part of this review. 	

%\bigskip\noindent
%By negating the inclusion criteria above we get a hold of the exclusion criteria used for this review. 
%The only criteria which does not have any clear negated form is IC2, we therefore define EC2 to be "`does not include any form of assessment in the form of a study"'. 



