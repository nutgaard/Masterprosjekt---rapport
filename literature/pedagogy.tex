\chapter{Constructionism}\label{chap:constructionism}
As mentioned in section~\ref{sec:decideTopic} we looked research conducted by Seymour Papert and his pedagogical theory of constructionism. 
This was done in a response to seeing how the majority of research within educational robotics based itself on Papert's theory. 
To aid the reader, we therefore present the theory of contructionism before presenting the current state of the art for educational robotics. 

\bigskip\noindent
The theory of constructionism is heavily inspired by, but should not be confused with, the constructivist theory of knowledge created by Jean Piaget~\cite{cakir2008constructivist}.
Piaget's theory of knowledge is based on the notion that knowledge is constructed in the mind of the person learning(learner)~\cite{bodner1986constructivism}. It argues that knowledge cannot be directly tranferred from a teaching entity to the learner, but that the learner needs to create their own mental models in order to grasp the concepts. 
To describe this process Piaget articulated two mechanisms describing how knowledge can be internalized by the learner, \textit{accommodation} and \textit{assimilation}. 
Where the first is refering to the process of reframing one's mental model to fit new experiences, e.g. the learner observers something that does not fit within his or hers mental model, and thus have to change the model. The latter refers to the introduction of new knowledge into the existing mental framework without the need to change the framework itself. Both processes happen through a continous iterative process of redefining and extending the framework of a learner through based on  interactions between their environment and their mental models. 

%Constructionism is a pedagogical theory. It was created by Seymore Papert. It is inspired by the constructivist theory of knowledge. This constructivist model can be summarized in a single statement: Knowledge is constructed in the mind of the learner \cite{bodner1986constructivism}. It argues that knowledge cannot simply be transferred from the teacher to the students but that individuals create mental models to understand the world around them. These mental models are created from an interaction between their experiences and their ideas in an iterative process called assimilation. When we assimilate we incorporate the new experience into an already existing framework without changing that framework. 

\bigskip\noindent
In his grant proposal to the National Science Foundation, Seymour Papert describes the difference as such:  ``From constructivist theories of psychology we take a view of learning as a reconstruction rather than as a transmission of knowledge. Then we extend the idea of manipulative materials to the idea that learning is most effective when part of an activity the learner experiences as constructing is a meaningful product''~\cite{papertGrant}. 
The main difference between constructivism and constructionism being their main focus. Constructivism focuses on how children learn, what they are interested in, and what they are able to achieve during different stages of development. Whereas Paperts constructionism focuses on how learning can be accomplished through engaging the learner, and how this facilitates the construction of knowledge~\cite{ackermann2001piaget}.

%PASDKPASKD ??? Kilde osv?
%The main difference is that where Piaget would explain the slower development of a particular concept by its greater complexity or formality, Papert see the critical factor as the relative poverty of the culture in those materials that would make the concept concrete and simple.

%\bigskip\noindent%Tror jeg flytter det meste av dette til første avsnitt
%g	The main pedagogical theory used in educational robotics is Constructionism. In fact, every single paper we have read have either used this pedagogical theory or no pedagogical theory at all. To aid the reader this is discussed before state of the art, so that we may draw conclusions in a more intuitive way. We base this discussion on the book Mindstorms: Children, computers, and powerful ideas\cite{papert1980mindstorms} by Seymore Papert. The book was the first big publication about this topic and introduces Constructionism. The book is really about how an idea, a way of thinking, comes to inhabit a young mind.

\bigskip\noindent
One reason why educational robotics is particulary applicable for the theory of constructionism is because the use of robotics is inherently a construction-based process, e.g. contructing the robot, or using a robot to construct other things. This reason, alongside with Papert's extensive research within education robotics, do, in our opinion, explain why the theory of constructionism is so popular, and why it is also the basis for this project. 
The choice of using the theory of constructionism as a basis for the project does however have two major implications in terms of experimental design and intellectual model building. 

%The reason Constructionism fits educational robotics so well is because either you construct a robot or use a robot to construct other things, and that is why we have chosen to use this theory as so many before us. Drawing on Constructionism in this project has two major implications, experimental design and intellectual model building. 

\section{Intellectual model building}
Intellectual models means the structures we have in our mind that we use when we perform tasks such as mathematics. Seymour Papert wrote that through his childhood he used gears as a mental model, even though he wasn't aware of this at that time. Whenever attending school and learning about a new concept like multiplication, he would find a way to relate this concept to the gears. 

\bigskip\noindent
The mental models described by Papert are usually found individually, and is often related to something the learner finds interesting. 
%Tror ikke denne er nødvendig
%Not everybody find mental models that they can relate to by the time they start in school. 
One of the goals of constructionism is to help students by providing a culture that is rich in materials and artifacts that the students can build mental models of to make new concepts concrete and simple. 
%Ikke denne heller
%Like other builders, children use the materials they find about them. 

\bigskip\noindent
In Papert's book, \textit{Mindstorms: Children, Computers, and powerful ideas}\cite{papert1980mindstorms}, he introduces the notion of a turtle as such an artifact. 
He introduced and tested the turtle as both a robot and a simulator. In regards to mental models he argued that there would be no difference between the robot and the simulator. The turtle was described as an ``object to think with''. Although Papert highlight the usage of something similar to a turtle, it need to be mentioned that the theory of constructionism is a generalized theory applicable to any object or artifact. 
%Through creating ``works of art'' with the robot, students need to think about how to make the robot move in a certain pattern. Because they have to program the robot to move it, they have to reflect upon how they themselves think. 
The main example concept used throughout the book is geometry. The turtle can help make geometry simple to learn. Unlike a point which is a very abstract thing, a turtle has (like humans) a position and a direction. 
This is something the students can relate to and use as a mental model. 
This way the students may at a later point utilize their mental model of a turtle to solve other problems. 
%The students can relate to this turtle and hopefully create a mental model around it, using it and turning in their heads when they need to do something. 

\bigskip\noindent
%Mental models assist us in problem solving. 
As we experiment with new concepts, our understanding and mental models are changed by our surroundings. Mental models are always adapting through the mechanisms Piaget called assimilation and accommodation.
Though its said that one can't teach an old dog new tricks, we argue that our mental models are constantly changing and evolving to fit with new experiences. 
%Our mental models are never complete, but constantly evolving to fit new scenarios. 
Mental models do have parallells to how mathematics and physics are taught in school. In introductory courses there are a lot of simplifications in order to make the curricum easy to understand, similary can a mental model contain simplified explanations of, in reality, complex phenomena. When graduating, or getting to a higher stage in your education, you learn that the thing you learned does not apply anymore, and you have to redefine your mental model to fit the new information. 
%A mental model is a very simplified explanation of a complex phenomena. It contains errors and contradictions. 
%However when we face these contradictions, that is when we evolve. 
%We try to reason and make sense of what is happening. For example when teachers try to teach physics in a traditional blackboard approach, there is usually a bunch of rules and shortcuts in the curriculum, so that students can start working on simplified problems. These rules resembles what happens in our minds, but when teachers feed us these rules they are not our own, they are someone else's way of explaining what is happening. 
One violation to this parallell is how the knowledge is gained. In traditional blackboard learning the teacher is attempting to transfer their mental models to the student, which makes adapting to new knowledge much harder than if the students were to create their own models from the start.  
%These rules are not easily understood by everyone. When constructing the models on your own, maybe reinforced by the rules afterward they will become stronger and everyone can make the connections that suits them. 

\section{Experimental design}
In the majority of cases where technology, and especially computers, are used in an educational setting, we use it by first showing a concept then asking question about it at a later stage. 
%Normally when using computers or robots in education we make the computer ``program'' the child by asking questions, showing them something on the screen etc. 
This is similar to how the regular blackboard and task solving approach of teaching works. 
In constructionism however, we replace the "`teaching \textit{at}"' with "`assisting them to understand"' while the students themselves are investigating and experimenting using the technology provided. 
%In constructionism, teaching ``at'' students is replaced by assisting them to understand while they investigate and experiment on certain tasks, for example programming the computer or the robot. 
This is done to make students explore and reflect on how they think. They have to think about what instructions to give the robot so that it will behave in the way they want. In addition to this thinking process, when programming the computer the child acquires a sense of mastery over a modern powerful technology. When programming the children learn mathematics as a living language by learning to ``talk'' to a computer. Papert compares this to learning french while living in France, compared to learning french in french class. In this way we learn how to learn and love math. We continually use math in a way that feels useful, and we avoid math being experienced as a abstract concept that students feel they will never have any use for. 

\bigskip\noindent
A disadvantage of implementing constructionism in todays educational system is however that it requires a change in the classroom dynamic and curriculum, which may not be welcomed by open arms by everyone. 
%Constructionism calls for a change in the classroom dynamic and curriculum. 
The rather experimental design is often described as ``learning without a curriculum''. Papert agreed to this by stating; ``of course the turtle can help in the teaching of traditional curriculum, but I have thought of it as a vehicle for piagetian learning, which to me is learning without curriculum''. It is however clear that Papert sees this a postive thing, rather than a thing that be used to criticize the theory. 

\bigskip\noindent
The constructionism theory relies on the fact that students are to be active creators of their own knowledge. 
%We want students to be active creators of their own knowledge. 
In order to facilitate this, the teachers need to ask question, help them explore, and than assess their skill level. 
%To to this we must make them ask questions, explore and assess what they know. 
Teaching without curriculum in this context means supporting children as they build their own intellectual structures with materials and artifacts drawn from the surroundings. 
This way learning becomes like a guided story of discovery, and not an attempt to transfer mental models between different people. 
%It becomes sort of a guided discovery. 
The tasks are designed for students to hopefully understand how they got the answers by themselves. Teachers avoid most direct instruction, but rather lead the students on a journey through questions, activities, discussions and try to make them verbalize with their discoveries in the correct way, and by doing this create their own mental models.