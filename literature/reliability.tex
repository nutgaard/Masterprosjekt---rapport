\section{Reliability and validity}\label{sec:reliabilityLiterature}
Reliability and validity are important topics within all types of research. 
These topics provide a basis for the discussion for whether we are measuring what we think we are measuring and whether these measures can be viewed as consistent and valid. 
Both within reliability and validity you can find several different concepts addressing different types of reliability and validity. 
One of example of this can be seen with validity where you often can see papers refer to internal consistency, interal validity or external validity etc.

\bigskip\noindent
This chapter will first present a short introduction to what reliability and validity are and some of the most common threats to these concepts.  
Though some factors relating to validity and reliability have already been mentioned in other chapters, 
we will include a concluding section about here. 
This is because these concepts can in various ways been seen as the linchpin of any experiment.

\subsection*{Reliability}
	When talking about reliability we usually talk about the four key concepts, \textit{equivalence reliability}, \textit{stability reliability}, \textit{internal consistency} and  \textit{interrater reliability}~\cite{colostateReliability,laerdReliability}. 
	Each estimating different constructs regarding how reliable a study is. 
	It is here important to mention that one can not measure reliability, but several strategies exist in order to attempt to estimate how reliable an experiment is.
	
	%\subsection{Equivalence reliability}
	\bigskip\noindent
	Equivalence reliability is the extent to which two items measure identical concepts at an identical level of difficulty. This is often measured as the correlation coefficient, measuring the strength of the correlation between the dependent variable and the independent variabel. 
	
	%\subsection{Stability reliability}
	\bigskip\noindent
	Stability reliability is a measure of the instruments stability, e.g how accurate an instrument is. To give an answer to this, one would usually repeat a given test to see if it gives the same results. A analysis with Cronbach's alpha between the first test and the second test can give an indicator to the stability of the test.
	
	%\subsection{Internal consistency}
	\bigskip\noindent
	Internal consistency is a measure of how well an instructument measures the same underlying concepts. A common way of measuring this is to use Cronbach's alpha between the items on a questionaire.
	
	%\subsection{Interrater reliability}
	\bigskip\noindent
	Interrater reliability is the extent to which raters agree, and is used as a measure of the rating system. Normally this is established by using a Cohen's kappa if there is only two raters, and Fleiss' kappa \cite{gwet2001handbook, shrout1979intraclass}.
	
\subsection*{Validity}
	Validity assesses the degree of which an experiment and design measures the concept that the researchers intended to measure.
	When talking about validity, we usually talk about \textit{internal validity} and \textit{external validity}, each with its own subcategories like \textit{face validity}, \textit{construct validity} and \textit{content validity}~\cite{colostateReliability,laerdQuality}.
	Internal validity refers to how the study was design, organized and conducted, and is important in order to say that an experiment
 accurately reflects the underlying concepts and constructs. 
External validity, on the other hand, is important because the results of an experiment doesn't mean much if it is only applicable for the population that participated(e.g. we want the results to be generalised and transferrable).
	We will not provide you with an exhaustive list of potensial threats to the validity of an experiment, but will here give a short overview over the most common ones.
	
	\subsubsection{Internal validity}
	Some of the threats that may threaten the internal validity of an experiment include \textit{testing effects}, \textit{selection bias}, \textit{experimental mortality} and \textit{diffusion between groups}.
	
	\bigskip\noindent
	The testing effects is an effect that can be present in any design which include multiple stages. 
	One example of this is the pretest-posttest design, where the results on the posttest can be influenced by the pretest in itself. 
	This can be because that the participants learned from completing the pretest and hence performed better on the posttest.
	
	\bigskip\noindent
	Selection bias can be another threat to the internal validity of an experiment and refers to differences between groups that may influence the results of the experiment. Though it is most prevalent in quasi-experimental design where the groups are not randomly assigned, it may occur in experimental design as well. 
	
	\bigskip\noindent
	Experimental mortality is a threat that refers to the fact that participants dropping out of the experiment for various reasons. This may become a real potensial threat in quasi-experimental design if you unintentionally created a group where the people were more likely to drop out of the experiment.
	
	\bigskip\noindent
	Diffusion between group refers to an effect where one of the group was effected by the other group. In most cases we are interested in the control group, and wether or not they have been affected by the experimental group. The "`contamination"' of the control group may happen in several different ways depending on how the experiment is done. 
	This effect can be seen as a umbrella-term for other threats involving cross-contamination between the groups (e.g. rivalry or demoralization).
	
	\subsubsection{External validity}
	Some threats to external validity have already been covered in the section about internal validity, \textit{selection bias} is one example of this. 
	Some of the more unique threats to external validity include the \textit{"`real-world"' versus "`experimental world"'} and \textit{"`faulty constructs"'} threats, where the first refers to when participants are aware that they are part of a study and may therefore alter their behaviour because of this, or that the environment in which the study is conducted influences the participants. The latter can be a bit more subtle in how it manifests itself, but refers to how well construct have been narrowed down from concepts, and how these constructs are measured.
	
	\bigskip\noindent
	The "`real world"' versus "`experimental world"' effect is often discussed in terms of the \textit{testing effect}, \textit{experimental effect} and the \textit{experimenter effect} as these effects also influence the internal validity of an experiment. Both the testing effect and the experimental effects have been mentioned previously in the section about internal validity.
	
	\bigskip\noindent
	The \textit{experimenter effect} refers to the personal biases of the the reseacher influencing the participants and/or the experiment design in such a way that the results would not be valid outside of the experimental setting. This can be as simple as non-verbal cues while observing the participants, and thus giving them a sort of validation that what they were currently doing was correct. 