\documentclass[12pt]{exam}

\usepackage[latin1]{inputenc}
\usepackage{graphicx}
\usepackage[hidelinks]{hyperref}
\usepackage{float}
\usepackage{mathtools}
\usepackage{longtable}
\usepackage[numbers]{natbib}
\usepackage{amssymb}
\usepackage{tabularx}
\usepackage{tikz}
\usepackage{longtable}
\usepackage{ifthen}
\usepackage[margin=0.75in]{geometry}
\usepackage{caption}

\newcommand{\argmax}[1]{\underset{#1}{\operatorname{arg}\,\operatorname{max}}\;}
\newcommand{\argmin}[1]{\underset{#1}{\operatorname{arg}
\,\operatorname{min}}\;}

%Figures aliases
%Usage: \image{path}{width}{caption}{label}
%Try to allways use \image and let latex fix positioning of figures.
%By passing 'null' as width the latex will use the figures default size
%Do however but the command at the wanted position within the text, 
%if we at a later stage want to fix the positioning of all images it could be done by rewriting this alias.
\newcommand{\image}[4]{
	\ifthenelse{\equal{false}{true}}{%true - true => latex fixes positioning, mismatch => forced positioning by float
		\begin{figure}[ht]
			\centering
			\insertImage{#1}{#2}
			\caption{#3}
			\label{#4}
		\end{figure}
	}{
		\imageHere{#1}{#2}{#3}{#4}
	}	
}
\newcommand{\imageHere}[4]{
	\begin{figure}[H]
		\centering
		\insertImage{#1}{#2}
		\caption{#3}
		\label{#4}
	\end{figure}
}
%Helper function, do not use
\newcommand{\insertImage}[2]{
	\ifthenelse{\equal{#2}{null}}{
		\includegraphics{#1}
	}{
		\includegraphics[width=#2]{#1}
	}
}
\newcommand{\tcite}[1]{
	\citeauthor{#1} \citeyearpar{#1}
}

%%Force correct display style for math environments
\everymath{\displaystyle}
\begin{document}
	\begin{center}
		\Huge{TurtleBot}
	\end{center}

	\section{The TurtleBot application}	
	At its very core is the TurtleBot application a combined integrated development environment and simulator,
	enabling children to explore the world of robotics and programming. 
	The application was created with extendability in mind, having two distinct methods for creating programs, 
	the possibility of extending the existing block language,
	several programming languages, and different visual setups depending on your preference and needs.
		
	\subsection{Basic screen layout}
	The application is divided into three major components; \textit{header}, \textit{body} and \textit{footer}(figure~\ref{fig:augmented}). 
	
	\bigskip\noindent
	In order to make the application easy to use by people unfamiliar to programming and robotics it utilizes a
	block programming interface as its default input method. 
	With this input method the user can drag \textit{codeblocks} from the header, down to 
	the body/programming area. After the blocks has been placed in the programming area they may be moved around
	to change the program or deleted by dragging them back to the header.
	
	\bigskip\noindent
	As seen in figure~\ref{fig:augmented} the application starts with four codeblocks the first time. 
	These blocks are the \textit{move-blocks} in the application. 
	Everytime one of these blocks are executed by the simulator it will move the robot around. 	
	
	\bigskip\noindent
	In order for you to get more familiar with the layout we have highlighted the different components, 
	and provided a simple description of them below figure~\ref{fig:augmented}. 
	\image{imgs/SimpleMainAugmented.png}{\columnwidth}{Graphics view. \\ \textbf{NB}: button 6 has been moved to the left side}{fig:augmented}
	
	\begin{description}
		\item[1] This is the available \textit{codeblocks} for your \textit{program}.
		\item[2] The \textit{startblock}. This is a special kind of \textit{codeblock}. 
			It may not be moved or removed, and must always be the first \textit{codeblock} in the \textit{program}.
		\item[3] The \textit{program}. This part consists of one of more \textit{codeblocks}, and are the part which tells the robot what to do.
			An explaination of the different codeblocks can be seen in section~\ref{sec:normal} and section~\ref{sec:extended}.
		\item[4] The \textit{Options} menu. 
		\item[5] The \textit{Run} button. Pressing this button run the \textit{program}.
		\item[6] The \textit{Clear} button. This button will remove all blocks except the \textit{startblock}.
		\item[7] The \textit{Input mode} button. This button will change the input view. 
		If you are working in the \textit{graphics view} it will change to the \textit{textual view}, and vise versa.
	\end{description}
		
	\subsection{Views}
	The header/programming area in the application has three different views, which all serve a different purpose.
	The view shown in figure~\ref{fig:augmented} is the \textit{graphics view}, here programming is done by drag'n'drop as explained.
	The second view is the \textit{textual view}, as seen in figure~\ref{fig:extendedTextual}. 
	The textual view provides the textual counterpart to the graphics view, and when used with block programming you are able to switch between the two
	without losing your program. The view also supports an alternative programming language for more experienced users.
	The textual view is considered harder to understand, and it is therefore recommended that most people start with the graphics view.
	The final view is the \textit{simulator view}. This view takes on many different visual appearances, 
	depending on the state of the application and different options. In its most basic form it will look like figure~\ref{fig:simulator},
	but depending on what has been done previously in the application it may have changed. 
	A list of all the different simulator appearances can be seen SEE SECTION IMAGE SIMULATORS.
	
	\subsection{Programming}
	\subsubsection{Normal mode}\label{sec:normal}
	When starting the application for the first time, it will start in what is called \textit{normal mode} showing the \textit{graphics view}.
	In \textit{normal mode} you have four \textit{codeblocks} available to you: 
	
	\begin{center}
		\begin{tabular}{ll}
			\textbf{FWD - Forward} & \textbf{BACK - Backward}\\
			\textbf{TL - Turn Left} & \textbf{TR - Turn Right}\\
			\label{tab:moves}
		\end{tabular}
	\end{center}
	
	\noindent
	\textbf{FWD} and \textbf{BACK} makes the robot move forward and backward respectively. The input value determines how far the robot will go.
	\textbf{TL} and \textbf{TR} makes the robot turn either left of right. The input value determines how many degrees it will turn. 
	
	\bigskip\noindent
	In figure~\ref{fig:augmented} you can see an example where all these four \textit{codeblocks} are used. 
	The resulting picture from this \textit{program} can be seen in figure~\ref{fig:simpleProg}.
	\image{imgs/SimpleProg.PNG}{null}{The result from running the program in figure~\ref{fig:augmented}}{fig:simpleProg}
	
	\subsubsection{Extended mode}\label{sec:extended}
	In \textit{extended mode} you will see five new \textit{codeblocks} appear as available (figure~\ref{fig:extendedMain}).
	These blocks are related to programs that need \textit{variables}, \textit{procedures} and \textit{loops}. 
	The usage of these blocks and how they work will be explained in another document.
	
	\image{imgs/ExtendedMain.PNG}{0.7\columnwidth}{The main view when extended mode is on.}{fig:extendedMain}
	\image{imgs/ExtendedTextual.PNG}{0.7\columnwidth}{The textual view when extended mode is on.}{fig:extendedTextual}
	\subsection{The simulator}
	The \textit{simulator view} is to show you the result of your program. 
	This view shows when you press the \textbf{run} button(numbered 5 in figure~\ref{fig:augmented}). 
	When the simulation is running you have the ability to stop it by pressing the \textbf{run} button (turned red), 
	or by pressing the \textbf{change view} button(numbered 7 in figure~\ref{fig:augmented}).
	\image{imgs/Simulator.PNG}{0.7\columnwidth}{The simulator view.}{fig:simulator}
		
	\subsection{Options}
	The \textit{options panel} allows you to change the behaviour of the application at runtime, change the inital setup, and connect to an external robot.
	The yellow \textit{Connect} button is only visible if the device has bluetooth enabled. 
	
	\bigskip\noindent
	To connect to an external robot; click on the \textit{Connect} button and wait for a list of robots to show up on the screen(figure~\ref{fig:connectionSelection}),
	click on the desired robot and wait for the connection to happen. If everything works fine the \textit{Connect} button should turn green, this is an indication that
	the application is in contact with the robot.
	\image{imgs/Options.PNG}{0.7\columnwidth}{The options view.}{fig:options}
	\image{imgs/ConnectionSelection.PNG}{0.7\columnwidth}{The list of available robots.}{fig:connectionSelection}
	\image{imgs/Connected.PNG}{0.7\columnwidth}{The options view when connected to a robot.}{fig:optionsConnected}
	\subsubsection{OnScreen keyboard}
	This is fallback solution, if the native keyboard doesn't work.
	\image{imgs/Keyboard.PNG}{0.7\columnwidth}{The on screen keyboard for numeric input.}{fig:keyboardNumeric}
	\image{imgs/KeyboardAlpha.PNG}{0.7\columnwidth}{The on screen keyboard for alphanumeric input.}{fig:keyboardAlpha}
\end{document}

