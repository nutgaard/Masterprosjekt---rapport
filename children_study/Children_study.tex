\documentclass[12pt]{article}

\usepackage[latin1]{inputenc}
\usepackage{graphicx}
\usepackage[hidelinks]{hyperref}
\usepackage{float}
\usepackage{mathtools}
\usepackage{longtable}
\usepackage{ifthen}

\newcommand{\argmax}[1]{\underset{#1}{\operatorname{arg}\,\operatorname{max}}\;}
\newcommand{\argmin}[1]{\underset{#1}{\operatorname{arg}
\,\operatorname{min}}\;}

%Figures aliases
%Usage: \image{path}{width}{caption}{label}
%Try to allways use \image and let latex fix positioning of figures.
%By passing 'null' as width the latex will use the figures default size
%Do however but the command at the wanted position within the text, 
%if we at a later stage want to fix the positioning of all images it could be done by rewriting this alias.
\newcommand{\image}[4]{
	\ifthenelse{\equal{false}{true}}{%true - true => latex fixes positioning, mismatch => forced positioning by float
		\begin{figure}[ht]
			\centering
			\insertImage{#1}{#2}
			\caption{#3}
			\label{#4}
		\end{figure}
	}{
		\imageHere{#1}{#2}{#3}{#4}
	}	
}
\newcommand{\imageHere}[4]{
	\begin{figure}[H]
		\centering
		\insertImage{#1}{#2}
		\caption{#3}
		\label{#4}
	\end{figure}
}
%Helper function, do not use
\newcommand{\insertImage}[2]{
	\ifthenelse{\equal{#2}{null}}{
		\includegraphics{#1}
	}{
		\includegraphics[width=#2]{#1}
	}
}
\newcommand{\tcite}[1]{
	\citeauthor{#1} \citeyearpar{#1}
}


%%Force correct display style for math environments
\everymath{\displaystyle}

\begin{document}
	\section{The application}
	The startup screen, see figure~\ref{fig:augmented}, is the first screen you'll see in the application (except for the colored buttons highlighted as \textbf{3}). 
	In order for you to get familiar with the application we have highlighted the different part of the application screen, 
	and provided a simple overview in section~\ref{sec:normal}. 
	\image{imgs/SimpleMainAugmented.png}{\columnwidth}{Initial application screen}{fig:augmented}
	
	\subsection{Normal mode}\label{sec:normal}
	When starting the application for the first time, it will start in what is called \textit{normal mode}.
	The differences between \textit{normal mode} and \textit{extended mode} will become clear in sections~\ref{sec:extended}, 
	but for now lets focus on the different parts highlighted in figure~\ref{fig:augmented} and what they do.
	
	\bigskip\noindent
	\begin{description}
		\item[1] This is the available \textit{codeblocks} for your \textit{program}.
		\item[2] The \textit{startblock}. This is a special kind of \textit{codeblock}. 
			It may not be moved or removed, and must always be the first \textit{codeblock} in the \textit{program}.
		\item[3] The \textit{program}. This part consists of one of more \textit{codeblocks}, and are the part which tells the robot what to do.
			An explaination of the different codeblocks can be seen in section~\ref{sec:codeblocks}.
		\item[4] The \texit{Options} menu. 
		\item[5] The \textit{Run} button. Pressing this button will change the screen into a simulator of your current \textit{program}.
		\item[6] The \textit{Clear} button. This button will remove all blocks except the \textit{startblock}.
		\item[7] The \textit{Input mode} button. This button will change the program input method, and is used to get back to programming after a simulation.
	\end{description}
	
	\bigskip\noindent
	In order to create program you drag the codeblock you want from the available blocks down to the startblock, a small indicator should appear.
	In order to change the order of your codeblocks, simply drag the codeblock you want to move into the correct position. 
	If you want to remove a codeblock from your program just drag it back to the list of available blocks at the top.
	
	\subsubsection{The different codeblocks}\label{sec:codeblocks}
	In \textit{normal mode} you have four \textit{codeblocks} available to you: 
	
	\begin{center}
		\begin{tabular}{ll}
			\textbf{FWD - Forward} & \textbf{BACK - Backward}\\
			\textbf{TL - Turn Left} & \textbf{TR - Turn Right}\\
			\label{tab:moves}
		\end{description}
	\end{center}
	
	\noindent
	\textbf{FWD} and \textbf{BACK} makes the robot move forward and backward respectively. The input value determines how far the robot will go.
	\textbf{TL} and \textbf{TR} makes the robot turn either left of right. The input value determines how many degrees it will turn. 
	
	\bigskip\noindent
	In figure~\ref{fig:augmented} you can see an example where all these four \textif{codeblocks} are used. 
	If you run the program you will end up with a picture as seen in figure~\ref{fig:simpleProg}.
	\image{imgs/SimpleProg.PNG}{null}{The result from running the program in figure~\ref{fig:augmented}}{fig:simpleProg}
	
	\subsection{Extended mode}\label{sec:extended}
	\subsection{Connecting to the robot}
\end{document}

