\section{Discussion}
In this chapter we try to answer our research questions based on the insights gained through the experiment. 

\paragraph{How can using robotics in school help students understand what an angle is and aid in their angle estimation skills?}~\\
Angles and mathematics in general can be taught with robotics by using the robot as an object to think with, and create mental models around. It is also an excellent object to use in an experimental classroom design, as the answer is not straight forward and experimentation is needed to create most programs. Through our experiments we found that both groups using robots and simulators performed better on the posttest than on the pretest. However these results were statistically insignificant. When an outlier from the robotics group were excluded it revealed a statistically significant mean increase. 

\bigskip\noindent
Both the robot and simulator fostered collaboration. We encountered a problem during the experimentation phase. If students in one group had different levels of understanding, the experimental classroom design failed, because one student dominated the group and had all the answers. The rest of the team members then did not get to experiment, fail and reason about these failures. More challenging exercises might be needed for these students, but it is hard to make customized tasks to different students in the same class, or in the same group. A different solution might be to force group roles as in \cite{mitnik2009collaborative}, to make the experiments interesting for everyone in the group, even though some students understand the concepts better.

\bigskip\noindent
Regarding problem solving, a trial and error approach was present to some degree in all groups and the students used their protractors wrong. The activities we presented did not alter the students' approach. We found that overall students need clear instructions and should be presented with opportunities to explain how they solved their tasks, or they might just apply a trial and error approach. However, asking every group to explain their solution is a time consuming process, and hard for teachers while at the same time maintaining classroom control. One possible solution is to ask the students to write down an explanation of their program and choose a random group to present this explanation at the end of the lesson.

\bigskip\noindent
Both the robots and simulators were fun to work with for the students. Teaching is easier when students are having fun and are engaged.

\bigskip\noindent
Through the depth questions it became clear that the activities helped students understand and describe what an angle is, as everyone managed to describe an angle accurately in the posttest. 

\bigskip\noindent
Our results are consistent with previous research presented in section \ref{ch:stateOfArt}. Robotics foster teamwork, motivation, and give students an increased sense of how robotics can utilize mathematics. We could not change the students' problem solving approach. This is a hard process and a clear step by step approach should possibly be explained to the students before letting them experiment on the robots. 

\paragraph{What are the strength and weaknesses of using a robot compared to a simulated environment?}~\\
Overall there was a big difference between using robots and using a simulator.
The increase from pretest to posttest was greater for the simulator group than for the robot group, but not a statistically significant difference. 

\bigskip\noindent
Through the questionnaire we found that the robotics group though the activities were more fun and interesting. This was also supported by our observations during the experiment. We also found that the students in the simulator scored higher on the question: ``Do you think imagining a robot turning can help you when working with angles?''. This result was surprising, as we thought the robotics group would score higher on this question.

\bigskip\noindent
Robots appear to be more fun, but this is not necessarily a good thing. Students were more preoccupied and did not listen to everything we said. In addition some students did not focus on the tasks given to them. Thus maybe hindering their learning gains from the experience. 

\bigskip\noindent
When using a physical robot students did not differentiate between two identical programs with the exception of a turn left or right at the start, as they could have done that themselves by putting the robot in that direction before starting the program. In the simulator group this was seen as two different approaches, even though they could have rotated the tablet in the same manner, this was not obvious to them as it was in the robotics group. This attests to the positive benefits of using a physical robot instead of a simulated environment. 