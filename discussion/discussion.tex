\section{Discussion}
A problem encountered are good students. Experimental classroom design failed when one student dominated the group and had all the answers. The rest of the team members then did not get to fail and try new solutions. More challenging exercises might be needed for these students, but it is hard to make customized tasks to different students in the same class. A different solution might be to force group roles as in \tcite{mitnik2009collaborative}~\cite{mitnik2009collaborative}, to the experiments interesting for everyone in the group, even though one student might understand it better.

\bigskip\noindent
We found that overall students need a clear instruction and should be presented with opportunities to explain how they solved their tasks, or they might just apply a trial and error approach. 

\bigskip\noindent
Below we try to answer our research questions based on the insights gained through the experiment. 

\paragraph{How can using robotics in school help students understand what an angle is and aid in their angle estimation skills?}~\\
Angles and mathematics in general can be taught with robotics by acting as an object to think with. It is also an excellent object to use in an experimental classroom design. Through our experiments we found that both groups using robots and simulators performed better on the posttest than on the pretest. However these results were statistically insignificant. When an outlier from the robotics group were excluded it revealed a statistically significant mean increase. 

\bigskip\noindent
Both the robots and simulators were fun to work with for the students. Teaching is made easier when students are having fun and are engaged.

\bigskip\noindent
Through the depth questions it became clear that the activities helped students understand and describe what an angle is, as everyone managed to describe an angle accurately in the posttest. 

\paragraph{What are the strength and weaknesses of using a robot compared to a simulated environment?}~\\
Overall there was not found a big difference between using robots and using a simulator.
The increase from pretest to posttest was greater for the simulator group than for the robot group, but not a statistically significant difference. 

\bigskip\noindent
Through the questionnaire we found that the robotics group though the activities were more fun and interesting. This was also supported by our observations during the experiment. We also found that the students in the simulator scored higher on the question: ``Do you think imagining a robot turning can help you
when working with angles?''. 

\bigskip\noindent
Robots appear to be more fun, but this is not necessarily a good thing. Students were more preoccupied and did not listen to everything we said. In addition some students did not focus on the tasks given to them. Thus maybe hindering their learning gains from the experience. 

\bigskip\noindent
When using a physical robot students did not differentiate between two identical programs with the exception of a turn left or right at the start, as they could have done that themselves by putting the robot in that direction before starting the program. In the simulator group this was seen as two different approaches, even though they could have rotated the tablet in the same manner, this was not obvious to them as it was in the robotics group. This attests to the positive benefits of using a physical robot instead of a simulated environment. 