\section{Literature study}
All the papers we have included have a pre post test.

Use constructionism as the theoretical base for designing robotics activities for educational purposes.

Almost all the papers say they use paperts theories in their experiements. However they only look at the experimental design of curriculum when they say that. In previous work some mention that papert talks about mental models and learning about learning. But this is never tested for. The pre post tests are merely on knowledge gains. 

They all create their own curriculum without consulting pedagogs or teacher in doing so. There is no effort to adapt it to traditional or the schools curriculum thus making it hard to say anything about applicability in schools. 

Most results are good, but the approaches are not realistic. It is hard to change the entire school system at once, and even though experimental design seems to be the most promising much other research say that math needs to be more explicitly connected to the curriculum.

We will look at how a robot can be integrated into the school. We will talk to teachers and find the best approach. We will only look at angles and turn measurement. We will focus on the robot as a mental model and ask questions that can answer whether or not a robot is better than a simulator.

\section{Pedagogic Theory}
Constructivism is not a pedagogy but a theory describing how learning happens. The theory suggests that we construct knowledge out of experiences. 
Constructionism is a pedagogy describing how to use objects to create mental models. From constructivist theories of psychology we take a view of learning as a reconstruction rather than as a transmission of knowledge. Then we extend the idea of manipulative materials to the idea that learning is most effective when part of an activity the learner experiences as constructing is a meaningful product. 

Constructivism, a model of children as builders of their own intellectual structures from jean piaget is the base constructionism is built upon. It basically says that humans generate knowledge and meaning from an interaction between their experiences and their ideas in a process called assimilation. When we assimilate we incorporate the new experience into an already existing framework without changing that framework. 

The main difference is that where piaget would explain the slower development of a particular concept by its greater complexity or formality, papert see the critical factor as the relative poverty of the culture in those materials that would make the concept concrete and simple. Aka a huge focus on objects as catalysts for knowledge.

constructivists seem to be having difficulties defining testable learning theories.

Constructionism

Chapter 1 paragrapf 2:
This powerful image of child as epistemologist caught my imagination while I was working with Piaget. In 1964, after five years at Piaget's Center for genetic epistemology in geneva, I came away impressed by his way of looking at children as the active builders of their own intellectual structures. But to say that intellectual structures are built by the learner rather than taught by a teacher does not mean that they are built from nothing. On the contrary: Like other builders, children appropriate to their own use materials they find about them, most saliently the models and methaphores suggested by the surrounding culture. 

Teaching "at" students is replaced by assisting them to understand. Computer aided learning usually means making the computers teach the child, the computer programs the child. In paperts vision the child programs the computer and in doing so aquires a scence of mastery over a modern powerful technology.

Intelectual model building

Children learn mathematics as a living language by learning to "`talk"' to a computer.

Things that are seen as too formal or too mathematical will be learned just as easy in a computer rich world. Kind of disproved i guess I should read up on this in future research by papert!

Learn much as we speak instead of learning it in school.

Book is really about how a culture, a way of thinking, an idea comes to inhabit a young mind.

Learn to learn and love math

"`object to think with"' - turtle/robot

Learn to use touch sensors to avoid objects. 

Learning how to exercice controll over an exceptionally rich and sophisticated "`micro-world"' in all different cases of stuff.

Want the child to program the computer, not the computer to program the child. This to explore how themselves think.

of course the turtle can help in the teaching of traditional curriculum, but I have thought of it as a vehicle for piagetian learning, which to me is learning without curriculum.

Teaching without curriculum does not mean anarchy. It means supporting children as they build their own intellectual structures with materials drawn from the surrounding culture. 

Turtle is only one example. Use anything!

What we draw from constructivism

First off we use logo. So pretty much everything from that relates to us

Mircoworlds

We do not concern ourselves with school curriculum but we want to create a platform for teaching with a robot and as such might introduce or propose small alterations to the classroom dynamic. As papert mentions it is important for students to reach a goal by themselves, and not be programmed by the curriculum.

We will make students walk and turn according to what they think the program will do. Maybe with one arm held out in the direction they started turning and then one arm in front of them as they turn.

Notes irrelevant for our study 
Mathphobia, only have problem with things recognized as math.
School math teaching is poision and creates mathphobia

\section{Our approach}
We will use the basic ideas presented by papert.

Research shows that robots create better results, and a lot of the research is based on papert. However none explicitly mentions the reason for learning being a stronger mental model and an easier time to imagine the robot instead of alternatives. 

We would like to test our belief that using robots can ease the creation of mental models and assist in intellectual model building. In addition it is shown to be more fun and foster collaboration.

Following paperts ideas we will not be teaching students by programming them. Instead of using the robot as a task giver we will want them to solve tasks by programming it themselves and learn. The main difference being traditional computer systems give students tasks to complete, here the robot is merely a slave and only obeys commands.

Research done by silk again and again shows that the math needs to be made explicit when working with robots, to avoid a big focus on trial and error by students. This is also shown in many logo papers. clements work, mainly Development of turn and turn measurement concepts in a computer-based instructional unit. also mentions that trial and error and misconceptions were a big issue. For example in silks work students just try to adapt one step at a time instead of understanding underlying math. In clements case studens turned a random direciton of left or right when creating a square. If they turned the wrong way they just moved backwards instead of forward. 

The students will learn how to exercise control over a micro-world namely the robot through the tablet app. We will use the robot as our microworld and compare it to a simulated microworld.

We want the child to program the computer to explore how themselves think. To highlight how the students think they will perform the actions they think the robot will do before trying to run the app. By for example extending one arm turning 

Of course the turtle can help in the teaching of traditional curriculum, but I have thought of it as a vehicle for piagetian learning, which to me is learning without curriculum. We will work around the existing curriculum found at THIS.

In addition to the pedagogic believes the research largely shows that robots are fun, motivating and fosters collaboration. We will have the students working in groups of three based on the experience of the teachers at THIS.
