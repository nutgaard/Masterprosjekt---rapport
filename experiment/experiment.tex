
\documentclass[12pt]{exam}

\usepackage[latin1]{inputenc}
\usepackage{graphicx}
\usepackage[hidelinks]{hyperref}
\usepackage{float}
\usepackage{mathtools}
\usepackage{longtable}
\usepackage[numbers]{natbib}
\usepackage{amssymb}
\usepackage{tabularx}
\usepackage{tikz}
\usepackage{longtable}
\usepackage{ifthen}
\usepackage[margin=0.75in]{geometry}
\usepackage{caption}
\usepackage{listings}
\usepackage{color}

\newcommand{\argmax}[1]{\underset{#1}{\operatorname{arg}\,\operatorname{max}}\;}
\newcommand{\argmin}[1]{\underset{#1}{\operatorname{arg}
\,\operatorname{min}}\;}

%Figures aliases
%Usage: \image{path}{width}{caption}{label}
%Try to allways use \image and let latex fix positioning of figures.
%By passing 'null' as width the latex will use the figures default size
%Do however but the command at the wanted position within the text, 
%if we at a later stage want to fix the positioning of all images it could be done by rewriting this alias.
\newcommand{\image}[4]{
	\ifthenelse{\equal{false}{true}}{%true - true => latex fixes positioning, mismatch => forced positioning by float
		\begin{figure}[ht]
			\centering
			\insertImage{#1}{#2}
			\caption{#3}
			\label{#4}
		\end{figure}
	}{
		\imageHere{#1}{#2}{#3}{#4}
	}	
}
\newcommand{\imageHere}[4]{
	\begin{figure}[H]
		\centering
		\insertImage{#1}{#2}
		\caption{#3}
		\label{#4}
	\end{figure}
}
%Helper function, do not use
\newcommand{\insertImage}[2]{
	\ifthenelse{\equal{#2}{null}}{
		\includegraphics{#1}
	}{
		\includegraphics[width=#2]{#1}
	}
}
\newcommand{\tcite}[1]{
	\citeauthor{#1} \citeyearpar{#1}
}

%%Force correct display style for math environments
\everymath{\displaystyle}

%%Code highlighting setup
\definecolor{lightgray}{rgb}{.9,.9,.9}
\definecolor{darkgray}{rgb}{.4,.4,.4}
\definecolor{purple}{rgb}{0.65, 0.12, 0.82}
\lstdefinelanguage{JavaScript}{
  keywords={break, case, catch, continue, debugger, default, delete, do, else, false, finally, for, function, if, in, instanceof, new, null, return, switch, this, throw, true, try, typeof, var, void, while, with},
  morecomment=[l]{//},
  morecomment=[s]{/*}{*/},
  morestring=[b]',
  morestring=[b]",
  ndkeywords={class, export, boolean, throw, implements, import, this},
  keywordstyle=\color{blue}\bfseries,
  ndkeywordstyle=\color{darkgray}\bfseries,
  identifierstyle=\color{black},
  commentstyle=\color{purple}\ttfamily,
  stringstyle=\color{red}\ttfamily,
  sensitive=true
}
\lstdefinelanguage{MyBasic}{
	keywords={FWD, BACK, TL, TR, SET, PROC, END, CALL, REP},
	ndkeywords={},
	keywordstyle=\color{blue}\bfseries,
  ndkeywordstyle=\color{darkgray}\bfseries,
	sensitive=true
}
\lstset{
   backgroundcolor=\color{lightgray},
   extendedchars=true,
   basicstyle=\footnotesize\ttfamily,
   showstringspaces=false,
   showspaces=false,
   numbers=left,
   numberstyle=\footnotesize,
   numbersep=9pt,
   tabsize=2,
   breaklines=true,
   showtabs=false,
   captionpos=b
}




\begin{document}
\usetikzlibrary{quotes,angles,calc,arrows}

\tikzset{
    %Define standard arrow tip
    >=latex,
    % Define arrow style
    pil/.style={
           ->,
           thick,
           shorten <=2pt,
           shorten >=2pt,},
    line/.style={scale=3,line width=1pt},
}

\newcommand{\tikzAngleOfLine}{\tikz@AngleOfLine}                               
  \def\tikz@AngleOfLine(#1)(#2)#3{%                                            
  \pgfmathanglebetweenpoints{%                                                 
    \pgfpointanchor{#1}{center}}{%                                             
    \pgfpointanchor{#2}{center}}                                               
  \pgfmathsetmacro{#3}{\pgfmathresult}%                                        
  }  
			
																																						
\newcommand{\pairspace}{
\vspace{2 cm}
}

\newcommand{\degrees}{
\square \hspace{0.5cm} 30^\circ \newline
\square \hspace{0.5cm} 60^\circ \newline
\square \hspace{0.5cm} 90^\circ \newline
\square \hspace{0.5cm} 120^\circ \newline
\square \hspace{0.5cm} 180^\circ \newline
\square \hspace{0.5cm} 270^\circ \newline
\square \hspace{0.5cm} 330^\circ 
}

\newcommand{\degreesReflex}{
\square \hspace{0.5cm} 30^\circ \newline
\square \hspace{0.5cm} 60^\circ \newline
\square \hspace{0.5cm} 90^\circ \newline
\square \hspace{0.5cm} 120^\circ \newline
\square \hspace{0.5cm} 150^\circ \newline
\square \hspace{0.5cm} 270^\circ \newline
\square \hspace{0.5cm} 330^\circ 
}

\newcommand{\anglePaint}[2]{
	\begin{tikzpicture}[scale=3,line width=1pt, baseline=(current bounding box.north)]                                    
		\coordinate (A) at (0,0);                                                    
		\coordinate (B) at ($(A)+(#1:1)$);                                                    
		\coordinate (C) at ($(A)+(#2:1)$);                                                
		\draw (A) -- (B);                                            
		\draw (A) -- (C);                                            
		\tikzMarkAngle{(A)}{(B)}{(C)}   
	\end{tikzpicture}  
}

\newcommand{\anglePaintReflex}[2]{
	\begin{tikzpicture}[scale=3,line width=1pt, baseline=(current bounding box.north)]                                    
		\coordinate (A) at (0,0);                                                    
		\coordinate (B) at ($(A)+(#1:1)$);                                                    
		\coordinate (C) at ($(A)+(#2:1)$);                                                
		\draw (A) -- (B);                                            
		\draw (A) -- (C);                                            
		\draw (#1:0.25) arc (#1:#2:0.25);
	\end{tikzpicture}  
}

\newcommand{\tikzMarkAngle}[3]{                                                
\tikzAngleOfLine#1#2{\AngleStart}                                              
\tikzAngleOfLine#1#3{\AngleEnd}                                                
\draw #1+(\AngleStart:0.15cm) arc (\AngleStart:\AngleEnd:0.15cm);              
}   

\newcommand{\settings}{
	scale=3,line width=1pt
}

%Start of test

{\Large Use the robots to draw these paths}
\begin{questions}

	\question
		\begin{tikzpicture}[line]
			\draw (0,0) -- (90:1);
			\draw (0,0) -- (180:1);
		\end{tikzpicture}

	\pairspace

	\question
		\begin{tikzpicture}[line]                                    
			\coordinate (A) at (0,0);                                                    
			\coordinate (B) at ($(A)+(45:1)$);                                                    
			\coordinate (C) at ($(A)+(180:1)$);                                                
			\draw (A) -- (B);                                            
			\draw (A) -- (C);                                            
			\draw (45:0.25) arc (45:180:0.25);
			\node at ($(A)+(135:0.5)$) {$135^\circ$};
		\end{tikzpicture}  

	\pairspace

	\question
		\begin{tikzpicture}[line]
			\draw (0,0) -- (135:1);
			\draw (0,0) -- (180:1);
		\end{tikzpicture}

	\pairspace

	\question
		\begin{tikzpicture}[line]
			\draw (0,0) -- (10:1);
			\draw (0,0) -- (180:1);
		\end{tikzpicture}
	
	\pairspace

	\question
		\begin{tikzpicture}[line]
			\draw (0,0) -- (170:1);
			\draw (0,0) -- (180:1);
		\end{tikzpicture}
	
	\pairspace

	\question
		\begin{tikzpicture}[line]
			\draw (0,0) -- (1,0);
			\draw[shift={(1,0)}] (0,0) -- (135:1);
			\draw[shift={($(1,0)+(135:1)$)}] (0,0) -- (1,0);
		\end{tikzpicture}
		
	\pairspace

	\question
		\begin{tikzpicture}[line]
			\draw (0,0) -- (1,0);
			\draw[shift={(1,0)}] (0,0) -- (40:1);
			\draw[shift={($(1,0)+(40:1)$)}] (0,0) -- (-2,0);
		\end{tikzpicture}
		
	\pairspace

	\question
		\begin{tikzpicture}[line]
			\draw (0,0) -- (1,0);
			\draw[shift={(1,0)}] (0,0) -- (120:1);
			\draw[shift={($(1,0)+(120:1)$)}] (0,0) -- (30:1);
		\end{tikzpicture}
		
	\pairspace

	\question
		\begin{tikzpicture}[line]
			\draw (0,0) -- (1,0);
			\draw (1,0) -- (1,1);
			\draw (1,1) -- (2,1);
			\draw (2,1) -- (2,2);
		\end{tikzpicture}
		
		\pairspace

	\question
		\begin{tikzpicture}[line]
			\draw (0,0) -- (1,0) -- (1,1) -- (0,1) -- cycle;
		\end{tikzpicture}


\end{questions}
\end{document}
