\section{Conclusion}
This project serves as an introduction into the educational world for the \chirp team at IDI.
We have performed a literature review to understand the educational robotics state of the art with a focus on mathematics. 
We have also given an introduction to the pedagogical aspects, and the research methodologies needed when conducting such research.
We created a test to measure knowledge in the area of angle estimation, calculation and understanding. (The test was found to be reliable and valid. )`?
We have created chassis extensions for the robot and a bluetooth module platform which can be used for several other components as well.
We have created a tablet application to control the robot with, but also serves as a simulator. This application was created with a focus on expandability and cross platform functionality for use in future research.

We have investigated how a robot can be used to teach mathematics to school students. We have also found strengths and weaknesses of using a robot compared to a simulator. 

We have established a relationship between IDI and the international school in Trondheim which can help other researchers when expanding on our project. In addition our project serves as an introduction in itself for researchers wanting to look into educational robotics. 

The robotic activities did help students content knowledge about angles, and did aid in their understanding and ability to explain what an angle is. However the difference from a simulator was not statistically significant.