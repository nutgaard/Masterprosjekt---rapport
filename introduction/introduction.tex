\chapter{Introduction}
Schools have problems with keeping students engaged and motivated in maths lessons. They also perform poorly to a large extent. The robotics business is growing all the time and robotics in education has been shown to have great promise. The focus of this project is to investigate the potential of robotics in schools and expand the robotics team at IDI into the world of education. This thesis serves as an introduction into the school world for the robotics team at IDI. A lot of groundwork had to be done in order to carry out this experiment. Thus there has been a focus throughout this project on how it can be made applicable to future work.

\bigskip\noindent
A lot of research has been done to show that robotics can be beneficial in an educational setting. 
Many mention Papert, but fail to explain how their approach relates to his theories of learning.
They merely say that Papert was an influence or that they use experimental design, but this is only a small part of Papert's vision. 
Another common sight while reading about educational robotics is lack of research regarding the usefulness of physical robots in itself. 
While still basing our approach on Papert's theories of mental model building. We want to test whether relatability and knowledge gains can be improved by using a physical robot compared to using a simulator confined to a local device.

\bigskip\noindent
What we are doing is the first thesis in this area in the team and that is hard. We are breaking new ground for the team, establishing collaborations and providing the basis for future work. What we have implemented is designed in a way for other students to expand on. We are taking the first step. It is always the hardest. There is also help in the literature but limited in this area. 

All we have done leads up to a pilot test at THIS. We successfully created contact with them and will hopefully create a cooperation with the ChIRP group for future research. 

%\section{Context} %"'Yeah, I changed the name of this"' - Nicklas


	

\section{Motivation}
The majority of research into educational robotics shows that gains in content knowledge can be achieved using robotics. 
%Robotics are proven to be good and sometimes bad. 
However the benefits of a robot in education discussed in several papers often limit themselves to only talking about superficial aspects like fun, engagement, motivation and cooperation. 
%We want to take a closer look at why, and one of the major issues identified during a literature review was the lack of research into simulated environments. 
We want to take a closer look into the pedagogic aspects of mental model building (intellectual model building) and investigate the differences in using a physical robot compared to a simulator which is significantly cheaper to develop, buy and maintain.% a robot is better (if that is the case) than for example a simulator which is significantly cheaper to develop, buy and maintain. 

\bigskip\noindent
We find that besides mental model building which might be improved by using robots the learning experience will most definitely be new and interesting. We see this as an important aspect of the robotics experience, and therefore aim to augment our investigation to include questions that probe for attitude changes. 

%\bigskip\noindent
%Present our literature study findings. Should literature study come before the task itself?
%Nah, blir ikke det feil?

\section{Project Goals}
The goal of this project is to investigate the use of robotics in schools for the ChIRP group. We will implement an intuitive way to control the robot with a solid foundation in pedagogic theory, make necessary alterations to the ChIRP and test our assumptions on school children. There will not be enough time to fully test this system but it will serve as a platform for others to expand on in the future. 

We will also try to document our design choices and break down the pedagogic theories and relating work through a literature study to support our choices and claims to create a solid platform. 

We wish to contribute to the discussion of educational robotics both through our software implementations but also the conclusions we draw on robot vs simulator. In this way we intend to further the research in this area. 

\section{Report Structure}
This report is split into six chapters before including a set of appendices for any information which may be relevant to the reader.

\bigskip\noindent
\textbf{Chapter~\ref{part:introduction}} will provide a short introduction to the project, including motivation behind conduction the project, reseach question and the different phases of the project.

\bigskip\noindent
\textbf{Chapter~\ref{part:literature}} is dedicated to the literature review and process of conducting the literature review. 
%will the pedagogical theories used throughout the project, the process of doing the literature review, and the current state of the art for educational robotics. 

\bigskip\noindent
\textbf{Chapter~\ref{part:development}} presents our design principles and a discussion related to other software solution, before presenting the system created during this project. 

\bigskip\noindent
\textbf{Chapter~\ref{part:method}} presents our experimental setup, the procedures used, and a discussion around the reliability and validity of the study.

\bigskip\noindent
\textbf{Chapter~\ref{part:results}} provides an overview of all the results and analyses from the experiment.

\bigskip\noindent
\textbf{Chapter~\ref{part:discussion}} is the culmination of this report, and presents a discussion of the whole project before presenting our conclusion and proposals for future work. 


