%	\chapter{Introduction}%Moved to master file instead since anchors are located there
After the technology boom in the 20th century, technology have become extremely important, and thus so have science, engineering and mathematics. It is therefore vital to adapt a continuous effort to improving the education provided within these fields. However from the year 2000 up to 2012, the PISA tests showed a decrease in the average score on the mathematics test (from 500 to 494) overall in Europe~\cite{OECDPISA2000, OECDPISA2012}. This is also visible in the Norwegian results, indicating a decrease from 499 in 2000 to 489 in 2012 within mathematics and generally a lower score then our neighbouring countries~\cite{OECDPISA}.
%Schools have problems with keeping students engaged and motivated in courses related to science, and especially in mathematics. 
%Norwegian students also perform poorly compared the students in the neighboring countries in these courses~\cite{OECDPISA}.
%They also perform poorly to a large extent. 

\bigskip\noindent
In 1980 Seymour Papert published the book ``Mindstorms: Children, computers, and powerful ideas''\cite{papert1980mindstorms}, where his ideas of an alternative 
learning environment, which utilizes robotics were presented. It was Papert's belief that educational robotics held a major potential improvement of the current learning environment.
Allowing children to interact and construct their their own knowledge in ways previously impossible. 
This book was the start of educational robotics and many researchers rely on the early work conducted by Seymour Papert within educational robotics. However, they often fail to explain how Papert's theories are applied within their own research. 

\bigskip\noindent
In the later years robotics has seen a huge increase within domestic use, and have now become an affordable tool in the daily life of most people~\cite{kara2003sizing,hsiu2003designing}.
A lot of research has been done to show that robotics can be beneficial in an educational setting.
The majority of research into educational robotics shows that gains in content knowledge can be achieved using robotics.

%Many mention Papert, but fail to explain how their approach relates to his theories of learning.
%They merely state that Papert was an influence or that they use experimental design inspired by Papert's work. %, but this is only a small part of Papert's vision. 

%\bigskip\noindent
%Removed due to comment from Pauline
%What we have implemented is designed in a way for other students to expand on. We are taking the first step. It is always the hardest. 
%There is also help in the literature but limited in this area. 
%We successfully created contact with them and will hopefully create a cooperation with the ChIRP group for future research. 
%\section{Context} %"'Yeah, I changed the name of this"' - Nicklas
\section{Motivation and goals}
	In the later years robotics has seen a huge increase within domestic use, and have now become an affordable tool in the daily life of most people~\cite{kara2003sizing,hsiu2003designing}.
	Yet robotics has only seen minimal use within an educational setting. Some researchers suggest that this may be due to the lack of empirical evidence
	supporting benefits of educational robotics~\cite{williams2007acquisition}. 
	
\bigskip\noindent
In 1980 Seymour Papert published the book ``Mindstorms: Children, computers, and powerful ideas''\cite{papert1980mindstorms}, where his ideas of a constructionismistic 
learning environment and robotics were presented. It was Papert's belief that educational robotics held a major potential improvement of the current learning environment.
Allowing children to interact and construct their their own knowledge in ways previously impossible. 

\bigskip\noindent
Even though educational theorists believed robotics could be utilized in a learning environment with success, there has been little
incorporation seen throughout the world. 
Some speculate that the limited adoption is due to lack of empirical evidence for the effect of robotics as a learning tool~\cite{williams2007acquisition}.  
Another possibility is that the usage of robotics in education usually has been as a tool to teach students about robotics itself, 
and thus have formed a narrow field of applicability~\cite{rusk2008new}. 
The third possibility comes down to the price of robotics equipment, e.g a lego mindstorm kit costs about 650 USD. 
An interesting field of research is therefore to look at how robotics can be utilized to teach about non-robotics subjects, and perhaps even be used as a motivational or attitude changing tool while still keeping the cost down. 

	

\section{Motivation}
The majority of research into educational robotics shows that gains in content knowledge can be achieved using robotics. 
%Robotics are proven to be good and sometimes bad. 
However the benefits of a robot in education discussed in several papers often limit themselves to only talking about superficial aspects like fun, engagement, motivation and cooperation. 
%We want to take a closer look at why, and one of the major issues identified during a literature review was the lack of research into simulated environments. 
We want to take a closer look into the pedagogic aspects of mental model building (intellectual model building) and investigate the differences in using a physical robot compared to a simulator which is significantly cheaper to develop, buy and maintain.% a robot is better (if that is the case) than for example a simulator which is significantly cheaper to develop, buy and maintain. 

\bigskip\noindent
We find that besides mental model building which might be improved by using robots the learning experience will most definitely be new and interesting. We see this as an important aspect of the robotics experience, and therefore aim to augment our investigation to include questions that probe for attitude changes. 

%\bigskip\noindent
%Present our literature study findings. Should literature study come before the task itself?
%Nah, blir ikke det feil?

\section{Project Goals}
The goal of this project is to investigate the use of robotics in schools for the ChIRP group. We aim to implement an intuitive way to control the robot with a solid foundation in pedagogic theory. This will require customizations to the ChIRP robot and to test our assumptions with school children. There will not be enough time to fully test this system but it will serve as a platform, provide guidelines and pitfalls for others to expand on in the future. 

%\bigskip\noindent
%We will also try to document our design choices and break down the pedagogic theories and relating work through a literature study to support our choices and claims to create a solid platform. 
\bigskip\noindent
In the end we wish to contribute to the discussion of educational robotics both through our software implementations but also the conclusions we draw on robot versus simulator. In this way we intend to further the research in this area. 

%\bigskip\noindent
%The end goa	l of the project is to provide more empirical evidence regarding the effectiveness of educational robotics, and perhaps an alternative solution to using expensive and commercial robots. 

\chapter{Approach}
\chapter{Research questions}
We have found a couple of research questions through an initial literature study. We propose to create an intuitive platform for learning math with robotics that take the emphasis of programming.
\section{Execution of project}
This entire project was conducted from February to July 2014. During this time period we've created a pilot-project looking into the use of robotics in an educational setting with a focus on angles and how these can be taught.
%This chapter will provide a walkthrough of the different phases of this project. 
%
%This chapter will highlight some of the tasks needed to conduct this project. 
%Everything we have done has been done since february. This made our process long and intense. We have conducted a pre-project to find a relevant angle on robotics in education and then we have focused further into the specific field of robotics in maths education.

\paragraph{Literature Study on robotics in education}~\\	
In order to get an idea of the state of the art of robotics in education we initially performed a literature study. The results from this study can be read in chapter~\ref{ch:literatureProcess} through \ref{ch:literatureConclusion}.
%We also want to get an idea of what way we can introduce robotics to a school.

\bigskip\noindent
In general the literature study showed great promise for educational robotics. However few papers have collaborated with schools and pedagogs in an attempt to integrate or check how their ideas will integrate into the school curriculum. Many mention that educational robotics requires a change in the traditional curriculum in order to be more consistent with the constructionism theories. This does, in our opinion, pose one of the greatest barriers for educational robotics.
%However calling for a huge change in schools is probably not the easiest way to introduce robotics to schools if this require lots of alterations. 

\bigskip\noindent
One major denominator in the current research is the lack of details in publications. While general trends and statistics are included in the majority of publications, we still found that the discussions related to validity and reliability often were excluded. This made our process of determining the current state of educational robotics extremely difficult. 
%Another lack in research is an explanation of how they taught the students. They may mention that they implemented a 1 year curriculum but don't provide any examples of tasks they created. It is therefore very hard to build upon and do further work on it. 


\paragraph{Deciding on a specific topic}\label{sec:decideTopic}~\\
The applicability of robotics in an educational setting has proven itself over and over again. Thus, in order to narrow down the project, we chose to focus on mathematics and especially angles. 
%A lot of topics can be learned using robotics. We have focused on mathematics. It is said that all math can be taught in some way with robotics. We present examples for some of them.

\bigskip\noindent
We have chosen to focus on angles as it is one of the core concepts within mathematics, people tend to struggle with it, and visualization of angles is easily achieved with the help of a robot. %with a robot can be achieved without to much trouble. 
%We have chosen to focus on angles as it is very basic, people struggle with it and using logo uses angles to decide how much to turn. It is therefore a natural starting point for our investigation in mathematics with robotics. 

\bigskip\noindent
As a pedagogical foundation we looked to research conducted by Seymour Papert, who is known for his early contributions to the field of educational robotics.
%We also draw pedagogical reasoning behind angles and geometry from papert. The reason angles and geometry in general is easier is that a point rather than being defined as a place in space with no heading the robot that serves as the point here has a heading just like us humans. It can therefore be related to. 

%\section{Literature Study on available programming languages and pedagogy}
\paragraph{Secondary Literature Study}~\\
After deciding to investige the concept of angles we identified the need for a secondary literature study. 
This study was conducted in order to learn more about the current pedagogical theories and test setups, which would help us find a suitable approach to create a solid platform for this project and potensial successors.

%Avsnittet virker som om det handler mer om det første literature studiet.
%In order to focus on the right points, learn about the pedagogic theory and testing setup, use a suitable approach and create a solid platform to work on we needed to conduct a literature review to find state of the art on educational robotics as well as pedagogic theory. 

\paragraph{Design and Development}~\\
In order to conduct this project we needed to develop several accessories for the \chirp and a tablet application for testing in an elementary school. 
Since this was considered a pilot-project for the \chirp robot we aimed at designing and developing a prototype with emphasis on expandability, modifiability and portability. 
%Vet ikke helt hvordan denne skal uttrykkes, eller den i det hele tatt bør bli med?
%We will find workarounds for the chirp that does not require major modifications to the chassis or default circuit board with sensors. 
We also needed to develop a small circuit to enable bluetooth communication between the robot and tablet application. 
All in all there was a lot of work to be done before a prototype would be ready for the experiment. 
%Tror ikke vi trenger å skrive så mye begrunnelser her, siden det meste kommer til å bli nevnt i design and development kapitlet. 
%as we found that compiling and uploading code to the ChIRP is not suitable for every school grade and we don't know which grade we will be working with, another point is that this application again should be easily expandable and implementable for different grade students.  

%Igjen, tror ikke vi trenger begrunnelse her. 
%\bigskip\noindent
%The design and development of this software and hardware will be a very demanding phase. Designing a new application from scratch can be very time-consuming([39] - Another master %.Mention hardware too REFERANSE). However we build upon LOGO so we have some guidance to what the functionality should entail. 

%Again we will not have time to create an optimal solution as this would require testing and then redesign etc. We aim to create an easily manageable and changeable platform to build upon in future research. 

\paragraph{Experiment}~\\
To answer our research questions we had to test the robot in a school context. We contacted several schools and had meetings with Ungt Entreprenørskap and other researchers at IDI responsible for attracting students to study at IDI. We did not get much response, but eventually we got in contact with the international school in Trondheim. This is where we conducted our experiments. We had 2 meetings with the teacher of the class we were assigned. In these meetings we discussed our tests and curriculum and got feedback which we used to improve our experiment. It was important for us to include teachers as the end goal is to integrate the robotic activities to normal schools.

\bigskip\noindent
The experiments were run at the end of the school year because of the long process of finding a school and developing the system. This resulted in a smaller experimental time span than initally intended.


%Seems better here, this way we get the structure at the end of chapter 1
\section{Report Structure}
This report is split into six chapters before including a set of appendices for any information which may be relevant to the reader.

\bigskip\noindent
\textbf{Chapter~\ref{part:introduction}} will provide a short introduction to the project, including motivation behind conduction the project, reseach question and the different phases of the project.

\bigskip\noindent
\textbf{Chapter~\ref{part:literature}} is dedicated to the literature review and process of conducting the literature review. 
%will the pedagogical theories used throughout the project, the process of doing the literature review, and the current state of the art for educational robotics. 

\bigskip\noindent
\textbf{Chapter~\ref{part:development}} presents our design principles and a discussion related to other software solution, before presenting the system created during this project. 

\bigskip\noindent
\textbf{Chapter~\ref{part:method}} presents our experimental setup, the procedures used, and a discussion around the reliability and validity of the study.

\bigskip\noindent
\textbf{Chapter~\ref{part:results}} provides an overview of all the results and analyses from the experiment.

\bigskip\noindent
\textbf{Chapter~\ref{part:discussion}} is the culmination of this report, and presents a discussion of the whole project before presenting our conclusion and proposals for future work. 


