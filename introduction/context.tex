\section{Context}
	In the later years robotics has seen a huge increase within domestic use, and have now become an affordable tool in the daily life of most people~\cite{kara2003sizing,hsiu2003designing}.
	Yet robotics has only seen minimal use within an educational setting. Some researchers suggest that this may be due to the lack of empirical evidence
	supporting benefits of educational robotics~\cite{williams2007acquisition}. 
	
\bigskip\noindent
In 1980 Seymour Papert published the book ``Mindstorms: Children, computers, and powerful ideas''\cite{papert1980mindstorms}, where his ideas of a constructionismistic 
learning environment and robotics were presented. It was Papert's belief that educational robotics held a major potential improvement of the current learning environment.
Allowing children to interact and construct their their own knowledge in way previously impossible. 

\bigskip\noindent
Even though educational theorists believed robotics could be utilized in a learning environment with success, there has been little
incorporation seen throughout the world. 
Some speculate that the limited adoption is due to lack of empirical evidence for the effect of robotics as a learning tool~\cite{williams2007acquisition}.  
Another possibility is that the usage of robotics in education usually has been as a tool to teach students about robotics itself, 
and thus have formed a narrow field of applicability~\cite{rusk2008new}. 
The third possibility comes down to the price of robotics equipment, e.g a lego mindstorm kit costs about 650 USD. 
An interesting field of research is therefore to look at how robotics can be utilized to teach about non-robotics subjects, and perhaps even be used as a motivational or attitude changing tool while still keeping the cost down. 

	
