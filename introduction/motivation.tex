%\section{Motivation}%Merged with previous section because of comment from Pauline
% a robot is better (if that is the case) than for example a simulator which is significantly cheaper to develop, buy and maintain. 

%Commented out because of a comment from Pauline. Should maybe be rewritten instead.
%\bigskip\noindent
%We find that besides mental model building which might be improved by using robots the learning experience will most definitely be new and interesting. We see this as an important aspect of the robotics experience, and therefore aim to augment our investigation to include questions that probe for attitude changes. 

%\bigskip\noindent
%Present our literature study findings. Should literature study come before the task itself?
%Nah, blir ikke det feil?
\section{Motivation and goals}
Educational theorists believed robotics could be utilized in a learning environment with success, but there has been little
incorporation seen throughout the world. 
Some speculate that the limited adoption is due to lack of empirical evidence for the effect of robotics as a learning tool~\cite{williams2007acquisition}.  
Another possibility is that the usage of robotics in education usually has been as a tool to teach students about robotics itself, 
and thus have formed a narrow field of applicability~\cite{rusk2008new}. 
The third possibility comes down to the price of robotics equipment, e.g a lego mindstorm kit costs about 650 USD. 
An interesting field of research is therefore to look at how robotics can be utilized to teach about non-robotics subjects, and perhaps even be used as a motivational or attitude changing tool while still keeping the cost down. 

\bigskip\noindent
Another common sight while reading about educational robotics is a lack of research regarding the usefulness of a physical robots in itself, as it seems like most people made the jump from traditional teaching to teaching with a robot in one big jump. 
Several papers also limit themselves by only targeting superficial aspects like fun, engagement, motivation and cooperation. 
While still basing our approach on Papert's pedagogical theories, we will try to fill some of these holes. We want to test whether relatability and knowledge gains can be improved by using a physical robot compared to using a simulator confined to a local device. 
We want to take a closer look into the pedagogic aspects of mental model building and investigate the differences in using a physical robot compared to a simulator which is significantly cheaper to develop, buy and maintain.
%Robotics are proven to be good and sometimes bad. 

%We want to take a closer look at why, and one of the major issues identified during a literature review was the lack of research into simulated environments. 

\bigskip\noindent
Since the robotics business is growing all the time and educational robotics has been shown to have great promise we focus this project on an investigation into the potential of robotics in schools and make an effort to expand the \chirp team's horizon into the world of education. This thesis serves as an introduction into the educational world for the \chirp team at IDI. We are breaking new ground for the team, establishing collaborations and providing the basis for future work. 
Because of this we've designed our system implementation and report in such a way that others can learn from and expand upon it. 