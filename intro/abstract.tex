\section*{Abstract}
In the later years robotics has seen a huge increase within domestic use, and have now become an affordable tool in the daily life of most people.
The goal of this project was to investigate the differences between a physical and virtual robot in terms of increased content knowledge, learning motivation, and interest in science, technology, engineering and mathematics (STEM).
	%The goal of this thesis is to look into how robotics can be combined with the early education of children in K-12, in order to increase content knowledge, motivate learning and generate more interest within the STEM fields.
To investigate this we conducted an experiment at Trondheim'm International School (THIS), using a quasi-experimental setup with two treatment group (virtual and physical robot) .
	%To investigate the applicability of robotics an experiment was conducted at Trondheim's International School (THIS) 
The results showed that there does not exist a statistically significant difference in content knowledge gain, motivation or interest between the groups. 

\newpage
\section*{Sammendrag}
 De siste årene har roboter økt i popularitet innenfor vanlige hushold, og har kommet ned på ett prisnivå som gjør robotene tilgjengelig for folk flest.
Målet med denne oppgaven var å undersøke forskjellene mellom en fysisk og virtuell robot med tanke på å øke kunnskapsnivået, motivasjonen og interessen for vitenskap, teknologi, ingeniørskap og matematikk (STEM).
For å undersøke dette utførte we ett eksperiment hos Trondheim's International School (THIS), hvor vi brukte ett kvasieksperimentelt oppsett med to behandlingsgrupper (virtuell og fysisk robot).
Resultatene viste at det ikke fantes noen statistisk signifikant forskjell i økning av kunnskapsnivå, motivasjon eller interesse mellom de to gruppene.